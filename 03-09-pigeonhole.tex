\documentclass{hippoidC}

\memoto{Idris}
\memosubject{Book of Proof}
\memodate{2024.03.13}
\status{\S 3.9 Division and Pigeonhole Principles}

\newcommand{\up}[1]{ \left\lceil#1\right\rceil }
\newcommand{\down}[1]{ \left\lfloor#1\right\rfloor}

\begin{document}
\toc
\thispagestyle{fancy}
\pagebreak

\begin{prooflist}{1. Show that if six integers are chosen at random, then at
		least two of them will have the same remainder when divided by 5.}
	\item Let $A=\{0, 1, 2, 3, 4\}$, and for any $i\in\mathbb{Z}, i\mod 5 \in
		A$.
	\item The division principle states that for $n, k\in \mathbb{N}$, if $n$
	objects are placed in $k$ slots, then at least one box contains
	$\up{\dfrac{n}{k}}$ or more objects. Here $|A|=k=5$, and $n=6$.
	\item Therefore at least $\up{\dfrac{n}{k}}=\up{\dfrac{6}{5}}=2$
	two of the randomly chosen integers mod 5 are the same value.
\end{prooflist}

\begin{prooflist}{2. You deal a pile of cards, face down, from a standard
		52-card deck. What is the least number of cards the pile must have before
		you can be assured that it contains at least five cards of the same suit?}
	\item The division principle states that for $n, k\in \mathbb{N}$, if $n$
	objects are placed in $k$ slots, then at least one box contains
	$\up{\dfrac{n}{k}}$ or more objects
	\item Each suit is a slot. We need to calculate the number of cards needed
	$n$ such that $\up{\dfrac{n}{k}}>=5$.
	\item Because $k=4$, then $n=17$.
\end{prooflist}

\begin{prooflist}{3. What is the fewest number of times you must roll a
		six-sided dice before you can be assured that 10 or more of the rolls
		resulted in the same number?}
	\item The division principle states that for $n, k\in \mathbb{N}$, if $n$
	objects are placed in $k$ slots, then at least one box contains
	$\up{\dfrac{n}{k}}$ or more objects
	\item Each dice-face is a slot. We need to calculate the number of rolls needed
	$n$ such that $\up{\dfrac{n}{k}}>=10$.
	\item Because $k=6$, then $n=k\cdot(10 -1)+ 1 = 55$.
\end{prooflist}

\begin{prooflist}{4. Select any five points on a square whose side-length is one
		unit. Show that at least two of these points are within $\frac{\sqrt{2}}{2}$
		units of each other.}
	\item Divide the square into four equal subsquares with side length
	$\frac{1}{2}$, and notice that the diagonal of each subsquare has a length
	of $\frac{\sqrt{2}}{2}$.
	\item Since we have 5 points and only 4 subsquares, at least two points must lie
	in the same subsquare by the pigeonhole principle.
	\item By the nature of the subsquare, the maximum distance between any two
	points within the same subsquare is the diagonal length, which is
	$\frac{\sqrt{2}}{2}$.
	\item Therefore, at least two of the five points are within $\frac{\sqrt{2}}{2}$
	units of each other.
\end{prooflist}

\begin{prooflist}{5. Prove that any set of seven distinct natural numbers
		contains a pair of numbers whose sum or difference is divisible by 10.}
	\item Consider the final digit $d$ of some $a \in \mathbb{N}$.
	\item $d$ will be in 0..9.
\end{prooflist}

\begin{prooflist}{6. Given a sphere S, a great circle of S is the intersection
		of S with a plane through its center. Every great circle divides S into two
		parts. A hemisphere is the union of the great circle and one of these two parts.
		Show that if five points are placed arbitrarily on S, then there is a hemisphere
		that contains four of them.}
	\item Use two points to define a great circle, which also creates two
	hemispheres.
	\item Three points remain, which by the pigeonhole principle means, one
	hemisphere has two points of those points, and the other has one.
	\item The points used to construct the great circle can be considered part of
	the hemisphere with 2 points, therefore that hemisphere has 4 points.
\end{prooflist}

\end{document}
