\documentclass{hippoidC}

\memoto{Idris}
\memosubject{Book of Proof}
\memodate{2024.03.22}
\status{\S 12.1 Functions}

\begin{document}
\toc
\thispagestyle{styleTOC}
\pagebreak
\pagestyle{styleE}

\begin{prooflist}{1. Suppose $A=\{0,1,2,3,4\}, B=\{2,3,4,5\}$ and\\
		$f=\{(0,3),(1,3),(2,4),(3,2),(4,2)\}$. State the domain and range of $f$.
		Find $f(2)$ and $f(1)$.}
	\item Domain is $A$.
	\item Rance is $\set{3, 4, 2}$.
	\item $f(2) = 4$.
	\item $f(1) = 3$.
\end{prooflist}

\begin{prooflist}{2. Suppose $A=\{a, b, c, d\}, B=\{2,3,4,5,6\}$ and\\
		$f=\{(a, 2),(b, 3),(c, 4),(d, 5)\}$. State the domain and range of $f$. Find
		$f(b)$ and $f(d)$.}
	\item Domain is $A$.
	\item Range is $\set{2, 3, 4, 5}$.
	\item $f(b) = 3$.
	\item $f(d) = 5$.
\end{prooflist}

\begin{prooflist}{3. There are four different functions $f:\set{a, b}
			\rightarrow\set{0,1}$. List them. Diagrams suffice.}
	\item $f_1 = \set{(a, 0), (b, 0)}$
	\item $f_2 = \set{(a, 0), (b, 1)}$
	\item $f_3 = \set{(a, 1), (b, 0)}$
	\item $f_4 = \set{(a, 1), (b, 1)}$
\end{prooflist}

\begin{prooflist}{4. There are eight different functions $f:\{a, b, c\}
			\rightarrow\{0,1\}$. List them. Diagrams suffice.}
	\item $f_1 = \set{(a, 0), (b, 0), (c, 0)}$
	\item $f_2 = \set{(a, 0), (b, 0), (c, 1)}$
	\item $f_3 = \set{(a, 0), (b, 1), (c, 0)}$
	\item $f_4 = \set{(a, 0), (b, 1), (c, 1)}$
	\item $f_5 = \set{(a, 1), (b, 0), (c, 0)}$
	\item $f_6 = \set{(a, 1), (b, 0), (c, 1)}$
	\item $f_7 = \set{(a, 1), (b, 1), (c, 0)}$
	\item $f_8 = \set{(a, 1), (b, 1), (c, 1)}$
\end{prooflist}

\begin{prooflist}{5. Give an example of a relation from $\set{a, b, c, d}$ to
		$\{d, e\}$ that is not a function.}
	\item The relation $\set{(a, d), (a, e)}$ is not a function because it does not
	adhere to the property that for every $a \in A$, the relation $f$ contains
	exactly one pair of the form $(a, b)$.
\end{prooflist}

\begin{prooflist}{6. Suppose $f: \mathbb{Z} \rightarrow \mathbb{Z}$ is defined
		as $f=\{(x, 4 x+5): x \in \mathbb{Z}\}$. State the domain, codomain and
		range of $f$. Find $f(10)$.}
	\item The domain of $f$ is $\mathbb{Z}$.
	\item The codomain of $f$ is $\mathbb{Z}$.
	\item The range of $f$ is $\mathbb{Z}$.
	\item $f(10)=45$.
\end{prooflist}

\begin{prooflist}{7. Consider the set $f=\{(x, y) \in \mathbb{Z} \times
			\mathbb{Z}: 3 x+y=4\}$. Is this a function from $\mathbb{Z}$ to $\mathbb{Z}$
		? Explain.}
	\item No this is not a function from $\mathbb{Z}\rightarrow\mathbb{Z}$, because
	it is a function $\mathbb{Z}\times\mathbb{Z}\rightarrow\mathbb{Z}$.
\end{prooflist}

\begin{prooflist}{8. Consider the set $f=\{(x, y) \in \mathbb{Z} \times
			\mathbb{Z}: x+3 y=4\}$. Is this a function from $\mathbb{Z}$ to $\mathbb{Z}$
		? Explain.}
	\item No this is not a function from $\mathbb{Z}\rightarrow\mathbb{Z}$, because
	it is a function $\mathbb{Z}\times\mathbb{Z}\rightarrow\mathbb{Z}$.
\end{prooflist}

\begin{prooflist}{9. Consider the set $f=\set{(x^2, x: x \in
				\mathbb{R}}$. Is this a function from $\mathbb{R}$ to $\mathbb{R}$?}
	\item This is a function from $\mathbb{R}$ to $\mathbb{R}$. If $x\in \mathbb{R}$, then
	$x^2 \in \mathbb{R}$. Therefore both the domain and codomain of the function
	is $\mathbb{R}$.
\end{prooflist}

\begin{prooflist}{10. Consider the set $f=\set{(x^3, x): x \in
				\mathbb{R}}$. Is this a function from $\mathbb{R}$ to $\mathbb{R}$ ?
		Explain.}
	\item This is a function from $\mathbb{R}$ to $\mathbb{R}$. If $x\in \mathbb{R}$, then
	$x^3 \in \mathbb{R}$. Therefore both the domain and codomain of the function
	is $\mathbb{R}$.
\end{prooflist}

\begin{prooflist}{11. Is the set $\theta=\left\{(X,|X|): X \subseteq
			\mathbb{Z}_5\right\}$ a function? If so, what is its domain and range?}
	\item
	The set $\theta=\left\{(X,|X|): X \subseteq \mathbb{Z}_5\right\}$ is a
	function, because there is only one relation of the form $(a, f(b))$ for
	each $a\in X$.
	\item The domain is the set $X$, and the range is the set $\set{a\in X\mid a\geq
			0}$.

\end{prooflist}

\begin{prooflist}{12. Is the set $\theta=\{((x, y),(3 y, 2 x, x+y)): x, y \in
			\mathbb{R}\}$ a function? If so, what is its domain and range? What can be
		said about the codomain?}
	\item Yes it is a function, because each input $(x, y)\mid x, y\in\mathbb{R}$
	maps to just one value $f(x, y)$.
	\item The domain is $\mathbb{R}\times\mathbb{R}$, and the range and codomain are
	$\mathbb{R}\times\mathbb{R}\times\mathbb{R}$.
\end{prooflist}

\end{document}
