\documentclass{idrisMemo}

\usepackage{amsthm}
\usepackage{amsfonts}
\usepackage{hyperref}
\usepackage{enumitem}
\usepackage{amssymb}
\usepackage{graphicx}

\usepackage{tocloft} % Include the package

\memoto{Idris}
\memosubject{Book of Proof}
\memodate{2024.03.13}
\status{\S 3.10 Combinatorial Proofs}

\newcounter{exercise} % This defines a new counter named 'exercise'
\newcommand{\exercise}[1]{\refstepcounter{exercise} \noindent Exercise: #1
    \addcontentsline{toc}{subsection}{#1}} % Use 'section' level for TOC

\newenvironment{prooflist}[1]
    {
    \pagebreak
    \exercise{#1}
    \begin{itemize}[label={}, leftmargin=1mm, itemsep=0.5mm]}
   {\end{itemize} {\hfill $\square$} }

\begin{document}

\tableofcontents
\pagebreak

\begin{prooflist}{1. Show that
$$1(n-0)+2(n-1)+3(n-2)+4(n-3)+\dots+(n-1)2+(n-0)1 = \binom{n+2}{3}$$ .}
\item The expression $\binom{n+2}{3}$ is equivalent to the ways we can select
    size-3 subsets from a set of size $(n+2)$.
$$
n+2n-2+3n-6+4n-12+\dots+2n-2+n = \binom{n+2}{3}
$$

$$
\binom{n+2}{3}=
\dfrac{(n+2)!}{3!(n-1)!}=
\dfrac{(n+2)(n+1)(n)(n-1)!}{3!(n-1)!}=
\dfrac{(n+2)(n+1)(n)}{6}=
$$

$$
\binom{n+2}{3} = \binom{n+1}{3} + \binom{n+1}{2}
$$
yeah, no idea. see solution
\end{prooflist}

\begin{prooflist}{2. Show that $1 + 2 + 3 + \dots + n = \binom{n+1}{2}$.}
\item Let $A=\{0, 1, ... n\}$, which has cardinality $n+1$.  From this set we'll
    choose subsets of size 2.
$$
\binom{n+1}{2} =
\dfrac{(n+1)!}{2!(n-1)!} =
\dfrac{(n+1)n(n-1)!}{2!(n-1)!} =
\dfrac{n^2+n}{2}
$$
\item We can pair together $n+0$, $(n-1) + 1$, $(n-2) + 2$, etc such that we
    have $\dfrac{n}{2}\cdot n$, and one middle element which is $\dfrac{n}{2}$.
\item Therefore we have shown that $1 + 2 + 3 + \dots + n = \binom{n+1}{2}$.
\end{prooflist}

\begin{prooflist}{3. Show that
    $$ \dbinom{n}{2} \dbinom{n-2}{k-2}= \dbinom{n}{k} \dbinom{k}{2} $$
}
\item
    $$ \dfrac{n!}{2!(n-2)!} \dfrac{(n-2)!}{(k-2)!(n-2-k+2)!}=
    \dfrac{n!}{k!(n-k)!} \dfrac{k!}{2!(k-2)!} $$
\item
    $$ \dfrac{n!}{2!} \dfrac{1}{(k-2)!(n-2-k+2)!}= \dfrac{n!}{(n-k)!} \dfrac{1}{2!(k-2)!} $$
\item
    $$ \dfrac{n!}{2(k-2)!(n-k)!}= \dfrac{n!}{(n-k)!2(k-2)!} $$
\end{prooflist}

\begin{prooflist}{4. Show that $P(n,k) = P(n-1,k)+ k \cdot P(n-1,k -1)$.}
\item
    $$ P(n,k) = \dfrac{n!}{(n-k)!}$$
    $$ P(n-1,k) =  \dfrac{(n-1)!}{(n-1-k)!}\cdot\dfrac{n-k}{n-k} =\dfrac{(n-k)(n-1)!}{(n-k)!}
    $$
    $$ k\cdot P(n-1,k -1) = \dfrac{k(n-1)!}{(n-k)!}$$
    $$ P(n-1,k)+ k \cdot P(n-1,k -1) = \dfrac{(n-k+k)(n-1)!}{(n-k)!} =
    \dfrac{n!}{(n-k)!}
    $$
\item There is surely some fancy, less grunt-mole way of solving this.
\end{prooflist}

breaking my brain, pausing this here.

\end{document}
