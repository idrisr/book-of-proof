\documentclass{idrisMemo}

\usepackage{amsthm}
\usepackage{amsfonts}
\usepackage{hyperref}
\usepackage{enumitem}
\usepackage{amssymb}
\usepackage{tocloft}
\usepackage{bookOfProof}

\memoto{Idris}
\memosubject{Book of Proof}
\memodate{2024.03.09}
\status{}

\begin{document}

\tableofcontents
\pagebreak

\begin{prooflist} {1. Suppose $n \in \mathbb{Z}$. If $n^2$ is even, then $n$ is even.}
    \item We proceed with the contrapositive, which says: if $n$ is odd, then $n^2$ is
        odd.
    \item Suppose $n$ is odd.
    \item Then we can write $n$ as $2k + 1$, for some integer $k$.
    \item Therefore, $n^2 = (2k + 1)^2 = 4k^2 + 4k + 1$.
    \item We can see that $n^2$ is odd since it leaves a remainder of 1 when divided by 2.
    \item Thus, we have proven the contrapositive of the original
        statement.
    \item Therefore, our original statement holds true: if $n^2$ is even, then
        $n$ must be even.
\end{prooflist}

\begin{prooflist} {2. Suppose $n \in \mathbb{Z}$. If $n^2$ is odd, then $n$ is odd.}
    \item We proceed with the contrapositive, which says: if $n$ is even, then $n^2$ is
        even.
    \item Suppose $n$ is even.
    \item Then we can write $n$ as $2k$, for some integer $k$.
    \item Therefore, $n^2 = (2k)^2 = 4k^2$.
    \item We can see that $n^2$ is even since it leaves a remainder of 0 when divided by 2.
    \item Thus, we have proven the contrapositive of the original
        statement.
    \item Therefore, our original statement holds true: if $n^2$ is odd, then
        $n$ must be odd.
\end{prooflist}

\begin{prooflist} {3. Suppose $a,b \in \mathbb{Z}$. If $a^2 (b^2 - 2b)$ is odd, then $a$ and $b$ are odd.}
    \item The contrapositive equivalently restates the problem as the following.
    \item If $a$ or $b$ is even, then $a^2 (b^2 - 2b)$ is even.
    \item Suppose the first case, where $a$ is even.
    \item Then we can write $a = 2k$, for some $k \in \mathbb{Z}$.
    \item Therefore, $a^2 = (2k)^2 = 2 \cdot 2k^2$, and $a$ is even.
    \item If $(b^2 - 2b)$ is odd, then $(b^2 - 2b) = 2m+1$, for some $m \in \mathbb{Z}$.
    \item Then $a^2 \cdot m = 4k^2(2m+1)= 2 \cdot [2k^2(2m+1)]$, which is even because it is
        divible by $2$.
    \item If $(b^2 - 2b)$ is even, then $(b^2 - 2b) = 2n$, for some $n \in \mathbb{Z}$.
    \item Then $a^2 \cdot m = 4k^2(2n)= 2 (2k^2n)$, which is even because it is
        divible by $2$.
    \item Therefore if $a$ is even, then $a \vee b$ is even, then $a^2 (b^2 - 2b)$ is even, proving the
        contrapositive, and the original statement.
\end{prooflist}

\begin{prooflist} {4. Suppose $a,b,c  \in \mathbb{Z}$. If $a$ does not divide $bc$,
    then $a$ does not \mbox{divide $b$}.}
    \item The original problem is the following. $$a \nmid bc \rightarrow a \nmid b$$.
    \item The contrapositive equivalently restates the problem as the following.
    \item $$a \mid b \rightarrow a \mid bc$$.
    \item Suppose $a \mid b$.
    \item Therefore $na = b$, for some $n \in \mathbb{N}$.
    \item Thus $(cn)a = bc$, and $ a \mid bc $.
    \item Thus by proving the contrapositive, we have proved the original
        statement.
\end{prooflist}

\begin{prooflist} {5. Suppose $x \in \mathbb{R}$. If $x^2 + 5x < 0$ then $x<0$.}
    \item The original problem is the following.
        $$x^2 + 5x < 0 \implies x<0$$.
    \item The contrapositive equivalently restates the problem as the following.
        $$\neg(x < 0) \implies \neg(x^2 + 5x < 0)$$.
        $$x \geq 0 \implies x^2 + 5x \geq 0$$.
    \item Suppose $x \geq 0$, then $x^2 \geq 0$ and $5x \geq 0$.
    \item Therefore $x^2 + 5x \geq 0$, proving the contrapositive statement.

\end{prooflist}

\begin{prooflist} {6. Suppose $x \in \mathbb{R}$. If $x^3 -x > 0$ then $x>-1$.}
    \item The original problem is the following. $$x^3 -x > 0 \implies x>-1$$.
    \item The contrapositive equivalently restates the problem as the following.
        $$x \leq -1 \implies x^3 -x \leq 0$$
    \item Suppose $x \leq -1 $.
    \item We can multiply both sides by $x^2 > 0$, and get $x^3 \leq -x^2.$
    \item Since $x^2$ is always non-negative, and $x<-1$, we get $x^2 > -x$.
    \item Muliply both sides by $-1$, and then $-x^2 < x$.
    \item From $x^3 \leq -x^2$ and $-x^2 < x$, we get $x^3 \leq x$.
    \item Subtract $x$ from both sides and $x^3 -x \leq 0$, and we have proven the
        contrapositive.
\end{prooflist}

\begin{prooflist} {7. Suppose $a, b \in \mathbb{Z}$. If both $ab$ and $a + b$
    are even, then both $a$ and $b$ are even. }
    \item Let $E(x)$ mean that $x$ is even, and $O(x)$ is odd.
    \item The original problem is the following.
        $$E(ab) \land E(a+b) \implies E(a) \land E(b) $$
    \item The contrapositive equivalently restates the problem as the following.
        $$\neg\big( (E(a) \land E(b)\big) \implies \neg \big(E(ab) \land E(a+b)\big) $$
        $$ O(a) \lor O(b) \implies O(ab) \lor O(a+b) $$
    \item Suppose $a$ is odd, and $b$ is even.
    \item Let $a=2n+1$ and $b=2m$, for some $m, n \in \mathbb{N}$.
    \item Then $a + b = 2n+1 + 2m = 2(n + m) +1$, which is odd.
    \item Therefore we have proven the contrapositive, and the original statement.
\end{prooflist}

\begin{prooflist} {8. Suppose $x \in \mathbb{R}$. If $x^5 - 4x^4 + 3x^3 - x^2 +
    3x -4 \geq 0$, then $x \geq 0$.}
    \item The original problem is the following.
    $$x^5 - 4x^4 + 3x^3 - x^2 + 3x -4 \geq 0 \implies x \geq 0$$
    \item The contrapositive of the statement is the following.
    $$ x < 0 \implies x^5 - 4x^4 + 3x^3 - x^2 + 3x -4 < 0 $$
    \item Since $ x<0$, then all odd powers of $x$ will be negative.
    \item Also, since all the coefficients for the odd powers are positive, all
        of terms with an odd power will be negative.
    \item Since $ x<0$, then all even powers of $x$ will be positive.
    \item Also, since all the coefficients for the even powers are negative, all
        of terms with an even power will be negative.
    \item The sum of a set of negative values is also negative, therefore we
        have proven the contrapositive.
\end{prooflist}

\begin{prooflist} {9. Suppose $n \in \mathbb{Z}$. If $3 \nmid n^2$, then $3 \nmid n$.}
    \item The original problem is the following.
        $$3 \nmid n^2 \implies 3 \nmid n$$
    \item The contrapositive is stated as the following.
        $$3 \mid n \implies 3 \mid n^2$$
    \item Suppose $3 \mid n$, then $3a=n$, for some $a \in \mathbb{Z}$.
    \item Then $(3a)^2=n^2$, and $3 \cdot (3a^2) = n^2$.
    \item Therefore $3 \mid n^2$, and we have proven the contrapositive.
\end{prooflist}

\begin{prooflist} {10. Suppose $x,y,z \in \mathbb{N}$, and $x\neq0$. If $x
    \nmid yz$, then $x\nmid y$ and $x \nmid z$.}
    \item The original problem is the following.
        $$ x \nmid yz \implies x\nmid y \land x \nmid z$$
    \item The contrapositive of the statement is the following.
        $$
        \neg(x\nmid y \land x \nmid z)
        \implies
        \neg(x \nmid yz)
        $$
        $$
        x\mid y \lor x \mid z
        \implies
        x \mid yz
        $$
    \item Suppose $x \mid y$, thus $xa = y$ for some $a \in \mathbb{N}$.
    \item Then multiply both sides by $z$ and $x(az) = yz$.
    \item Therefore $x \mid yz$, and we have proven the contrapositive.
\end{prooflist}

\begin{prooflist} {11. Suppose $x,y \in \mathbb{Z}$. If $x^2(y+3)$ is even, then
    $x$ is even or $y$ is odd.}
    \item Let $E(x)$ mean that $x$ is even, and $O(x)$ is odd.
    \item The original problem is the following.
        $$
            E\big(x^2(y+3)\big)
            \implies
            E(x) \lor O(y)
        $$
    \item It can be restated in the contrapositive as the following.
        $$
            \neg\big(E(x) \lor O(y)\big)
            \implies
            \neg\big(E\big(x^2(y+3)\big)\big)
        $$
        $$
            O(x) \land E(y)
            \implies
            O\big(x^2(y+3)\big)
        $$
    \item Suppose $O(x)$ and $E(y)$, then $x=2n+1$ and $y=2m$, for some $m,n \in
        \mathbb{N}$.
    \item Then via substitution
            $$x^2(y+3) = (2n+1)^2 (2m+3)$$
            $$ = (4n^2 + 4n +1) (2m+2+1)$$
    \item Therefore both quantities in the product are odd, and the product of two
        odd numbers is odd. Therefore we have proven the contrapositive, and the
        original statement.
\end{prooflist}

\begin{prooflist} {12. Suppose $a \in \mathbb{Z}$. If $a^2$ is not divisible by $4$, then $a$ is odd.}
    \item Let $E(x)$ mean that $x$ is even, and $O(x)$ is odd.
    \item The original problem is the following.
        $$
            4 \nmid a^2
            \implies
            O(a)
        $$
    \item The same problem can be restated via the contrapositive with the
        following.
        $$
            \neg(O(a))
            \implies
            \neg(4 \nmid a^2)
        $$
        $$
            E(a)
            \implies
            4 \mid a^2
        $$
    \item Suppose $E(a)$, therefore $a = 2n$ for some $a \in \mathbb{N}$.
    \item Therefore $a^2 = 4n^2$, and $4 | a^2$ and we have proven the
        contrapositive.
\end{prooflist}

\begin{prooflist} {13. Suppose $x \in \mathbb{R}$. If $x^5 + 7x^3 + 5x \geq x^4
    + x^2 +8$, then $x\geq 0$.}
    \item The original problem is the following.
        $$x^5 + 7x^3 + 5x \geq x^4 + x^2 +8
        \implies
        x\geq 0$$

    \item The same problem can be restated via the contrapositive with the
        following.
        $$
        \neg(x \geq 0)
        \implies
        \neg(x^5 + 7x^3 + 5x \geq x^4 + x^2 +8)
        $$
        $$
        x < 0
        \implies
        x^5 + 7x^3 + 5x < x^4 + x^2 +8
        $$
    \item Suppose $ x < 0$. Then $x$ to any odd power will be negative, and
        $x$ to any even power will be positive.
    \item Because all of the coefficients to the polynomial terms in
        $x^5 + 7x^3 + 5x < x^4 + x^2 +8$ are positive, the sign of each term is
        determined exclusively by the exponent.
    \item All the terms of $x^5 + 7x^3 + 5x$ are negative, and all the terms of
        $x^4 + x^2 +8$ are positive. The sum of a set of negative numbers
        is always less than the sum of a set of positive numbers. Therefore we
        have proven the contrapositive.
\end{prooflist}

\begin{prooflist} {14. If $a, b \in \mathbb{Z}$, and $a$ and $b$ have the same
    parity, then $3a+7$ and $7b-4$ do not have the same parity.}
    \item The original problem is the following.
        $$
        \big(E(a) \land E(b)\big)
        \lor
        \big(O(a) \land O(b)\big)
        \implies
        \big(E(3a+7) \land O(7b-4)\big)
        \lor
        \big(O(3a+7) \land E(7b-4)\big)
        $$
    \item Suppose $E(a) \land E(b)$, and $a=2n$ and $b=2m$, for some $a, b \in
        \mathbb{N}$.
    \item Then $3a+7 = 6n + 7 = 2(3n+3) + 1$, which is odd, and $7b-4 = 14m - 4
        = 2(7m -2)$, which is even.
    \item Now Suppose $O(a) \land O(b)$, and $a=2n+1$ and $b=2m+1$, for some $a, b \in
        \mathbb{N}$.
    \item Then $3a+7 = 3(2n+1) + 7 = 2(3n+5)$, which is even, and $7b-4 =
        7(2m+1) -4 = 14m - 3 = 2(7m -2) +1$, which is odd.
    \item Therefore we've proven the original statement, that if $a, b \in
        \mathbb{Z}$, and $a$ and $b$ have the same parity, then $3a+7$ and
        $7b-4$ do not have the same parity.
\end{prooflist}

\begin{prooflist} { 15. Suppose $x \in \mathbb{Z}$. If $x^3 - 1$ is even, then $x$ is odd.}
    \item Let $E(x)$ mean that $x$ is even, and $O(x)$ mean that $x$ is odd.
    \item The original problem is the following.
        $$
        E(x^3-1) \implies O(x)
        $$
    \item The contrapositive is the following.
        $$
        \neg(O(x))
        \implies
        \neg(E(x^3-1))
        $$
        $$
        E(x)
        \implies
        O(x^3-1)
        $$
    \item Suppose $E(x)$. Therefore $x=2n$, for some $n \in \mathbb{N}$.
    \item Then $x^3-1=(2n)^3 -1 = 8n^3 -1 = 2 (4n^3) -1$, and therefore
        $O(x^3-1)$.
    \item We have proven the contrapositive, and thus the original statement as
        well.
\end{prooflist}

\begin{prooflist} { 16. Suppose $x, y \in Z$. If $x+y$ is even, then $x$ and $y$ have the same parity.}
    \item Let $E(x)$ mean that $x$ is even, and $O(x)$ mean that $x$ is odd.
    \item The original problem is the following.
        $$
        E(x+y)
        \implies
        \big(E(x) \land E(y)\big) \lor \big(O(x) \land O(y)\big)
        $$
    \item The contrapositive is the following.
        $$
        \neg\Big(\big(E(x) \land E(y)\big) \lor \big(O(x) \land O(y)\big)\Big)
        \implies
        \neg(E(x+y))
        $$
        $$
        \neg(\big(E(x) \land E(y)\big)) \land \neg(\big(O(x) \land O(y)\big)
        \implies
        \neg(E(x+y))
        $$
        $$
        \big(O(x) \lor O(y)\big) \land \big(E(x) \lor E(y)\big)
        \implies
        O(x+y)
        $$
    \item Suppose
        $\big(O(x) \lor O(y)\big) \land \big(E(x) \lor E(y)\big)$. The only way
        for it to be true is $E(x) \land O(Y)$ or $O(x) \land E(Y)$.
    \item Therefore $x$ and $y$ must have different parity.
    \item Let $x=2n$ and $y=2m+1$, for some $m, n \in \mathbb{N}$.
    \item Then $x+y = 2(n+m) +1$, which is odd.
    \item It would have been the same result to let $x$ be odd and $y$ be even.
    \item Therefore $x+y$ is odd and we have proven the contrapositive of the
        original statement.
\end{prooflist}

\begin{prooflist} {17. If $n$ is odd, then $8 \mid (n^2-1)$.}
    \item Let $E(x)$ mean that $x$ is even, and $O(x)$ mean that $x$ is odd.
    \item The original problem is the following.
        $$
        O(n)
        \implies
        8 \mid (n^2-1)
        $$
    \item The contrapositive of the original forumation is as follows.
        $$
        \neg(8 \mid (n^2-1))
        \implies
        \neg(O(n))
        $$
        $$
        8 \nmid (n^2-1)
        \implies
        E(n)
        $$
    \item Suppose that $O(n)$, and $n=2a+1$ for some $a \in \mathbb{N}$.
    \item Therefore $n^2=(2a+1)^2= 4a^2+4a+1$.
    \item Therefore $n^2-1= 4a^2+4a = 4a(a+1)$.
    \item Either $a$ is even, or $a+1$ is even.
    \item If $a$ is even, then $a=2c$, for some $c \in \mathbb{N}$.
    \item Therefore $n^2-1 = 8c(a+1)$, and $8 \mid (n^2-1)$.
    \item If $a$ is odd, then $a+1=2c+2$, for some $c \in \mathbb{N}$.
    \item Therefore $n^2-1 = 8a(c+1)$, and $8 \mid (n^2-1)$.
\end{prooflist}

\begin{prooflist} {18. If $a,b \in \mathbb{Z}$, then $(a+b)^3 \equiv a^3 + b^3 \pmod 3$.}
    \item The original problem is the following.
        $$
            a,b \in \mathbb{Z}
            \implies
            (a+b)^3 \equiv a^3 + b^3 \pmod 3
        $$
    \item Suppose $a,b \in \mathbb{Z}$
    \item $a^3 + b^3 = (a + b)(a^2 - ab + b^2)$
    \item $3 \mid (a+b)^3 - (a^3 + b^3)$
    \item $3x = (a+b)^3 - a^3 - b^3$ , for some $x \in \mathbb{N}$
    \item $3x = (a+b)^3 - (a^3 - b^3)$

\end{prooflist}
\end{document}
