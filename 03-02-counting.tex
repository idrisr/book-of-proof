\documentclass{idrisMemo}

\usepackage{amsthm}
\usepackage{amsfonts}
\usepackage{hyperref}
\usepackage{enumitem}
\usepackage{amssymb}
\usepackage{tocloft} % Include the package

\memoto{Idris}
\memosubject{Book of Proof}
\memodate{2024.03.10}
\status{Section 3.2}

\newcounter{exercise} % This defines a new counter named 'exercise'

\newcommand{\exercise}[1]{\refstepcounter{exercise} \noindent Exercise: #1
    \addcontentsline{toc}{subsection}{#1}} % Use 'section' level for TOC

\newenvironment{prooflist}[1]
    {
    \pagebreak
    \exercise{#1}
    \begin{itemize}[label={}, leftmargin=1mm, itemsep=0.5mm]}
   {\end{itemize} {\hfill $\square$} }

\begin{document}

\tableofcontents
\pagebreak

\begin{prooflist} {1.  Consider lists made from the letters T, H, E, O, R, Y, with repetition allowed.}
    \item (a) How many length-4 lists are there?
    \item $ 6^4$
    \item (b) How many length-4 lists are there that begin with T ?
    \item $ 1 \cdot 6^3$
    \item (c) How many length-4 lists are there that do not begin with T ?
    \item $ 5 \cdot 6^3$
\end{prooflist}

\begin{prooflist} {2. Airports are identified with 3-letter codes. For example, Richmond, Virginia has the code RIC, and Memphis, Tennessee has MEM. How many different 3-letter codes are possible?}
    \item $ 26^3$
\end{prooflist}

\begin{prooflist} {3. How many lists of length 3 can be made from the symbols A, B, C, D, E, F if...}
    \item (a) repetition is allowed.
    \item $ 6^3$
    \item (b) repetition is not allowed.
    \item $ 6\cdot 5 \cdot 4$
    \item (c) repetition is not allowed and the list must contain the letter A.
    \item $ 6\cdot 5 \cdot 3$
    \item (d) repetition is allowed and the list must contain the letter A.
    \item Let $A=$ list of all length-three strings.
    \item Let $B=$ list of all length-three strings without A.
    \item Then $|A| - |B| = 6^3 - 5^3$
\end{prooflist}

\begin{prooflist}{ 4. In ordering coffee you have a choice of regular or decaf; small, medium or large; here or to go. How many different ways are there to order a coffee? }
\item Let $A = \{\text{regular}, \text{decaf}\}$.
\item Let $B = \{\text{small}, \text{medium}, \text{large}\}$.
\item Let $C = \{\text{here}, \text{to-go}\}$.
\item Different ways to order is the product $|A|\cdot|B|\cdot|C| = 2 \cdot 3
    \cdot 2$.
\end{prooflist}

\begin{prooflist}{
5. This problem involves 8-digit binary strings such as 10011011 or 00001010
(i.e., 8-digit numbers composed of 0’s and 1’s).}
    \item (a) How many such strings are there?
    \item There are $2^8$ different 8-digit binary strings.
    \item (b) How many such strings end in 0?
    \item There are $2^7$ different 8-digit binary strings that end in $0$.
    \item (c) How many such strings have 1’s for their second and fourth digits?
    \item There are $2^6$ different 8-digit binary strings that have two of
        their bits constant.
    \item (d) How many such strings have 1’s for their second or fourth digits?
    \item Here we must be careful not to double count.
    \item Let $A =$ 8-digit strings with 1 as the fourth digit.
    \item Let $B =$ 8-digit strings with 1 as the second digit.
    \item Let $C =$ 8-digit strings with 1 as the second digit and 1 as the
        fourth digit.
    \item The total number of 8-digit strings with 1 as the second or fourth
    digit is $|A| + |B| - |C| = 2^7+2^7-2^6$.
\end{prooflist}

\begin{prooflist}{ 6. You toss a coin, then roll a dice, and then draw a card from a 52-card deck.}
    \item Let $A = \{\text{H}, \text{T}\}$
    \item Let $B = \{1, 2, 3, 4, 5, 6\}$
    \item Let $C = \{1\ldots 52 \}$
    \item How many different outcomes are there?
    \item $|A| \cdot |B| \cdot |C| = 2 \cdot 6 \cdot 52$.
    \item How many outcomes are there in which the dice lands on $3$?
    \item $|A| \cdot |C| = 2 \cdot 52$.
    \item How many outcomes are there in which the dice lands on an odd number?
    \item $|A| \cdot |\text{odd}| \cdot |C| = 2 \cdot 3 \cdot 52$.
    \item How many outcomes are there in which the dice lands on an odd number
        and the card is a King?
    \item $|A| \cdot |\text{odd}| \cdot |\text{King}| = 2 \cdot 3 \cdot 4$.
\end{prooflist}

\begin{prooflist}{ 7. This problem concerns 4-letter codes made from the letters A, B, C, D, ... , Z. }
\item (a) How many such codes can be made?
\item $26^4$, assuming repetitions.
\item (b) How many such codes have no two consecutive letters the same?
\item We can pick any letter to be the first one. To assure a non-consecutive
    2nd letter, we can pick from 25 letters.  To pick a 3rd letter which is
    non-consecutive with the 2nd letter, there are 25 choices, as it is now ok
    to re-use the first letter. And so it goes, so that for every non-first
    letter, there are $n-1$ choices.
\item Therefore the answer is $26\cdot 25^3$.
\end{prooflist}

\begin{prooflist}{8. A coin is tossed 10 times in a row. How many possible sequences of heads and
tails are there?
}
\item $2^{10}$
\end{prooflist}

\begin{prooflist} {9. A new car comes in a choice of five colors, three engine sizes and two transmissions. How many different combinations are there?}
\item $5 \cdot 3 \cdot 2$.
\end{prooflist}

\begin{prooflist}{ 10. A dice is tossed four times in a row. There are many possible outcomes. How many different outcomes are possible?}
\item $ 6^4$
\end{prooflist}

\end{document}
