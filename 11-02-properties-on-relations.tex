\documentclass{idrisMemo}
\usepackage{bookOfProof}

\memoto{Idris}
\memosubject{Book of Proof}
\memodate{2024.03.21}
\status{\S 11.2 Properties on Relations}

\begin{document}
\toc
\thispagestyle{styleTOC}
\pagebreak
\pagestyle{styleE}

\begin{prooflist}{1. Consider the relation $R=\set{(a, a),(b, b),(c, c),(d, d),(a,
    b),(b, a)}$ on set $A=\{a, b, c, d\}$. Is $R$ reflexive? Symmetric?
Transitive? If a property does not hold, say why.}
\item $R$ is reflective because $(x, x) \in R, \forall x \in A$.
\item $R$ is symmetric because $(x, y) \in R \implies (y, x) \in R, \forall x,y \in A$.
\item $R$ is transitive because $\forall (x, y), (y,z) \in R
    \implies (x, z) \in R, \forall x,y,z \in A$.
\end{prooflist}

\begin{prooflist}{2. Consider the relation $R=\set{(a, b),(a, c),(c, c),(b, b),(c,
    b),(b, c)}$ on the set $A=\set{a, b, c}$. Is $R$ reflexive? Symmetric?
Transitive? If a property does not hold, say why.}
\item $R$ is not reflective because $(a, a) \not\in R$.
\item $R$ is not symmetric because $(a, b) \in R \not\implies (b, a) \in R$.
\item $R$ is not transitive because $(b, c) \land (c, c) \in R \not \implies (b,
    c) \in R$.
\end{prooflist}

\begin{prooflist}{3. Consider the relation $R=\set{(a, b),(a, c),(c, b),(b, c)}$
    on the set $A=\set{a, b, c}$. Is $R$ reflexive? Symmetric? Transitive? If a
property does not hold, say why.}
\item $R$ is not reflective because $(a, a) \not\in R$.
\item $R$ is not symmetric because $(a, b) \in R \not\implies (b, a) \in R$.
\item $R$ is not transitive because $(c, b) \land (b, c) \in R \not \implies (c, c) \in R$.
\end{prooflist}

\begin{prooflist}{4. Let $A=\set{a, b, c, d}$. Suppose $R$ is the relation}
\item
\begin{align*}
R= & (a, a),(b, b),(c, c),(d, d),(a, b),(b, a),(a, c),(c, a), \\
& (a, d),(d, a),(b, c),(c, b),(b, d),(d, b),(c, d),(d, c)
\end{align*}
\item $R$ is reflective because $(x, x) \in R, \forall x \in A$.
\item $R$ is symmetric because $(x, y) \in R \implies (y, x) \in R, \forall x, y\in A$.
\item $R$ is transitive because $|R|=16$, implying that all possible
    pairs of a $A$ are in $R$, including all the ones to satisfy the transitive
    property.
\end{prooflist}

\begin{prooflist}{5. Consider the relation $R=\{(0,0),(\sqrt{2}, 0),(0, \sqrt{2}),(\sqrt{2}, \sqrt{2})\}$ on $\mathbb{R}$. Is $R$ reflexive? Symmetric? Transitive? If a property does not hold, say why.}
\item Let $A=\set{\sqrt{2}, 0}$.
\item $R$ is reflective because $(x, x) \in R, \forall x \in A$.
\item $R$ is symmetric because $(x, y) \in R \implies (y, x) \in R, \forall x, y\in A$.
\item $R$ is transitive because $|R|=4, |A|=2, R=A\times A$, implying that all possible
    pairs of a $A$ are in $R$, including all the ones to satisfy the transitive
    property.
\end{prooflist}

\begin{prooflist}{6. Consider the relation $R=\{(x, x): x \in \mathbb{Z}\}$ on
    $\mathbb{Z}$. Is this $R$ reflexive? Symmetric? Transitive? If a property
does not hold, say why. What familiar relation is this?}
\item $R$ defines an equality relationship.  The equality relationship is
reflexive, symmetric, and transitive, therefore R is all of those things.
\end{prooflist}

\begin{prooflist}{7. There are 16 possible different relations $R$ on the set $A=\{a, b\}$. Describe all of them. (A picture for each one will suffice, but don't forget to label the nodes.) Which ones are reflexive? Symmetric? Transitive?}
\item Because $|A|=2$, there are 4 different pairs in $AxA$. The number of
    relatinoships is the powerset of $AxA$, and its cardinality is $2^4=16$.
\item I'm not going to draw all of them.
\end{prooflist}

\begin{prooflist}{8. Define a relation $R$ on $\mathbb{Z}$ as $x R y$ if $|x-y|<1$. Is $R$ reflexive? Symmetric? Transitive? If a property does not hold, say why. What familiar relation is this?}
\item The only way for two integers to be less
    than one apart, is for the the integers to be the same. Therefore $R$
    defines and equality relationship which is refesive, symmetric, and transitive.
\end{prooflist}

\begin{prooflist}{9. Define a relation on $\mathbb{Z}$ by declaring $x R y$ if and only if $x$ and $y$ have the same parity. Is $R$ reflexive? Symmetric? Transitive? If a property does not hold, say why. What familiar relation is this?}
\item This relation is equivalent to a partition of $\mathbb{Z}$ into the evens
    and the odds.  Each partition also defines an equivalence class, and an
    equivalence class is by definition reflexive, symmetric, and transitive.
\end{prooflist}

\begin{prooflist}{10. Suppose $A \neq \varnothing$. Since $\varnothing \subseteq A \times A$, the set $R=\varnothing$ is a relation on $A$. Is $R$ reflexive? Symmetric? Transitive? If a property does not hold, say why.}
\item The relation $R$ has cardinality 0. Therefore all of the implications
    necesarray to prove a relation is reflexive, symmetric, and transitive are
    all trivially true. Therefore $R$ is reflexive, symmetric, and transitive.
\end{prooflist}

\begin{prooflist}{11. Let $A=\{a, b, c, d\}$ and $R=\{(a, a),(b, b),(c, c),(d, d)\}$. Is $R$ reflexive? Symmetric? Transitive? If a property does not hold, say why.}
\item $R$ defines an equality relationship which is reflexive, symmetric, and transitive.
\end{prooflist}

\begin{prooflist}{12. Prove that the relation $\mid$ (divides) on the set $\mathbb{Z}$ is reflexive and transitive. (Use Example 11.8 as a guide if you are unsure of how to proceed.)}
\item Let $R$ be the relation $\mid$ (divides).
\item $x \mid x \implies x \mid x$, therefore R is reflexive.
\item Let us prove that $x \mid y \land y \mid z \implies x \mid z: x,y,z \in\mathbb{Z}$.
\item Suppose $x\mid y$ and $y\mid z$. Therefore there exists some integers a
    and b such that $xa=y$ and $yb=z$. Thus $x(ab)=z$, which shows that $x\mid
    z$. Thusly we have shown that $R$ is transitive.
\end{prooflist}

\begin{prooflist}{13. Consider the relation $R=\{(x, y) \in \mathbb{R} \times \mathbb{R}: x-y \in \mathbb{Z}\}$ on $\mathbb{R}$. Prove that this relation is reflexive, symmetric and transitive.}
\item Suppose $x - x \in \mathbb{Z}$.  By definition $x-x=0 \in \mathbb{Z}$,
    therefore $R$ is reflexive.
\item Suppose $x - y \in \mathbb{Z}$. Therefore $x-y + y-x = 0$. Thus
    $y-x\in\mathbb{Z}$. Thus $R$ is symmetric.
\item Suppose $x - y \in \mathbb{Z} \land y - z\in\mathbb{Z}$. Therefore $x-z =
    (x-y) + (y-z)$. Since both $x-y$ and $y-z$ are integers, the difference of those
    two quantities is also an integer, and therefore $x-z\in\mathbb{Z}$, and
    thus $R$ is transitive.
\end{prooflist}

\begin{prooflist}{14. Suppose $R$ is a symmetric and transitive relation on a
    set $A$, and there is an element $a \in A$ for which $Rax$ for every $x
\in A$. Prove that $R$ is reflexive.}
\item Because $R$ is symmetric, then $Rxy \implies Ryx$.
\item Because $R$ is transitive, then $R x y \land Ryz \implies Rxz$.
\item Suppose some relation $Rxa\implies Rax$ because of symmetry.
\item $Rxa \land Rax \implies Rxx$ because of transitivity. Therefore we have
    shown $R$ is also reflexive.
\end{prooflist}

\begin{prooflist}{15. Prove or disprove: If a relation is symmetric and transitive, then it is also reflexive.}
\item $Rxy \implies Ryx$, due to symmetry.
\item $Rxy \land Ryx \implies Rxx$, due to transitivity.
\item Therefore a relation being symmetric and transitive implies the relation
    is also reflexive.
\end{prooflist}

\begin{prooflist}{16. Define a relation $R$ on $\mathbb{Z}$ by declaring that $x R y$ if and only if $x^2 \equiv y^2(\bmod 4)$. Prove that $R$ is reflexive, symmetric and transitive.}
\item Let $(x, y) \in R\subseteq \mathbb{Z}\times\mathbb{Z}; x^2\equiv y^2\bmod 4$.
\item Suppose $(x, y)$ in $R$. Therefore $4 | x^2 - y^2$. If $x=y$, then $4\mid
    0$, which is always true, therefore $R$ is reflexive.
\item Suppose $(x, y)$ in $R$. Therefore $4 | x^2 - y^2$. Hence, there exists
    some integer $a$ such that $4a = x^2-y^2$. Multiplying both sides by $-1$
    and we get $-4a = y^2-x^2$.  If a number is divisible by $n$, then it is also
    divisble by $-n$.
\item Suppose $(x, y)\land (y, z)$ in $R$. Therefore $4 \mid x^2 - y^2 \land
    4\mid y^2-z^2$. Hence, there exists integers $a$ and $b$ such that $4a =
    x^2-y^2$ and $4b=y^2-z^2$. Adding the two equations together yields
    $4(a+b) = x^2 -z^2$, which shows $4\mid x^2 - z^2$, and thus $R$ is
    transitive.
\end{prooflist}

\begin{prooflist}{17. Modifying Exercise 8 (above) slightly, define a relation
    $\sim$ on $\mathbb{Z}$ as $x \sim y$ if and only if $|x-y| \leq 1$. Say
    whether $\sim$ is reflexive. Is it symmetric? Transitive?}
\item For any integer $x$, $|x-x| \leq 1$, therefore $R$ is reflexive.
\item This relation holds for any integers which are either equal are
    consecutive. Assume some integer $x$ and its consecutive integer $x+1$.
    Therefore $|x-x-1|=1\leq 1 \implies |x+1-x|=1 \leq 1$.
    For the case of $Rxx$, $|x-x|=0 < 1 \implies 0 < 1$, therefore $R$ is symmetric.
\item This relation is not transitive. For example $2R3 \land 3R4 \not\implies
    2R4$.
\end{prooflist}

\begin{prooflist}{18. The table on page 205 shows that relations on Z may obey
        various combinations of the reflexive, symmetric and transitive
        properties. In all, there are 23 = 8 possible combinations, and the
        table shows 5 of them. (There is some redundancy, as $\neq$ and $\mid$
        have the same type.) Complete the table by finding examples of relations
    on Z for the three missing combinations.}
\item
\begin{tabular}{|c|c|c|c|}
    \hline
    \textbf{Reflexive} & \textbf{Symmetric} & \textbf{Transitive} & \textbf{Example} \\
    \hline $\checkmark$ & $\checkmark$ & $\checkmark$ & $=$ \\
    \hline $\checkmark$ & $\checkmark$ &  & $x-y\leq 1; x,y\in\mathbb{Z}$ \\
    \hline $\checkmark$ &  & $\checkmark$ & $\leq$ \\
    \hline & $\checkmark$ & $\checkmark$ & none \\
    \hline $\checkmark$ &  &  &  none \\
    \hline & $\checkmark$ &  & $\neq$ \\
    \hline &  & $\checkmark$ & $\set{(1,2), (2, 3), (1, 3)}$ \\
    \hline &  &  &  $\set{1,3}$\\
    \hline
\end{tabular}
\end{prooflist}


\end{document}
