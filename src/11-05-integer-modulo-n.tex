\documentclass{article}
\makeindex

\usepackage{amsmath}
\usepackage{amssymb}
\usepackage{booktabs}
\usepackage{csvsimple-l3}
\usepackage{bussproofs}
\usepackage{dirtytalk}
\usepackage[dvipsnames]{xcolor}
\usepackage{enumitem}
\usepackage{epigraph}
\usepackage{forest}
\usepackage{formal-grammar}
\usepackage{graphicx}
\usepackage[citecolor=blue,colorlinks=true, linkcolor=blue, urlcolor=blue]{hyperref}
\usepackage{kantlipsum}
\usepackage{makeidx}
\usepackage[margin=0.8in]{geometry}
\usepackage{mathrsfs}
\usepackage[outputdir=../build]{minted}
\usepackage{multicol}
\usepackage[mode=tex]{standalone}
\usepackage[style=authortitle]{biblatex}
\usepackage[T1]{fontenc}
\usepackage[tableaux]{prooftrees}
\usepackage{tcolorbox}
\usepackage{tikz}
\usepackage{titlesec}
\usepackage{xcolor}

\usetikzlibrary{arrows}
\usetikzlibrary{arrows.meta}
\usetikzlibrary{automata}
\usetikzlibrary{calc}
\usetikzlibrary{fit}
\usetikzlibrary{petri}
\usetikzlibrary{positioning}

\tcbuselibrary{breakable}
\tcbuselibrary{listings}
\tcbuselibrary{minted}
\tcbuselibrary{skins}
\tcbuselibrary{theorems}

\newcounter{filePrg}

\addbibresource{biblio.bib}
\setlength{\parindent}{0pt}

\renewcommand{\emph}[1]{\textit{#1}}
\setlength{\parindent}{0pt}

\newcommand\setboxcounter[2]{\setcounter{tcb@cnt@#1}{#2}}
\newcommand\qed[0]{\blacksquare}
\setlength{\parindent}{10pt}
\newcommand{\set}[1]{\{#1\}}
\newcommand{\up}[1]{ \left\lceil#1\right\rceil }
\newcommand{\down}[1]{ \left\lfloor#1\right\rfloor}

\definecolor{CaribbeanBlue}{RGB}{0, 206, 209} % Define Caribbean Blue
\NewTcbTheorem[list inside=definition]{definition}
{Definition}{
	breakable,
	colback=CaribbeanBlue!05,
	colframe=CaribbeanBlue!35!black,
	fonttitle=\bfseries}{th}

\NewTcbTheorem[list inside=intuition]{intuition}{Intuition}{
	breakable,
	colback=blue!5,
	colframe=blue!35!black,
	fonttitle=\bfseries}{th}

\NewTcbTheorem{example}{Example}{
	breakable,
	colback=white,
	colframe=green!35!black,
	fonttitle=\bfseries}{th}

\NewTcbTheorem{verify}{Verify}{
	breakable,
	float,
	colback=red!5,
	colframe=red!35!black,
	fonttitle=\bfseries}{th}

\NewTcbTheorem[list inside=theorem]{theorem}{Theorem}{
	breakable,
	colback=gray!10,
	colframe=gray!35!black,
	fonttitle=\bfseries}{th}

% \NewTcbTheorem[
% list inside=exercise,
% number within=chapter,
% number within=section,
% ]

\NewTcbTheorem[
	list inside=exercise,
	number within=section,
]{exercise}{Exercise}{
	breakable,
	colback=white,
	colframe=black,
	fonttitle=\bfseries}{th}

\newcommand{\hask}[1]{\mintinline{haskell}{#1}}

\newenvironment{alist}
{\begin{enumerate}[label={*}, leftmargin=*, itemsep=0pt, parsep=0pt]}
		{\end{enumerate}}

\newenvironment{blist}
{\begin{enumerate}[label={}, leftmargin=*, itemsep=0pt, parsep=0pt]}
		{\end{enumerate}}

\renewcommand{\thesection}{\arabic{section}}
\tcbset{enhanced jigsaw}

\newtcbinputlisting{\codeFromFile}[2]{
	listing file={#1},
	listing engine=minted,
	minted style=colorful,
	minted language=haskell,
	minted options={breaklines,linenos,numbersep=3mm},
	colback=blue!5!white,colframe=blue!75!black,listing only,
	left=5mm,enhanced,
	title={#2},
	overlay={\begin{tcbclipinterior}\fill[red!20!blue!20!white] (frame.south west)
				rectangle ([xshift=5mm]frame.north west);\end{tcbclipinterior}}
}

\newtcblisting{haskell}[1]
{
	listing engine=minted,
	minted style=colorful,
	minted language=haskell,
	minted options={breaklines,linenos,numbersep=3mm},
	colback=blue!5!white,colframe=blue!75!black,listing only,
	left=5mm,enhanced,
	title={#1},
	overlay={\begin{tcbclipinterior}\fill[red!20!blue!20!white] (frame.south west)
				rectangle ([xshift=5mm]frame.north west);\end{tcbclipinterior}}
}


\begin{document}
\section{The Integers Modulo $n$}

\begin{exercise}{}{}
	{1. Write the addition and multiplication tables for $\mathbb{Z}_2$.}
	\begin{alist}
		\item
		\[
			\begin{array}{c|cc}
				\times & 0 & 1 \\
				\hline
				0      & 0 & 0 \\
				1      & 0 & 1 \\
			\end{array}
		\]
		\item
		\[
			\begin{array}{c|cc}
				+ & 0 & 1 \\
				\hline
				0 & 0 & 1 \\
				1 & 1 & 0 \\
			\end{array}
		\]
	\end{alist}
\end{exercise}

\begin{exercise}{}{}
	{2. Write the addition and multiplication tables for $\mathbb{Z}_3$.}
	\begin{alist}
		\item \[
			\begin{array}{c|ccc}
				\times & 0 & 1 & 2 \\
				\hline
				0      & 0 & 0 & 0 \\
				1      & 0 & 1 & 2 \\
				2      & 0 & 2 & 1 \\
			\end{array}
		\]
		\item \[
			\begin{array}{c|ccc}
				+ & 0 & 1 & 2 \\
				\hline
				0 & 0 & 1 & 2 \\
				1 & 1 & 2 & 0 \\
				2 & 2 & 0 & 1 \\
			\end{array}
		\]
	\end{alist}
\end{exercise}{}{}

\begin{exercise}{}{}
	{3. Write the addition and multiplication tables for $\mathbb{Z}_4$.}
	\begin{alist}
		\item \[
			\begin{array}{c|cccc}
				\times & 0 & 1 & 2 & 3 \\
				\hline
				0      & 0 & 0 & 0 & 0 \\
				1      & 0 & 1 & 2 & 3 \\
				2      & 0 & 2 & 0 & 2 \\
				3      & 0 & 3 & 2 & 1 \\
			\end{array} \]
		\item \[
			\begin{array}{c|cccc}
				+ & 0 & 1 & 2 & 3 \\
				\hline
				0 & 0 & 1 & 2 & 3 \\
				1 & 1 & 2 & 3 & 0 \\
				2 & 2 & 3 & 0 & 1 \\
				3 & 3 & 0 & 1 & 2 \\
			\end{array} \]
	\end{alist}
\end{exercise}{}{}

\begin{exercise}{}{}
	{4. Write the addition and multiplication tables for $\mathbb{Z}_6$.}
	\begin{alist}
		\item \[
			\begin{array}{c|cccccc}
				\times & 0 & 1 & 2 & 3 & 4 & 5 \\
				\hline
				0      & 0 & 0 & 0 & 0 & 0 & 0 \\
				1      & 0 & 1 & 2 & 3 & 4 & 5 \\
				2      & 0 & 2 & 4 & 0 & 2 & 4 \\
				3      & 0 & 3 & 0 & 3 & 0 & 3 \\
				4      & 0 & 4 & 2 & 0 & 4 & 2 \\
				5      & 0 & 5 & 4 & 3 & 2 & 1 \\
			\end{array}
		\]
		\item \[
			\begin{array}{c|cccccc}
				+ & 0 & 1 & 2 & 3 & 4 & 5 \\
				\hline
				0 & 0 & 1 & 2 & 3 & 4 & 5 \\
				1 & 1 & 2 & 3 & 4 & 5 & 0 \\
				2 & 2 & 3 & 4 & 5 & 0 & 1 \\
				3 & 3 & 4 & 5 & 0 & 1 & 2 \\
				4 & 4 & 5 & 0 & 1 & 2 & 3 \\
				5 & 5 & 0 & 1 & 2 & 3 & 4 \\
			\end{array}
		\]
	\end{alist}
\end{exercise}{}{}

\begin{exercise}{}{}
	{5. Suppose $[a],[b] \in \mathbb{Z}_5$ and $[a] \cdot[b]=[0]$. Is it
		necessarily true that either $[a]=[0]$ or $[b]=[0]$ ?}
	\begin{alist}
		\item Assume: $[a]\cdot[b]=[0]$ for some $[a],[b]\in\mathbb{Z}_5$.
		\item Translate: This means $[a\cdot b]\equiv[0]$, so $a\cdot b\equiv 0\bmod 5$
		\item Property of Modulo 5: This implies that 5 divides the product $a\cdot b$.
		\item Prime Divisibility: Since 5 is a prime number, it must divide either a or b.
		\item Back to Equivalence Classes: Therefore, $a \equiv 0 \pmod 5$ (meaning
		$[a]=[0]$) or $b \equiv 0 \pmod 5$ (meaning $[b]=[0]$).
		\item \[
			\begin{array}{c|ccccc}
				\times & 0 & 1 & 2 & 3 & 4 \\
				\hline
				0      & 0 & 0 & 0 & 0 & 0 \\
				1      & 0 & 1 & 2 & 3 & 4 \\
				2      & 0 & 2 & 4 & 1 & 3 \\
				3      & 0 & 3 & 1 & 4 & 2 \\
				4      & 0 & 4 & 3 & 2 & 1 \\
			\end{array}
		\]
	\end{alist}
\end{exercise}{}{}

\begin{exercise}{}{}
	{6. Suppose $[a],[b] \in \mathbb{Z}_6$ and $[a] \cdot[b]=[0]$. Is it
		necessarily true that either $[a]=[0]$ or $[b]=[0]$ ? What if $[a],[b] \in
			\mathbb{Z}_7$ ?}
	\begin{alist}
		\item Assume: $[a]\cdot[b]=[0]$ for some $[a],[b]\in\mathbb{Z}_6$.
		\item Translate: This means $[a\cdot b]\equiv[0]$, so $a\cdot b\equiv 0\bmod 6$
		\item Property of Modulo 6: This implies that 6 divides the product $a\cdot b$.
		\item Prime Divisibility: Since the prime factors of 6 are 2 and 3, 2 must
		factor a or b and 3 must factor a or b. Assume $3\mid a \land 2\mid b$. Let
		$a=3$ and $b=2$, and then $a\cdot b \equiv 0 \mod 6$.
		\item Back to Equivalence Classes: Therefore, it's not necessarily true that
		either $[a] = [0] \lor [b] = [0]$ when $[a]\cdot[b]=[0]$ for $\mathbb{Z}_6$.
		\item Assume: $[a]\cdot[b]=[0]$ for some $[a],[b]\in\mathbb{Z}_7$.
		\item Translate: This means $[a\cdot b]\equiv[0]$, so $a\cdot b\equiv 0\bmod 6$
		\item Property of Modulo 7: This implies that 7 divides the product $a\cdot b$.
		\item Prime Divisibility: Since 7 is a prime number, it must divide either a or b.
		\item Back to Equivalence Classes: Therefore, $a \equiv 0 \pmod 7$ (meaning
		$[a]=[0]$) or $b \equiv 0 \pmod 7$ (meaning $[b]=[0]$).
	\end{alist}
\end{exercise}{}{}

\begin{exercise}{}{}
	{7. Do the following calculations in $\mathbb{Z}_9$, in each case expressing
		your answer as $[a]$ with $0 \leq a \leq 8$.}
	\begin{alist}
		\item (a) $[8]+[8] = [7]$
		\item (b) $[24]+[11] = [8]$
		\item (c) $[21] \cdot[15] = [0]$
		\item (d) $[8] \cdot [8] = [1]$
	\end{alist}
\end{exercise}{}{}

\begin{exercise}{}{}
	{8. Suppose $[a],[b] \in \mathbb{Z}_n$, and $[a]=\left[a^{\prime}\right]$
		and $[b]=\left[b^{\prime}\right]$. Alice adds $[a]$ and $[b]$ as $[a]+[b]=$
		$[a+b]$. Bob adds them as
		$\left[a^{\prime}\right]+\left[b^{\prime}\right]=\left[a^{\prime}+b^{\prime}\right]$.
		Show that their answers $[a+b]$ and $\left[a^{\prime}+b^{\prime}\right]$ are the
		same.}
	\begin{alist}
		\item
	\end{alist}
\end{exercise}{}{}

\end{document}
