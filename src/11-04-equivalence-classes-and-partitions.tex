\documentclass{article}
\makeindex

\usepackage{amsmath}
\usepackage{amssymb}
\usepackage{booktabs}
\usepackage{csvsimple-l3}
\usepackage{bussproofs}
\usepackage{dirtytalk}
\usepackage[dvipsnames]{xcolor}
\usepackage{enumitem}
\usepackage{epigraph}
\usepackage{forest}
\usepackage{formal-grammar}
\usepackage{graphicx}
\usepackage[citecolor=blue,colorlinks=true, linkcolor=blue, urlcolor=blue]{hyperref}
\usepackage{kantlipsum}
\usepackage{makeidx}
\usepackage[margin=0.8in]{geometry}
\usepackage{mathrsfs}
\usepackage[outputdir=../build]{minted}
\usepackage{multicol}
\usepackage[mode=tex]{standalone}
\usepackage[style=authortitle]{biblatex}
\usepackage[T1]{fontenc}
\usepackage[tableaux]{prooftrees}
\usepackage{tcolorbox}
\usepackage{tikz}
\usepackage{titlesec}
\usepackage{xcolor}

\usetikzlibrary{arrows}
\usetikzlibrary{arrows.meta}
\usetikzlibrary{automata}
\usetikzlibrary{calc}
\usetikzlibrary{fit}
\usetikzlibrary{petri}
\usetikzlibrary{positioning}

\tcbuselibrary{breakable}
\tcbuselibrary{listings}
\tcbuselibrary{minted}
\tcbuselibrary{skins}
\tcbuselibrary{theorems}

\newcounter{filePrg}

\addbibresource{biblio.bib}
\setlength{\parindent}{0pt}

\renewcommand{\emph}[1]{\textit{#1}}
\setlength{\parindent}{0pt}

\newcommand\setboxcounter[2]{\setcounter{tcb@cnt@#1}{#2}}
\newcommand\qed[0]{\blacksquare}
\setlength{\parindent}{10pt}
\newcommand{\set}[1]{\{#1\}}
\newcommand{\up}[1]{ \left\lceil#1\right\rceil }
\newcommand{\down}[1]{ \left\lfloor#1\right\rfloor}

\definecolor{CaribbeanBlue}{RGB}{0, 206, 209} % Define Caribbean Blue
\NewTcbTheorem[list inside=definition]{definition}
{Definition}{
	breakable,
	colback=CaribbeanBlue!05,
	colframe=CaribbeanBlue!35!black,
	fonttitle=\bfseries}{th}

\NewTcbTheorem[list inside=intuition]{intuition}{Intuition}{
	breakable,
	colback=blue!5,
	colframe=blue!35!black,
	fonttitle=\bfseries}{th}

\NewTcbTheorem{example}{Example}{
	breakable,
	colback=white,
	colframe=green!35!black,
	fonttitle=\bfseries}{th}

\NewTcbTheorem{verify}{Verify}{
	breakable,
	float,
	colback=red!5,
	colframe=red!35!black,
	fonttitle=\bfseries}{th}

\NewTcbTheorem[list inside=theorem]{theorem}{Theorem}{
	breakable,
	colback=gray!10,
	colframe=gray!35!black,
	fonttitle=\bfseries}{th}

% \NewTcbTheorem[
% list inside=exercise,
% number within=chapter,
% number within=section,
% ]

\NewTcbTheorem[
	list inside=exercise,
	number within=section,
]{exercise}{Exercise}{
	breakable,
	colback=white,
	colframe=black,
	fonttitle=\bfseries}{th}

\newcommand{\hask}[1]{\mintinline{haskell}{#1}}

\newenvironment{alist}
{\begin{enumerate}[label={*}, leftmargin=*, itemsep=0pt, parsep=0pt]}
		{\end{enumerate}}

\newenvironment{blist}
{\begin{enumerate}[label={}, leftmargin=*, itemsep=0pt, parsep=0pt]}
		{\end{enumerate}}

\renewcommand{\thesection}{\arabic{section}}
\tcbset{enhanced jigsaw}

\newtcbinputlisting{\codeFromFile}[2]{
	listing file={#1},
	listing engine=minted,
	minted style=colorful,
	minted language=haskell,
	minted options={breaklines,linenos,numbersep=3mm},
	colback=blue!5!white,colframe=blue!75!black,listing only,
	left=5mm,enhanced,
	title={#2},
	overlay={\begin{tcbclipinterior}\fill[red!20!blue!20!white] (frame.south west)
				rectangle ([xshift=5mm]frame.north west);\end{tcbclipinterior}}
}

\newtcblisting{haskell}[1]
{
	listing engine=minted,
	minted style=colorful,
	minted language=haskell,
	minted options={breaklines,linenos,numbersep=3mm},
	colback=blue!5!white,colframe=blue!75!black,listing only,
	left=5mm,enhanced,
	title={#1},
	overlay={\begin{tcbclipinterior}\fill[red!20!blue!20!white] (frame.south west)
				rectangle ([xshift=5mm]frame.north west);\end{tcbclipinterior}}
}


\begin{document}
\section{Equivalence Classes and Partitions}
\begin{exercise}{}{}
	{1. List all the partitions of the set $A=\{a, b\}$. Compare your answer to
		the answer to Exercise 5 of Section 11.3.}
	\begin{alist}
		\item
		\begin{align*}
			\set{\set{a}, \set{b}} \\
			\set{\set{a, b}}
		\end{align*}
	\end{alist}
\end{exercise}{}{}

\begin{exercise}{}{}{2. List all the partitions of the set $A=\{a, b, c\}$. Compare
		your answer to the answer to Exercise 6 of Section 11.3.}
	\begin{alist}
		\item
		\begin{align*}
			\set{\set{a}, \set{b}, \set{c}} \\
			\set{\set{a, b}, \set{c}}       \\
			\set{\set{a, c}, \set{b}}       \\
			\set{\set{b, c}, \set{a}}       \\
			\set{\set{a, b, c}}
		\end{align*}
		\item
	\end{alist}
\end{exercise}{}{}

\begin{exercise}{}{}
	{3. Describe the partition of $\mathbb{Z}$ resulting from the equivalence
		relation $\equiv(\bmod 4)$.}
	\begin{alist}
		\item Let $R$ be the relation $\equiv\bmod 4$.
		\item For any integer $a$, the value of $a\bmod 4$ is in the set $\set{0, 1, 2, 3}$.
		\item Therefore each element in $\set{0, 1, 2, 3}$ defines an equivalence class
		on $\mathbb{Z}$ from the equivalence relation $\equiv(\bmod 4)$.
		\item Thus the 4 equivalence classes are
		\begin{align*}
			\set{-15, -11, -7, -3, 1 \dots x\bmod4=3} \\
			\set{-14, -10, -6, -2, 2 \dots x\bmod4=2} \\
			\set{-13, -9, -4, -1, 3 \dots x\bmod4=1}  \\
			\set{-12, -8, -4, 0, 4 \dots x\bmod4=0}   \\
		\end{align*}
	\end{alist}
\end{exercise}{}{}

\begin{exercise}{}{}{4. Suppose $P$ is a partition of a set $A$. Define a
		relation $R$ on $A$ by declaring $x R y$ if and only if $x, y \in X$ for
		some $X \in P$. Prove $R$ is an equivalence relation on $A$. Then prove that $P$
		is the set of equivalence classes of $R$.}
	\begin{alist}
		\item Let $P$ be a partition of set $A$.
		\item Define relation $R$ on $A$ as follows: $x R y$ if and only if $x$
		and $y$ belong to the same set $X \in P$.
		\item To prove $R$ is an equivalence relation:
		\begin{itemize}
			\item Reflexivity: For any $x \in A$, since $x$ belongs to some set
			      $X \in P$, $x R x$ by definition. Hence, $R$ is reflexive.
			\item Symmetry: If $x R y$, then $x$ and $y$ belong to the same set
			      $X \in P$. But this also means $y$ and $x$ belong to the same
			      set $X$, so $y R x$. Thus, $R$ is symmetric.
			\item Transitivity: If $x R y$ and $y R z$, then both $x$ and $y$,
			      and $y$ and $z$, belong to the same sets $X$ and $Y$ in $P$
			      respectively. Since $P$ is a partition, $X$ and $Y$ are either
			      identical or disjoint. If they are identical, then $x R z$
			      trivially. If they are disjoint, then $x$ and $z$ cannot belong
			      to the same set in $P$, so $x R z$ holds vacuously. In either
			      case, $R$ is transitive.
		\end{itemize}
		\item Now, to prove that $P$ is the set of equivalence classes of $R$:
		\begin{itemize}
			\item Each set $X \in P$ is an equivalence class of $R$: By
			      definition, elements in the same set $X$ under $P$ are related
			      by $R$. So, $P$ consists of equivalence classes of $R$.
			\item Every equivalence class of $R$ is a set in $P$: Since $P$ is a
			      partition, each element of $A$ belongs to exactly one set $X \in
				      P$. Thus, each equivalence class of $R$ corresponds to a set in
			      $P$.
		\end{itemize}
	\end{alist}
\end{exercise}{}{}

\begin{exercise}{}{}
	{5. Consider the partition $P=\{\{\ldots,-4,-2,0,2,4,
			\ldots\},\{\ldots,-5,-3,-1,1,3,5, \ldots\}\}$ of $\mathbb{Z}$. Let $R$ be
		the equivalence relation whose equivalence classes are the two elements of $P$.
		What familiar equivalence relation is $R$ ?}
	\begin{alist}
		\item $xRy$ is true when $x$ and $y$ have the same parity.
	\end{alist}
\end{exercise}{}{}

\begin{exercise}{}{}
	{6. Consider the partition
		$P=\{\{0\},\{-1,1\},\{-2,2\},\{-3,3\},\{-4,4\}, \ldots\}$ of $\mathbb{Z}$.
		Describe the equivalence relation whose equivalence classes are the elements of
		$P$.}
	\begin{alist}
		\item $xRy$ is defined by $|x|=|y|$.
	\end{alist}
\end{exercise}{}{}

\end{document}
