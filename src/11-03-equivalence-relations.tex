\documentclass{article}
\makeindex

\usepackage{amsmath}
\usepackage{amssymb}
\usepackage{booktabs}
\usepackage{csvsimple-l3}
\usepackage{bussproofs}
\usepackage{dirtytalk}
\usepackage[dvipsnames]{xcolor}
\usepackage{enumitem}
\usepackage{epigraph}
\usepackage{forest}
\usepackage{formal-grammar}
\usepackage{graphicx}
\usepackage[citecolor=blue,colorlinks=true, linkcolor=blue, urlcolor=blue]{hyperref}
\usepackage{kantlipsum}
\usepackage{makeidx}
\usepackage[margin=0.8in]{geometry}
\usepackage{mathrsfs}
\usepackage[outputdir=../build]{minted}
\usepackage{multicol}
\usepackage[mode=tex]{standalone}
\usepackage[style=authortitle]{biblatex}
\usepackage[T1]{fontenc}
\usepackage[tableaux]{prooftrees}
\usepackage{tcolorbox}
\usepackage{tikz}
\usepackage{titlesec}
\usepackage{xcolor}

\usetikzlibrary{arrows}
\usetikzlibrary{arrows.meta}
\usetikzlibrary{automata}
\usetikzlibrary{calc}
\usetikzlibrary{fit}
\usetikzlibrary{petri}
\usetikzlibrary{positioning}

\tcbuselibrary{breakable}
\tcbuselibrary{listings}
\tcbuselibrary{minted}
\tcbuselibrary{skins}
\tcbuselibrary{theorems}

\newcounter{filePrg}

\addbibresource{biblio.bib}
\setlength{\parindent}{0pt}

\renewcommand{\emph}[1]{\textit{#1}}
\setlength{\parindent}{0pt}

\newcommand\setboxcounter[2]{\setcounter{tcb@cnt@#1}{#2}}
\newcommand\qed[0]{\blacksquare}
\setlength{\parindent}{10pt}
\newcommand{\set}[1]{\{#1\}}
\newcommand{\up}[1]{ \left\lceil#1\right\rceil }
\newcommand{\down}[1]{ \left\lfloor#1\right\rfloor}

\definecolor{CaribbeanBlue}{RGB}{0, 206, 209} % Define Caribbean Blue
\NewTcbTheorem[list inside=definition]{definition}
{Definition}{
	breakable,
	colback=CaribbeanBlue!05,
	colframe=CaribbeanBlue!35!black,
	fonttitle=\bfseries}{th}

\NewTcbTheorem[list inside=intuition]{intuition}{Intuition}{
	breakable,
	colback=blue!5,
	colframe=blue!35!black,
	fonttitle=\bfseries}{th}

\NewTcbTheorem{example}{Example}{
	breakable,
	colback=white,
	colframe=green!35!black,
	fonttitle=\bfseries}{th}

\NewTcbTheorem{verify}{Verify}{
	breakable,
	float,
	colback=red!5,
	colframe=red!35!black,
	fonttitle=\bfseries}{th}

\NewTcbTheorem[list inside=theorem]{theorem}{Theorem}{
	breakable,
	colback=gray!10,
	colframe=gray!35!black,
	fonttitle=\bfseries}{th}

% \NewTcbTheorem[
% list inside=exercise,
% number within=chapter,
% number within=section,
% ]

\NewTcbTheorem[
	list inside=exercise,
	number within=section,
]{exercise}{Exercise}{
	breakable,
	colback=white,
	colframe=black,
	fonttitle=\bfseries}{th}

\newcommand{\hask}[1]{\mintinline{haskell}{#1}}

\newenvironment{alist}
{\begin{enumerate}[label={*}, leftmargin=*, itemsep=0pt, parsep=0pt]}
		{\end{enumerate}}

\newenvironment{blist}
{\begin{enumerate}[label={}, leftmargin=*, itemsep=0pt, parsep=0pt]}
		{\end{enumerate}}

\renewcommand{\thesection}{\arabic{section}}
\tcbset{enhanced jigsaw}

\newtcbinputlisting{\codeFromFile}[2]{
	listing file={#1},
	listing engine=minted,
	minted style=colorful,
	minted language=haskell,
	minted options={breaklines,linenos,numbersep=3mm},
	colback=blue!5!white,colframe=blue!75!black,listing only,
	left=5mm,enhanced,
	title={#2},
	overlay={\begin{tcbclipinterior}\fill[red!20!blue!20!white] (frame.south west)
				rectangle ([xshift=5mm]frame.north west);\end{tcbclipinterior}}
}

\newtcblisting{haskell}[1]
{
	listing engine=minted,
	minted style=colorful,
	minted language=haskell,
	minted options={breaklines,linenos,numbersep=3mm},
	colback=blue!5!white,colframe=blue!75!black,listing only,
	left=5mm,enhanced,
	title={#1},
	overlay={\begin{tcbclipinterior}\fill[red!20!blue!20!white] (frame.south west)
				rectangle ([xshift=5mm]frame.north west);\end{tcbclipinterior}}
}


\begin{document}
\section{Equivalence Relations}
\begin{exercise}{}{}{1. Let $A=\{1,2,3,4,5,6\}$, and consider the following
		equivalence relation on $A$ :
		\[
			R=\set{(1,1),(2,2),(3,3),(4,4),(5,5),(6,6),(2,3),(3,2),(4,5),(5,4),(4,6),(6,4),(5,6),(6,5)}
		\]
		List the equivalence classes of $R$.
	}
	\begin{alist}
		\item $[1]=\set{(1, 1)}$
		\item $[2]=\set{(2, 2), (2, 3), (3, 2), (3, 3)}$
		\item $[4]=\set{(4, 4), (4, 5), (5, 4), (4, 6), (6, 4), (5,5), (5, 6), (6, 5) }$
	\end{alist}
\end{exercise}{}{}

\begin{exercise}{}{}{2. Let $A=\{a, b, c, d, e\}$. Suppose $R$ is an equivalence
		relation on $A$. Suppose $R$ has two equivalence classes. Also $a R d, b R
			c$ and $e R d$. Write out $R$ as a set.}
	\begin{alist}
		\item $[a] = \set{(a, a), (d, d), (e, e), (a, d), (d, a), (d, e), (e, d), (e,
				a), (a, e)}$.
		\item $[b] = \set{(b, b), (c, c), (b, c), (c, b)}$.
	\end{alist}
\end{exercise}{}{}

\begin{exercise}{}{}{3. Let $A=\{a, b, c, d, e\}$. Suppose $R$ is an equivalence
		relation on $A$. Suppose $R$ has three equivalence classes. Also $a R d$ and $b
			R c$. Write out $R$ as a set.}
	\begin{alist}
		\item $[a] = \set{(a, a), (d, d), (a, d), (d, a)}$.
		\item $[b] = \set{(b, b), (c, c), (b, c), (c, b)}$.
		\item $[e] = \set{(e, e)}$.
	\end{alist}
\end{exercise}{}{}

\begin{exercise}{}{}{4. Let $A=\{a, b, c, d, e\}$. Suppose $R$ is an equivalence
		relation on $A$. Suppose also that $a R d$ and $b R c, e R a$ and $c R e$. How
		many equivalence classes does $R$ have?}
	\begin{alist}
		\item $a$ is related to $d$ and $e$, which is related to $c$, which is related
		to $b$. Therefore this one equivalence class traverses the entire set, so
		there is just one equivalence class.
	\end{alist}
\end{exercise}{}{}

\begin{exercise}{}{}{5. There are two different equivalence relations on the set
		$A=\{a, b\}$. Describe them. Diagrams will suffice.}
	\begin{alist}
		\item One equivalence relation is $\set{(a, a), (a, b), (b, a), (b, b)}$.
		\item One equivalence relation is $\set{(a, a), (b, b)}$.
	\end{alist}
\end{exercise}{}{}

\begin{exercise}{}{}{6. There are five different equivalence relations on the set
		$A=\{a, b, c\}$. Describe them all. Diagrams will suffice.}
	\begin{alist}
		\item One equivalence relation admits 1 equivalence class which has $a=b=c$.
		\item One equivalence relation admits 2 equivalence classes which has $a=b\neq c$.
		\item One equivalence relation admits 2 equivalence classes which has $a=c\neq b$.
		\item One equivalence relation admits 3 equivalence classes which has $b=c\neq a$.
		\item One equivalence relation admits 3 equivalence classes which has $a\neq b\neq c$.
	\end{alist}
\end{exercise}{}{}

\begin{exercise}{}{}{7. Define a relation $R$ on $\mathbb{Z}$ as $x R y$ if and
		only if $3 x-5 y$ is even. Prove $R$ is an equivalence relation. Describe its
		equivalence classes.}
	\begin{alist}
		\item $R\subseteq \mathbb{Z}\times\mathbb{Z}\mid 3x-5y=2c, c\in\mathbb{N}$.
		\item For any $x, y$, $3x+5y=3x+5y$, therefore $R$ is reflexive.
		\item For any $x, y$, $3x+5y=2c$, for some integer $c$. Because $3x+5y$ is even,
		$x$ and $y$ must have the same parity.  Therefore $3y+5x$ will also be even,
		thus we have shown $R$ is symmetric.
		\item For any $x, y, z$, $3x+5y=2c \land 3y+5z=2d$, for some integers $c, d$.
		Because $3x+5y$ is even, $x$ and $y$ must have the same parity.
		Because $3y+5z$ is even, $y$ and $z$ must have the same parity. Therefore,
		$x$ and $z$ have the same parity and we have shown that $R$ is transitive.
		\item Because $R$ is reflexive, symmetric, and transitive, it is an equivalence
		relation.
		\item Now that we have proven $R$ is an equivalence relation, we can show its
		equivalence classes. The equivalence classes are defined by parity,
		therefore all evens form one class, and the odds another class.
	\end{alist}
\end{exercise}{}{}

\begin{exercise}{}{}{8. Define a relation $R$ on $\mathbb{Z}$ as $x R y$ if and
		only if $x^2+y^2$ is even. Prove $R$ is an equivalence relation. Describe its
		equivalence classes.}
	\begin{alist}
		\item For any $x, y$, $x^2+y^2=2c=x^2+y^2$, for some integer $c$. Therefore $R$
		is reflexive.
		\item For any $x, y$, $x^2+y^2=2c$. Suppose $x=2a$ and $y=2b+1$, then
		$x^2+y^2 = 4a^2 + 4b^2 + 4b + 1 = 2(a^2 + b^2 + 2b) + 1$, which is not even.
		Therefore x cannot be odd while y is even. Through a similar procedure, we
		can show that x cannot be even while y is odd. Therefore $x$ and $y$ cannot
		have a different parity for the relation to hold.
		\item For any $x, y$, $x^2+y^2=2c$. Suppose $x=2a$ and $y=2b$, then
		$x^2+y^2 = 4a^2 + 4b^2 = 2(a^2 + b^2)$, which is even.
		Therefore x and y both being even numbers will let the relation hold.
		\item For any $x, y$, $x^2+y^2=2c$. Suppose $x=2a+1$ and $y=2b+1$, then
		$x^2+y^2 = 4a^2 + 4a + 1 + 4b^2 + 4b + 1 = 2(a^2 + b^2 + 2b + 2a + 1)$, which is even.
		Therefore x and y both being odd numbers will let the relation hold.
		\item If x and y have the same parity, then $x^2+y^2$ is even and also $y^2+x^2$
		will be even. Therefore $R$ is symmetric.
		\item If x and y and z have the same parity, then $x^2+y^2$ is even $y^2+z^2$ is
		even, and $x^2+y^2-y^2 + z^2$, will be the difference between two even
		integers, which is also even. Therefore $R$ is transitive.
		\item The set $\mathbb{Z}$ is partitioned into two classes over the relation
		$R$, namely the evens and odds.
	\end{alist}
\end{exercise}{}{}

\begin{exercise}{}{}{9. Define a relation $R$ on $\mathbb{Z}$ as $x R y$ if and
		only if $4 \mid(x+3 y)$. Prove $R$ is an equivalence relation. Describe its
		equivalence classes.}
	\begin{alist}
		\item
		\begin{align*}
			4  & \mid(x+3 y) \\
			4a & =x+3 y
		\end{align*}
		\item For any $x,y\in\mathbb{Z}$, $4a=x+3y=x+3y$, for some integer $a$,
		therefore $R$ is reflexive.
		\item For any $x,y\in\mathbb{Z}$, $4a=x+3y$, for some integer $a$.
		\begin{align*}
			4a          & =x+3y                 \\
			4(4a)       & =4x+12y               \\
			4(4a)       & =(y+3x) + (x+3y) + 8y \\
			4(4a)       & =(y+3x) + 4a + 8y     \\
			4(3a)-4(2y) & =y+3x                 \\
			4(3a-2y)    & =y+3x                 \\
		\end{align*}
		Therefore $4\mid y+3x$, and $R$ is symmetric.
		\item For any $x,y,z\in\mathbb{Z}$, $4a=x+3y \land 4b=y+3z$, for some integers $a, b$.
		\begin{align*}
			4a       & =x+3y      \\
			4b       & =y+3z      \\
			4(a+b)   & =x+3y+y+3z \\
			4(a+b)   & =x+4y+3z   \\
			4(a+b-y) & =x+3z
		\end{align*}
		\item Therefore $R$ is transitive.
	\end{alist}
\end{exercise}{}{}

\begin{exercise}{}{}{10. Suppose $R$ and $S$ are two equivalence relations on a set
		$A$. Prove that $R \cap S$ is also an equivalence relation. (For an example
		of this, look at Figure 11.2. Observe that for the equivalence relations $R_2,
			R_3$ and $R_4$, we have $R_2 \cap R_3=R_4$.)}
	\begin{alist}
		\item Suppose $(x, x) \in R\cap S$. Therefore $(x, x)$ is also in $R \cap S$ and
		we have shown the intersection is reflexive.
		\item Suppose $(x, y) \in R\cap S$.  Because $R$ and $S$ are both equivalence
		relations, $(y, x$) is in both $R$ and $S$ and therefore it will also be in
		$R\cap S$, by the definition of intersection. Therefore $R\cap S$ is
		symmetric.
		\item Suppose $(x, y), (y, z) \in R\cap S$.  Because $R$ and $S$ are both equivalence
		relations, $(x, z$) is in both $R$ and $S$ and therefore it will also be in
		$R\cap S$, by the definition of intersection. Therefore $R\cap S$ is
		transitive.
		\item Because $R\cap S$ is reflexive, symmetric, and transitive, we have shown
		that $R\cap S$ is an equivalence relation.
	\end{alist}
\end{exercise}{}{}

\begin{exercise}{}{}{11. Prove or disprove: If $R$ is an equivalence relation on an
		infinite set $A$, then $R$ has infinitely many equivalence classes.}
	\begin{alist}
		\item Suppose $A=\mathbb{Z}$, and $R$ is the equivalence relation defined by an
		integer's parity. Therefore there will be equivalence classes (even and
		odd).
		\item Therefore, by example, we have shown an example of an infinite set with a
		finite number of equivalence classes, thus disproving the statement
		\quote{
			If $R$ is an equivalence relation on an
			infinite set $A$, then $R$ has infinitely many equivalence classes.
		}
	\end{alist}
\end{exercise}{}{}

\begin{exercise}{}{}{12. Prove or disprove: If $R$ and $S$ are two equivalence
		relations on a set $A$, then $R \cup S$ is also an equivalence relation on $A$.}
	\begin{alist}
		\item Suppose $(x, x) \in R\cup S$. Therefore $(x, x)$ is also in $R
			\cup S$ and we have shown the intersection is reflexive.
		\item Suppose $(x, y) \in R\cup S$.  Because $R$ and $S$ are both
		equivalence relations, $(y, x$) is in both $R$ or $S$ and therefore
		it will also be in $R\cup S$, by the definition of union. Therefore
		$R\cup S$ is symmetric.
		\item Suppose $(x, y)\in R \land (y, z) \in S$. Therefore $(x, z), (y,
			z)$ will be in $R\cup S$. For $R\cup S$ to be transitive, $(x, z)$
		must be in either $R$ or $S$, but there is no justification to
		believe it is in either. Therefore we can not prove that $R\cup S$
		is transitive.
		\item Because $R\cup S$  we cannot prove that $R\cup S$ is transitive,
		it is not an equivalence relation.
	\end{alist}
\end{exercise}

\begin{exercise}{}{}
	{13. Suppose $R$ is an equivalence relation on a finite set
		$A$, and every equivalence class has the same cardinality $m$. Express $|R|$
		in terms of $m$ and $|A|$.}
	\begin{align*}
		|R| = m \cdot |A|
	\end{align*}
\end{exercise}{}{}

\begin{exercise}{}{}
	{14. Suppose $R$ is a reflexive and symmetric relation on a
		\mbox{finite set $A$.}}
	\begin{alist}
		\item Define a relation $S$ on $A$ by declaring $x S y$,
		\item if and only if for some $n \in \mathbb{N}$ there are elements $x_1, x_2, \ldots, x_n \in A$
		\item satisfying $x R x_1, x_1 R x_2, x_2 R x_3, x_3 R x_4, \ldots, x_{n-1} R x_n$, and $x_n R y$
		\item Show that $S$ is an equivalence relation and $R \subseteq S$.
		\item Prove that $S$ is the unique smallest equivalence relation on $A$ containing $R$.
		\begin{align*}
			A=                     & \set{1, 2, 9}                                \\
			R\subseteq A\times A=  & \set{(1, 1), (2, 2), (9, 9), (2, 9), (9, 2)} \\
			4S\subseteq A\times A= & \set{}
		\end{align*}
	\end{alist}
\end{exercise}{}{}

\begin{exercise}{}{}
	{15. Suppose $R$ is an equivalence relation on a set $A$, with
		four equivalence classes. How many different equivalence relations $S$ on $A$
		are there for which $R \subseteq S$ ?}
	\begin{alist}
		\item $R$ can be partitioned into 4 different subsets, each one an equivalence
		class.
		\item Let $T$ be the set with elements that are the equivalence classes of $R$.
		\item Each element of $T$ is also an equivalence relation. The
		intersection of equivalence relations yields another equivalence
		relation.Therefore this question is the same as for the cardinality of the
		powerset of $T$.
		\item Therefore there are $\mathscr{P}(T) = 2^4 = 16$.
		\item But the answer in the book is 15...?
	\end{alist}
\end{exercise}{}{}

\begin{exercise}{}{}
	{16. Show that the relation $\doteq$ defined on page 213 is transitive.}
\end{exercise}{}{}

\end{document}
