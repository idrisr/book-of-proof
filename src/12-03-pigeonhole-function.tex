\documentclass{article}
\input{preamble}

\begin{document}
\section{Pigeonhole Principle Revisited}

\begin{exercise}{}{}
	{1. Prove that if six integers are chosen at random, then at
		least two of them will have the same remainder when divided by 5 .}
	\begin{alist}
		\item Let $A=\set{a_1, a_2, a_3, a_4, a_5, a_6}$.
		\item Let $B=\set{a\mid a \bmod 5, a\in\mathbb{N}}$.
		\item Let $f=A\rightarrow B$ be the function from $A$ to $B$.  Because
		$|A|=6>|B|=5$, $f$ cannot be injective and at least 2 elements from $a, b\in
			A, a\neq b$ result in $f(a)=f(b)$.
		\item
	\end{alist}
\end{exercise}{}{}

\begin{exercise}{}{}
	{2. Prove that if $a$ is a natural number, then there exist two
		unequal natural numbers $k$ and $\ell$ for which $a^k-a^{\ell}$ is divisible
		by 10 .}
	\begin{alist}
		\item $f = (a^k - a^l) \mod 10$.
		\item Let $A=\mathbb{N}\times\mathbb{N}$
		\item Let $B=\set{a\mid a<10, a\in\mathbb{N}}$.
		\item Let $f=A\rightarrow B$ be the function from $A$ to $B$.
		\item Because the cardinality of $A$ is $\infty$ and the cardinality of $B$
		is 10, then $f$ cannot be injective. Therefore there exists values $k,
			l\in\mathbb{N}, k\neq l$ such that $(a^k-a^l)\mod 10\equiv 0$.
		\item
	\end{alist}
\end{exercise}{}{}

\begin{exercise}{}{}
	{3. Prove that if six natural numbers are chosen at random,
		then the sum or difference of two of them is divisible by 9 .}
	\begin{alist}
		\item Let $A=\set{a_1, a_2, a_3, a_4, a_5, a_6\mid a_i\in\mathbb{N}}$, $|A|=6$.
		\item Let $B=\set{\set{0}, \set{1, 8},\set{2, 7},\set{3, 6},\set{4, 5}}$,
		$|B|=5$.
		\item Let $f=A\rightarrow B$ be the function from $A$ to $B$. Because $|A|>|B|$,
		$f$ is not injective. The function $f$ takes a natural number $a$ and then maps
		the value $a\mod 9=b$ to the element of $B$ which contains $b$. For example
		$f(10)=\set{1, 8}$, $f(40) = \set{4, 5}$, and $f(-18) = \set{0}$.

		Because $f$ is not injective, 2 elements $a_i, a_j\in A$ will map to the same element
		in $B$. We now show that $9\mid a_i+a_j \lor 9\mid a_i-a_j$.

		\item First we consider the case when $f(a_i) = f(a_j) \neq \set{0}$.
		Suppose $a_i = 9\cdot n + r$, where $r$ is the remainder after
		dividing by 9 and $n$ is some integer. Similarly $a_j=9\cdot m + s$,
		where $s$ is the remainder and $m$ is some integer. Because of the
		way $B$ is defined, $m+n=9$.
		\begin{align*}
			a_i =       & 9\cdot n + r                \\
			a_j =       & 9\cdot m + s                \\
			a_i + a_j = & 9\cdot n + r + 9\cdot m + s \\
			a_i + a_j = & 9(n + r) + m + s            \\
			a_i + a_j = & 9(n + r) + 9                \\
			a_i + a_j = & 9(n + r + 1)
		\end{align*}
		Therefore $9\mid a_i + a_j$.
		\item Now we consider the case when $f(a_i) = f(a_j)= \set{0}$. Then $9\mid a_i
			\land 9\mid a_j$.
		\begin{align*}
			a_i =       & 9\cdot n            \\
			a_j =       & 9\cdot m            \\
			a_i + a_j = & 9\cdot n + 9\cdot m \\
			a_i + a_j = & 9(n + r)
		\end{align*}
		Therefore $9\mid a_i + a_j$.
		\item Thus we have proven the following statement
		\quote{If six natural numbers are chosen at random,
			then the sum or difference of two of them is divisible by 9.}
		\item
	\end{alist}
\end{exercise}{}{}

\begin{exercise}{}{}
	{4. Consider a square whose side-length is one unit. Select any
		five points from inside this square. Prove that at least two of these points
		are within $\frac{\sqrt{2}}{2}$ units of each other.}
	\begin{alist}
		\item Divide the unit square into 4 equal squares by bisecting each side of the
		original square. This results in 4 equal squares with sides of length
		$\dfrac{1}{2}$. The maximum distance between two points in the same smaller square is
		given by the diagonal of the smaller square. That diagonal $c$ has length
	\end{alist}
	\begin{align*}
		a^2 + b^2 =                                               & c^2 \\
		\left(\frac{1}{2}\right)^2 + \left(\frac{1}{2}\right)^2 = & c^2 \\
		\frac{1}{2} =                                             & c^2 \\
		\sqrt{\frac{1}{2}} =                                      & c   \\
		\frac{\sqrt{2}}{2} =                                      & c
	\end{align*}
	By the pigeonhole principle, at least 2 points must be in the same square, and
	the max distance of those points is $\frac{\sqrt{2}}{2}$.
\end{exercise}{}{}

\begin{figure}
	\centering
	\includegraphics[width=0.5\textwidth]{images/12-03-04.png}
\end{figure}

\begin{exercise}{}{}{5. Prove that any set of seven distinct natural numbers
		contains a pair of numbers whose sum or difference is divisible by 10 .}
	\begin{alist}
		\item Let $A=\set{a_1, a_2, a_3, a_4, a_5, a_6, a_7\mid a_i\in\mathbb{N}}$, $|A|=7$.
		\item Let $B=\set{\set{0}, \set{1, 9},\set{2, 8},\set{3, 7},\set{5}, \set{4, 6}}$,
		$|B|=6$.
		\item Let $f=A\rightarrow B$ be the function from $A$ to $B$. Because $|A|>|B|$,
		$f$ is not injective. The function $f$ takes a natural number $a$ and then maps
		the value $a\mod 10=b$ to the element of $B$ which contains $b$. For example
		$f(10)=\set{0}$, $f(41) = \set{1, 9}$, and $f(-18) = \set{2, 8}$.
		Because $f$ is not injective, 2 elements $a_i, a_j\in A$ will map to the same element
		in $B$. We now show that $10\mid a_i+a_j \lor 10\mid a_i-a_j$.
		\item First we consider the case when $f(a_i) = f(a_j) \neq \set{0}$. Suppose
		$a_i = 10\cdot n + r$, where $r$ is the remainder after dividing by
		10 and $n$ is some integer. Similarly $a_j=10\cdot m + s$, where $s$ is the
		remainder and $m$ is some integer. Because of the way $B$ is defined,
		$m+n=10$.
		\begin{align*}
			a_i =       & 10\cdot n + r                 \\
			a_j =       & 10\cdot m + s                 \\
			a_i + a_j = & 10\cdot n + r + 10\cdot m + s \\
			a_i + a_j = & 10(n + r) + m + s             \\
			a_i + a_j = & 10(n + r) + 10                \\
			a_i + a_j = & 10(n + r + 1)
		\end{align*}
		Therefore $10\mid a_i + a_j$.
		\item Now we consider the case when $f(a_i) = f(a_j)= \set{0}$. Then $10\mid a_i
			\land 10\mid a_j$.
		\begin{align*}
			a_i =       & 10\cdot n             \\
			a_j =       & 10\cdot m             \\
			a_i + a_j = & 10\cdot n + 10\cdot m \\
			a_i + a_j = & 10(n + r)
		\end{align*}
		Therefore $10\mid a_i + a_j$.
		\item Thus we have proven the following statement
		\quote{Any set of seven distinct natural numbers
			contains a pair of numbers whose sum or difference is divisible by 10.}
	\end{alist}
\end{exercise}{}{}

\begin{exercise}{}{}
	{6. Given a sphere $S$, a great circle of $S$ is the
		intersection of $S$ with a plane through its center. Every great circle
		divides $S$ into two parts. A hemisphere is the union of the great circle and
		one of these two parts. Prove that if five points are placed arbitrarily on $S$,
		then there is a hemisphere that contains four of them.}
	\begin{alist}
		\item Select two of the five arbitrary points. With those two points and the
		center of the sphere, define a plane that bisects the sphere. The
		intersection of that plane and the sphere is a great circle $G$.
		\item The remaining three points are placed in the two bisections defined by
		$G$. By the pigeonhole principle, one of those bisections $B_1$ has two of
		the remaining points.
		\item A hemisphere is defined as the union of one of the bisections caused the
		by the great circle and the great circle itself. Therefore $H_1 = G \cup B_1$. $G$ has two points and
		$B_1$ has two points, therefore $H_1$ has four points.
	\end{alist}
\end{exercise}{}{}

\begin{exercise}{}{}
	{7. Prove or disprove: Any subset $X \subseteq\set{1,2,3, \ldots,
				2 n}$ with $|X|>n$ contains two (unequal) elements $a, b \in X$ for which
		$a \mid b$ or $b \mid a$.}
	\begin{alist}
		\item Let $A=\set{1,2,3, \ldots, 2 n}$.
		\item Let $X\subseteq A$ such that $|X|>n$.
		\item Let $E(x)$ mean $x$ is even.
		\item Let $O(x)$ mean $x$ is odd.
		\item The set $A$ can be partitioned by parity---evens and odds.
		\item Let $B=\set{x\in A \mid E(x)}$. Each $b\in B$ can be written as $b=2c$,
		for some integer $c\in A$.

		\item Let $A_1=\set{1,2,3,4, 5, 6, 7, 8, 9, 10 }, n=5$.
		\item Let $X_1=\set{1, 3, 5, 7, 9}, |X_1|=6$.

		\item  Let us partition $A$ into two sets $B=\set{1\dots n}$ and
		$C=\set{n+1\dots 2n}$.
		\item Each element of $C$ is either even or odd and can be written as either
		$2a$ or $2b+1$. For the even elements in $C$,
		\item maybe partition into 4 sets?
		\item Use the techniques from how to prove it.
		\item To use the pigeonhole principle, devise a function that proves the point
		and is not injective
		\item $f :: \mathbb{N} \rightarrow ? $
		\item to linger upon ...
	\end{alist}
\end{exercise}{}{}

\begin{exercise}{}{}
	{7-2. Prove or disprove: Any subset $X \subseteq\set{1,2,3, \ldots,
				2 n}$ with $|X|>n$ contains two (unequal) elements $a, b \in X$ for which
		$a \mid b$ or $b \mid a$.}
	\begin{alist}
		\item Let $A=\set{1,2,3, \ldots, 2 n}$.
		\item Let $B=$ set of prime factors for a natural number $n$.
		\item Let $X\subseteq A$ such that $|X|>n$.
		\item Let $f=\mathbb{N} \rightarrow B $ be a function that takes a natural
		number and returns a set which contains sets as elements. The elemental-sets
		are organized such that
		\item Example $f(9) = \set{3}$, $f(12) = \set{2, 2, 3}$.
		\begin{align*}
			f(1) =  & \set{1}    \\
			f(2) =  & \set{2}    \\
			f(3) =  & \set{3}    \\
			f(4) =  & \set{2}    \\
			f(5) =  & \set{5}    \\
			f(6) =  & \set{2, 3} \\
			f(7) =  & \set{7}    \\
			f(8) =  & \set{2}    \\
			f(9) =  & \set{3}    \\
			f(10) = & \set{2, 5}
		\end{align*}
	\end{alist}
\end{exercise}{}{}

\begin{exercise}{}{}
	{7-3. Prove or disprove: Any subset $X \subseteq\set{1,2,3, \ldots,
				2 n}$ with $|X|>n$ contains two (unequal) elements $a, b \in X$ for which
		$a \mid b$ or $b \mid a$.}
	\begin{alist}
		\item Let $A=\set{1,2,3, \ldots, 2 n}$.
		\item Let $B=$ set of prime factors for a natural number $n$.
		\item Let $X\subseteq A$ such that $|X|>n$.
		\item Let $f=\mathbb{N} \rightarrow B $ be a function that takes a natural
		number and returns a set which contains sets as elements. The elemental-sets
		are organized such that
		\item $f$ puts the element into a set with its highest factor if not prime.
		\item $B=\set{\set{2, 4, 8, 16,\dots}, \set{3, 6, 9, \dots}, \set{5, 25, 625,
					\dots}}$.
		\begin{align*}
			f(1) =  & \set{1}    \\
			f(2) =  & \set{2}    \\
			f(3) =  & \set{3}    \\
			f(4) =  & \set{2}    \\
			f(5) =  & \set{5}    \\
			f(6) =  & \set{2, 3} \\
			f(7) =  & \set{7}    \\
			f(8) =  & \set{2}    \\
			f(9) =  & \set{3}    \\
			f(10) = & \set{2, 5}
		\end{align*}
	\end{alist}
\end{exercise}{}{}

\end{document}
