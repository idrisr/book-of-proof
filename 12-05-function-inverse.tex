\documentclass{hippoidC}

\memoto{Idris}
\memosubject{Book of Proof}
\memodate{2024.03.24}
\status{\S 12.5 Inverse}

\begin{document}
\toc
\thispagestyle{styleTOC}
\pagebreak
\pagestyle{styleE}

\begin{prooflist}{1. Check that $f: \mathbb{Z} \rightarrow \mathbb{Z}$ defined
		by $f(n)=6-n$ is bijective. Then compute $f^{-1}$.}
	\inj{}
	\item Suppose some $a, b \in \mathbb{Z}$ and that $f(a) = f(b)$.
	\begin{align*}
		f(a) & =f(b) \\
		6-a  & =6-b  \\
		a    & =b
	\end{align*}
	Therefore we have proved the contrapositive of the injective implication and $f$
	is injective.
	\surj{}
	\item Suppose some $b\in \mathbb{Z}$.
	\begin{align*}
		b   & =f(a) \\
		b   & =6-a  \\
		b-6 & =-a   \\
		6-b & =a
	\end{align*}
	Therefore we have shown that there exists an $a\in\mathbb{Z}$ such that $f(a)=b$
	for all $b$, therefore $f$ is surjective. Since $f$ is surjective and injective
	it is also bijective. Because $f$ is bijective, it's inverse $f^{-1}$ does exist
	such that $a=f\circ f^{-1}(a)$.
	\begin{align*}
		f(x)=y=    & 6-x                                           \\
		x=         & 6-y &  & \text{swap x and y to get to }f^{-1} \\
		x-6=       & -y                                            \\
		6-x=       & y                                             \\
		f^{-1}(x)= & 6-x
	\end{align*}
\end{prooflist}

\begin{prooflist}{2. In Exercise 9 of Section 12.2 you proved that $f:
			\mathbb{R}-\{2\} \rightarrow \mathbb{R}-\{5\}$ defined by $f(x)=\frac{5
				x+1}{x-2}$ is bijective. Now find its inverse.}
	\item
	\begin{align*}
		f(x)=y=            & \frac{5 x+1}{x-2}                                           \\
		x=                 & \frac{5y+1}{y-2}  &  & \text{swap x and y to get to }f^{-1} \\
		(y-2)x=            & 5y+1                                                        \\
		yx-2x=             & 5y+1                                                        \\
		-2x-1=             & 5y-yx                                                       \\
		-2x-1=             & y(5-x)                                                      \\
		\frac{-2x-1}{5-x}= & y                                                           \\
		f^{-1}(x) =        & \frac{-2x-1}{5-x}
	\end{align*}
\end{prooflist}

\begin{prooflist}{3. Let $B=\left\{2^n: n \in \mathbb{Z}\right\}=\left\{\ldots,
			\frac{1}{4}, \frac{1}{2}, 1,2,4,8, \ldots\right\}$. Show that the function
		$f: \mathbb{Z} \rightarrow B$ defined as $f(n)=2^n$ is bijective. Then find
		$f^{-1}$.}
	\inj{}
	\begin{align*}
		f(a)     & =f(b)     \\
		2^a      & =2^b      \\
		\log 2^a & =\log 2^b \\
		a \log 2 & =b \log 2 \\
		a        & =b
	\end{align*}
	\surj{}
	\begin{align*}
		b                     & =f(a)     \\
		b                     & =2^a      \\
		\log b                & =a \log 2 \\
		\frac{\log b}{\log 2} & = a       \\
		\frac{b}{2}           & = a       \\
		\frac{b}{2}           & = a       \\
	\end{align*}
	Therefore $f$ is surjective and injective.
	\begin{align*}
		f(x)=y=     & 2^x                                                  \\
		x=          & 2^y        &  & \text{swap x and y to get to }f^{-1} \\
		\log_2 x=   & \log_2 2^y                                           \\
		\log_2 x=   & y                                                    \\
		f^{-1}(x) = & \log_2 x
	\end{align*}
\end{prooflist}

\begin{prooflist}{4. The function $f: \mathbb{R} \rightarrow(0, \infty)$ defined
		as $f(x)=e^{x^3+1}$ is bijective. Find its inverse.}
	\item
	\begin{align*}
		f(x)=y=            & e^{x^3+1}                                                   \\
		x=                 & e^{y^3+1}         &  & \text{swap x and y to get to }f^{-1} \\
		\ln x=             & \ln e^{y^3+1}                                               \\
		\ln x=             & y^3+1                                                       \\
		\ln x-1=           & y^3                                                         \\
		\sqrt[3]{\ln x-1}= & y                                                           \\
		f^{-1}(x) =        & \sqrt[3]{\ln x-1}
	\end{align*}
\end{prooflist}

\begin{prooflist}{5. The function $f: \mathbb{R} \rightarrow \mathbb{R}$ defined
		as $f(x)=\pi x-e$ is bijective. Find its inverse.}
	\item
	\begin{align*}
		f(x)=y=          & \pi x-e                                                   \\
		x=               & \pi y-e         &  & \text{swap x and y to get to }f^{-1} \\
		x+e=             & \pi y                                                     \\
		\frac{x+e}{\pi}= & y                                                         \\
		f^{-1}(x) =      & \frac{x+e}{\pi}
	\end{align*}
\end{prooflist}

\begin{prooflist}{6. The function $f: \mathbb{Z} \times \mathbb{Z} \rightarrow
			\mathbb{Z} \times \mathbb{Z}$ defined by the formula $f(m, n)=(5 m+4 n, 4
			m+3 n)$ is bijective. Find its inverse.}
	\item
	\begin{align*}
		f(m, n)=(j, k)= & (5 m+4 n, 4 m+3 n)                                               \\
		(m, n)=         & (5 j+4 k, 4 j+3 k) &  & \text{swap m,n and j,k to get to }f^{-1} \\
		m=              & 5 j+4 k                                                          \\
		n=              & 4 j+3 k                                                          \\
		3m=             & 15 j+12 k                                                        \\
		4n=             & 16 j+12 k                                                        \\
		3m-4n=          & -j                                                               \\
		4n-3m=          & j                                                                \\
		4n=             & 16 (4n-3m)+12 k                                                  \\
		4n=             & 64n-48m+12 k                                                     \\
		48m-60n=        & 12 k                                                             \\
		4m-5n=          & k                                                                \\
		f^{-1}(m, n) =  & (4n-3m, 4m-5n)
	\end{align*}
\end{prooflist}

\begin{prooflist}{7. Show that the function $f: \mathbb{R}^2 \rightarrow
			\mathbb{R}^2$ defined by the formula $f(x, y)=\left(\left(x^2+1\right) y,
			x^3\right)$ is bijective. Then find its inverse.}
	\inj{}
	\begin{align*}
		f(a, b)                               & =f(c, d)                                \\
		\left(\left(a^2+1\right)b, a^3\right) & = \left(\left(c^2+1\right)d, c^3\right) \\
		a^3                                   & =c^3                                    \\
		a                                     & =c                                      \\
		(a^2+1)b                              & = (c^2+1)d                              \\
		(a^2+1)b                              & = (a^2+1)d                              \\
		b                                     & = d
	\end{align*}
	Therefore we have shown that $f$ is injective by proving the injective
	implication contrapositive.
	\surj{}
	\begin{align*}
		(c, d)   & = f(a, b)                               \\
		(c, d)   & = \left(\left(a^2+1\right)b, a^3\right) \\
		a^3      & =d                                      \\
		a        & =\sqrt[3]{d}                            \\
		(a^2+1)b & =c                                      \\
		b        & =\dfrac{c}{a^2+1}                       \\
		b        & =\dfrac{c}{d^{\frac{2}{3}}+1}
	\end{align*}
	Therefore we have shown that $f$ is surjective by showing that for any arbitrary
	$(c, d)$ in the codomain, we can find an $(a, b)$ in the domain such that $f(a,
		b)=(c, d)$.
	\item We can re-use the calculations from the surjetive proof to determine the inverse
	$f^{-1}(a, b)$.
	\[f^{-1}(x, y) = \left(\dfrac{x}{y^{\frac{2}{3}}+1}, \sqrt[3]{x}\right)
	\]
\end{prooflist}

\begin{prooflist}{8. Is the function $\theta: \mathscr{P}(\mathbb{Z})
			\rightarrow \mathscr{P}(\mathbb{Z})$ defined as $\theta(X)=\bar{X}$
		bijective? If so, find $\theta^{-1}$.}
	\item
	\inj{}
	\begin{align*}
		\theta(X)    & =\theta(Y)    \\
		\overline{X} & =\overline{Y} \\
	\end{align*}
	If two sets are equal, they're the same set. Therefore $\theta$ is injective.
	\surj{}
	\begin{align*}
		\theta(X)    & =Y            \\
		\overline{X} & =Y            \\
		X            & =\overline{Y} \\
	\end{align*}
	For any value in the $Y$ codomain, we can find a value $X$ in the domain such
	that $\theta{X}=Y$, therefore $\theta$ is injective, surjective, and bijective.
	To find the inverse, note that $\theta$ takes a set $X$, and subtracts it from
	the universe of $U=\mathbb{Z}$, such that $\theta(X)=U-X$. To do the inverse, we
	must subtract that from the universe such that $\theta^{-1}=U-(U-X)=X$.
\end{prooflist}

\begin{prooflist}{9. Consider the function $f: \mathbb{R} \times \mathbb{N}
			\rightarrow \mathbb{N} \times \mathbb{R}$ defined as $f(x, y)=(y, 3 x y)$.
		Check that this is bijective; find its inverse.}
	\item nah, i get it.
\end{prooflist}

\begin{prooflist}{10. Consider $f: \mathbb{N} \rightarrow \mathbb{Z}$ defined as
		$f(n)=\frac{(-1)^n(2 n-1)+1}{4}$. This function is bijective by Exercise 18
		in Section 12.2. Find its inverse.}
	\item nah, i get it.
\end{prooflist}

\end{document}
