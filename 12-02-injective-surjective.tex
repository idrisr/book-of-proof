\documentclass{idrisMemo}
\usepackage{bookOfProof}

\memoto{Idris}
\memosubject{Book of Proof}
\memodate{2024.03.22}
\status{\S 12.2 Injective and Surjective Functions}

\begin{document}
\toc
\thispagestyle{styleTOC}
\pagebreak
\pagestyle{styleE}

\begin{prooflist}{1. Let $A=\{1,2,3,4\}$ and $B=\{a, b, c\}$. Give an example of
    a function $f: A \rightarrow B$ that is neither injective nor surjective.}
\inj{}
\item In other words, for $f$ to not be injective, two elements in $A$ must lead
    to the same element $B$. Therefore we will construct the set $\set{(1, a),
    (2, a)}$ to create a non-injective function.
\surj{}
\item In other words, for $f$ to not be surjective, there must be an element
    $b\in B$ which has no counterpart $a \in A$ such that $f(a) = b$. Therefore the set
    $\set{(1, a), (2, a)}$ defines a function which is neither surjective nor
    injective.
\end{prooflist}

\begin{prooflist}{2. Consider the logarithm function $\ln :(0, \infty)
    \rightarrow \mathbb{R}$. Decide whether this function is injective and
    whether it is surjective.}
\item For $f$ to be injective the following implication must hold
\begin{align*}
    \forall a, b \in A \land a\neq b\implies f(a) \neq f(b)
\end{align*}
\item Let $a, b\in (0, \infty)\mid a\neq b$. Then $\ln a \neq \ln b$, so $f$ is
    injective.
\item For $f$ to be surjective the following implication must hold
\begin{align*}
    \forall b \in B \implies a \in A \mid f(a) = b
\end{align*}
\item The codomain of $f$ is $\mathbb{R}$, and $\ln$ does indeed have a range
    that is also $\mathbb{R}$, therefore $\ln$ is both injective and surjective.
\end{prooflist}

\begin{prooflist}{3. Consider the cosine function $\cos : \mathbb{R} \rightarrow
\mathbb{R}$. Decide whether this function is injective and whether it is
surjective. What if it had been defined as $\cos : \mathbb{R} \rightarrow[-1,1]$
?}
\item Cosine is not injective. For every value $\cos 2n\pi, n\in \mathbb{Z}$,
    the function will map to the same value in the codomain.
\item Cosine is not surjective when the domain is $\mathbb{R}$, and it is
    surjective when the domain is the range $[-1, 1]$.
\end{prooflist}

\begin{prooflist}{4. A function $f: \mathbb{Z} \rightarrow \mathbb{Z} \times
\mathbb{Z}$ is defined as $f(n)=(2 n, n+3)$. Verify whether this function is
injective and whether it is surjective.}
\inj{}
\item Suppose $a, b\in A \land f(a)=f(b)$. Then $f(a)=f(b)=(2a, a+3)=(2b, b+3)$.
    If $2a=2b$ then $a=b$ and if $a+3=b+3$ then $a=b$, therefore we have shown
    that $a=b$. Thus we have used the contrapositive to show that $f$ is
    injective.
\surj{}
Suppose that $(x, y) \in B$, then for some $(x,y)\in B$, $x=2a, y=a+3$. For any
integer $a$, there is an integer $2a$, and for any
integer $a$, there is an integer $a+3$. Therefore we have shown that $f$ is
surjective.
\end{prooflist}

\begin{prooflist}{5. A function $f: \mathbb{Z} \rightarrow \mathbb{Z}$ is
    defined as $f(n)=2 n+1$. Verify whether this function is injective and
whether it is surjective.}
\inj{}
\item Suppose $a, b \in \mathbb{Z}$ and $f(a) = f(b)$. Then $2a+1=2b+1$, which
    proves that $a=b$. Thus by the contrapositive, we have shown that $f$ is
    injective.
\surj{}
\item Suppose some integer $a$.  There also exists an integer $2a+ 1$, therefore
    $f$ is surjective.
\end{prooflist}

\begin{prooflist}{6. A function $f: \mathbb{Z} \times \mathbb{Z} \rightarrow
    \mathbb{Z}$ is defined as $f(m, n)=3 n-4 m$. Verify whether this function is
injective and whether it is surjective.}
\inj{}
\item Let $A=\mathbb{Z}\times\mathbb{Z}$ and $B=\mathbb{Z}$.
\item Suppose $(a, b), (c, d) \in A$, and therefore
    $3a-4b=3c-4d$. Suppose $a=4, b=3, c=-4, d=-3$, then the $0=0$ when $a\neq c$
    and $b\neq d$. Therefore we have disproved the contrapositive of the
    injection implication, and thus $f$ is not injective.
\surj{}
\item Suppose some integer $c$, and an integer pair $(a, b)$. Thus $f(a,
    b)=c=3a-4b$. Additionally suppose that $a=b$. Then $c = -b$, and thus it will
    always be possible to construct the integer $c$ from a pair of equal
    integers. Thefore $f$ is surjective.
\end{prooflist}

\begin{prooflist}{7. A function $f: \mathbb{Z} \times \mathbb{Z} \rightarrow
    \mathbb{Z}$ is defined as $f(m, n)=2 n-4 m$. Verify whether this function is
injective and whether it is surjective.}
\inj{}
\item Let $A=\mathbb{Z}\times\mathbb{Z}$ and $B=\mathbb{Z}$.
\item Suppose $(a, b), (c, d) \in A \land f(a, b)=f(c, d)$, and therefore
    $2a-4b=2c-4d$. Suppose $a=2, b=0, c=0, d=-1$, then $0=0$ when $a\neq c$
    and $b\neq d$. Therefore we have disproved the contrapositive of the
    injection implication, and thus $f$ is not injective.
\surj{}
\item Suppose some integer $c$, and an integer pair $(a, b)$. Thus $f(a,
    b)=c=2a-4b$. Because $2a$ and $4b$ are even integers, their difference will
    be an even integer, therefore it's not possible for $f$ to result in an odd
    integer, and thus $f$ is not surjective.
\end{prooflist}

\begin{prooflist}{8. A function $f: \mathbb{Z} \times \mathbb{Z} \rightarrow
        \mathbb{Z} \times \mathbb{Z}$ is defined as \mbox{$f(m, n)=(m+n, 2 m+n)$}. Verify
whether this function is injective and whether it is surjective.}
\inj{}
\item Let $A=\mathbb{Z}\times\mathbb{Z}$ and $B=\mathbb{Z}\times\mathbb{Z}$.
\item Suppose $(a, b), (c, d) \in A \land f(a, b)=f(c, d)$, and then
    substituting those values into $f$ produces
\begin{align*}
    f(a, b)&=f(c, d)\\
    (a+b, 2a+b)&=(c+d, 2c+d)\\
    a+b&=c+d \\
    2a+b&=2c+d \\
    3a+2b &= 3c+2d
\end{align*}
Suppose $a=2, b=0, c=0, d=3$, then we get $f(a, b)=f(c, d)$ and $a\neq b \land
c\neq d$. Therefore by counterexample we have disproven the contrapositive, and
thus $f$ is not injective.
\surj{}
Suppose some integer pair $(c, d)\in B$, and an integer pair $(a, b)\in
    A$. Thus $f(a, b)=(c, d)=(a+b, 2a+b)$.
\begin{align*}
    c&=a+b\\
    d&=2a+b \\
    c+d&=3a+2b \\
\end{align*}
\item To be continuted$\dots$
\end{prooflist}

\begin{prooflist}{9. Prove that the function $f: \mathbb{R}-\{2\} \rightarrow
    \mathbb{R}-\{5\}$ defined by $f(x)=\dfrac{5 x+1}{x-2}$ is bijective.}
\inj{}
\item Let $A=\mathbb{R}-\set{2}$, and let $B=\mathbb{R}-\set{5}$.
\item Suppose that $a, b\in A$ and $f(a)=f(b)$. Then
    \begin{align*}
        \dfrac{5 a+1}{a-2}&=\dfrac{5 b+1}{b-2}\\
    (5 a+1)(b-2)&=(5 b+1)(a-2)\\
    5ab+b+10a-2&= 5ab+a+10b-2\\
    b+10a&= a+10b\\
    10a&= a+9b\\
    9a&= 9b\\
    a&=b
    \end{align*}
Therefore $a=b$, and we have proven the contrapositive, and thus $f$ is injective.
\surj{}
Suppose some integer $b\in B$. Then $b = f(a) = \dfrac{5a+1}{a-2}$.
    \begin{align*}
        f(a) = b &= \dfrac{5a+1}{a-2}\\
        5a+1 &= b(a-2)\\
        5a-ba &= -1-2b\\
        a(5-b)=-1-2b\\
        a=\dfrac{-1-2b}{5-b}&&\text{we can divide by }5-b\text{ because }
        B=\mathbb{R}-\set{5}\\
    \end{align*}
Therefore $\forall b\in B, \exists a \in A \mid f(a)=b$.  Thus $f$ is surjective
and injective, and thus is bijective.
\end{prooflist}

\begin{prooflist}{10. Prove the function $f: \mathbb{R}-\{1\} \rightarrow
    \mathbb{R}-\{1\}$ defined by $f(x)=\left(\dfrac{x+1}{x-1}\right)^3$ is
bijective.}
\item Let $A=B=\mathbb{R}-\set{1}$.
    \inj{}
Suppose $a, b \in A$ and that $f(a)=f(b)$.
\begin{align*}
f(a)=f(b)= \left(\dfrac{a+1}{a-1}\right)^3=& \left(\dfrac{b+1}{b-1}\right)^3\\
\sqrt[3]{\left(\dfrac{a+1}{a-1}\right)^3}=& \sqrt[3]{\left(\dfrac{b+1}{b-1}\right)^3}\\
\dfrac{a+1}{a-1}=& \dfrac{b+1}{b-1}\\
(a+1)(b-1)=& (b+1)(a-1)\\
ab-a+b-1=& ab+a-b-1\\
-a+b=& a-b\\
b=& 2a-b\\
2b=& 2a\\
b=& a
\end{align*}
Therefore we have proven the contrapositive, and thus $f$ is injective.
    \surj{}
Suppose $b\in B$, then $f(a)=b= \left(\dfrac{a+1}{a-1}\right)^3$.
    \begin{align*}
        f(a) = b &= \left(\dfrac{a+1}{a-1}\right)^3\\
        \sqrt[3]{b} &= \dfrac{a+1}{a-1}\\
        (a-1)\sqrt[3]{b} &= {a+1}\\
        a\sqrt[3]{b} - \sqrt[3]{b}&= a+1\\
        a\sqrt[3]{b} - a &= 1+ \sqrt[3]{b}\\
        a(\sqrt[3]{b} - 1) &= 1+ \sqrt[3]{b}\\
        a &= \dfrac{1+ \sqrt[3]{b}}{\sqrt[3]{b} - 1}
    \end{align*}
    Because the codomain $B$ excludes 1, the denominator will never be 0, and
    thus $\exists a\forall b, f(a)=b$, and thus $f$ is surjective and
    injective, and thus bijective.
\end{prooflist}

\begin{prooflist}{11. Consider the function $\theta:\{0,1\} \times \mathbb{N}
        \rightarrow \mathbb{Z}$ defined as \mbox{$\theta(a, b)=(-1)^a b$}. Is $\theta$
injective? Is it surjective? Bijective? Explain.}
\item Let $A=\set{0,1}$.
\item Let $B=\mathbb{N}$.
\item Let $C=A\times B$.
\item Let $D=\mathbb{Z}$.
\inj{}
\item Suppose $(a, b), (c, d) \in A$ and that $\theta(a, b)=\theta(c, d)$.
Because $A=\set{0, 1}$, $(-1)^a, a\in A$ will be positive when $a=0$, and
negative when $a=1$. Thus multiplying $(-1)^a$ by a natural number $b\in B$ will
map the natural numbers to the negative numbers, the positive numbers, and zero.
Only one such natural number $n\in B$ will lead to either $n$ or $-n$, depending on
$a\in A$, thus $\theta$ is injective.
\surj{}
\item Suppose $c\in D$, then we wish to prove that $\exists (a, b) \in C$ such
    that $\theta(a, b)=c$.
\begin{align*}
    \theta(a, b) = c &= (-1)^a\cdot b\\
    c &= (-1)^a\cdot b\\
c &= b && \text{when }a=0\\
c &= -b && \text{when }a=1\\
\end{align*}
Thus for any integer $c$, it can be constructed by $\theta$. Thus $\theta$
is injective, surjective, and by definition bijective.
\end{prooflist}

\begin{prooflist}{12. Consider the function $\theta:\{0,1\} \times \mathbb{N}
    \rightarrow \mathbb{Z}$ defined as $\theta(a, b)=a-2 a b+b$. Is $\theta$
injective? Is it surjective? Bijective? Explain.}
\inj{}
\item Let $A=\set{0,1}$.
\item Let $B=\mathbb{N}$.
\item Let $C=A\times B$.
\item Let $D=\mathbb{Z}$.
\item Suppose $(a, b), (c, d) \in C$ and $\theta(a, b)=\theta(c, d)$.
\begin{align*}
    \theta(a, b)=\theta(c, d)=2a - 2ab+b =& 2c - 2cd+d&&a=0,c=0\\
    b =& d&&a=0,c=0\\
    \theta(a, b)=\theta(c, d)=2a - 2ab+b =& 2c - 2cd+d&&a=1,c=1\\
    2 - 2b+b =& 2 - 2d+d&&a=1,c=1\\
    -b =& -d&&a=1,c=1\\
    b =& d&&a=1,c=1
\end{align*}
We consider two cases, one where $a=c=0$ and one where $a=c=1$. In both cases, we
show that $b=d$, therefore $\theta$ is injective.
\surj{}
\item Suppose that $c\in D$. Therefore
\begin{align*}
    \theta(a, b)=c = & a - 2ab+b &  & a=0 \\
    =           & b         &  & a=0 \\
    \theta(a, b)=c = & 1 - 2b+b  &  & a=1 \\
    =           & 1-b       &  & a=1
\end{align*}
All integers $c$ can be created by $c=b$, or $c=1-b$, for $b\in\mathbb{N}$.
Therefore $\theta$ is surjective and injective, and by definition bijective.
\end{prooflist}

\begin{prooflist}{13. Consider the function $f: \mathbb{R}^2 \rightarrow
    \mathbb{R}^2$ defined by the formula $f(x, y)=\left(x y, x^3\right)$. Is $f$
injective? Is it surjective? Bijective? Explain.}
\inj{}
\item Let $A=\mathbb{R}^2$.
\item Suppose $(a, b), (c, d) \in A$ and $f(a, b)=f(c, d)$.
\begin{align*}
    f(a, b)=f(c, d)=(ab, a^3) =& (cd, d^3)\\
    (ab, a^3) =& (cd, d^3)\\
    a^3=& d^3\\
    a=& d\\
    db=& cd\\
    b=& c
\end{align*}
\item Thus we have proven the contrapositive by showing that if $f(a, b)=f(c,
    d)$ then $a=b \land c=d$ and therefore $f$ is injective.
\surj{}
\item Suppose $(a, b)\in A$. Therefore $f(a, b)=(c, d)$.
\begin{align*}
    f(a, b)=(c,d)=&(ab, a^3)\\
    a^3=&d\\
    a=&\sqrt[3]{d}\\
    ab=&c\\
    b=&\dfrac{c}{\sqrt[3]{a}}
\end{align*}
Because $\dfrac{c}{\sqrt[3]{a}}$ is undefined for $a=0$, it is not true that
there exists an $(a, b)\in A$ such that $f(a, b)=(c, d), \forall (c, d) \in A$,
therefore $f$ is not surjective and not bijective.
\end{prooflist}

\begin{prooflist}{14. Consider the function $\theta: \mathscr{P}(\mathbb{Z})
    \rightarrow \mathscr{P}(\mathbb{Z})$ defined as $\theta(X)=\overline{X}$. Is
$\theta$ injective? Is it surjective? Bijective? Explain.}
\item Let $A=\mathscr{P}(\mathbb{Z})$.
\inj{}
\item Suppose $a, b \in A$ and $\theta(a)=\theta(b)$.
\begin{align*}
    \theta(a)=\theta(b)=\overline{a} =& \overline{b}\\
    \mathbb{Z} - a =& \mathbb{Z} -b \\
    a =& b
\end{align*}
Therefore $\theta$ is injective.
\surj{}
\item Suppose some $a\in A$. Therefore $\theta(a)=b\in A$.
\begin{align*}
    \theta(a)=b=&\bar{a}
\end{align*}
For any set $A$ of integers, there exists a set $\overline{A}$ of integers which excludes all
elements of $A$. Therefore $\theta$ is injective and surjective, and thus
bijective.
\end{prooflist}

\begin{prooflist}{15. This question concerns functions $f:\{A, B, C, D, E, F,
    G\} \rightarrow\{1,2,3,4,5,6,7\}$. How many such functions are there? How
many of these functions are injective? How many are surjective? How many are
bijective?}
\item Let $A= \set{A, B, C, D, E, F, G}$.
\item Let $B= \set{1,2,3,4,5,6,7}$.
\item A function is a set, and therefore we can measure its cardinality. For any
    function $f: A\rightarrow B$, its cardinality is measured by $|f|=|B|^{|A|}=7^7$.
\inj{}
\item For $f$ to be injective each element of $A$ must map to only one element
    of $B$. We can consider this problem to be analagous to selecting elements
    from set $B$ to match an element of $A$, without replacement. Therefore we
    can use the multiplication principle, and first choose $|B|$ selections,
    then $|B|-1$, and so on until all elements of $A$ have been matched.
    Therefore, for any $f=A\rightarrow  B$, the number of injective functions is
    \begin{align*}
    |f_{\text{injective}}| =&\dfrac{|B|!}{(|B|-|A|)!}&&|B|\geq |A|\\
    |f_{\text{injective}}| =&0 &&|B|<|A|
    \end{align*}
\surj{}
For $f$ to be surjective, we can consider it the dual to being injective, as if
we were turning around the direction of the arrow and then use the same
analysis.
    \begin{align*}
        |f_{\text{surjective}}| =&\dfrac{|A|!}{(|A|-|B|)!}&&|A|\geq |B|\\
        |f_{\text{surjective}}| =&0 &&|A|<|B|\\
    \end{align*}
For $f$ to be bijective, $|A|=|B|$ must be true. The number of bijective
functions is $|A|!=|B|!$.

\end{prooflist}

\begin{prooflist}{16. This question concerns functions $f:\{A, B, C, D, E\}
    \rightarrow\{1,2,3,4,5,6,7\}$. How many such functions are there? How many
of these functions are injective? How many are surjective? How many are
bijective?}
\item Let $A= \set{A, B, C, D, E}$, $|A|=5$.
\item Let $B= \set{1,2,3,4,5,6,7}$, $|B|=7$.
\item $|f|=|B|^{|A|}=7^5$.
\item $|f_{\text{injective}}|= \dfrac{|B|!}{(|B|-|A|)!}= \dfrac{7!}{2!}$.
\item $|f_{\text{surjective}}|= 0$.
\item $|f_{\text{bijective}}|= 0$.
\end{prooflist}

\begin{prooflist}{17. This question concerns functions $f:\{A, B, C, D, E, F,
    G\} \rightarrow\{1,2\}$. How many such functions are there? How many of
these functions are injective? How many are surjective? How many are bijective?}
\item Let $A= \set{A, B, C, D, E, F, G}$, $|A|=7$
\item Let $B= \set{1,2}$, $|B|=2$.
\item $|f|=|B|^{|A|}=2^7$.
\item $|f_{\text{injective}}|= 0$
\item $|f_{\text{surjective}}|= \dfrac{|A|!}{(|A|-|B|)!}= \dfrac{7!}{5!}$.
\item $|f_{\text{bijective}}|= 0$.
\end{prooflist}

\begin{prooflist}{18. Prove that the function $f: \mathbb{N} \rightarrow
    \mathbb{Z}$ defined as \mbox{$f(n)=\dfrac{(-1)^n(2 n-1)+1}{4}$} is bijective.}
\item To show that $f$ is bijective, we must show it's injective and surjective.
\item Let $O(x)$ mean x is odd and let $E(x)$ mean x is even.
\inj{}
\item Suppose $a, b \in \mathbb{N}$ and $f(a)=f(b)$.
\begin{align*}
    f(a)=f(b)=\dfrac{(-1)^a(2 a-1)+1}{4} =& \dfrac{(-1)^b(2 b-1)+1}{4}\\
    \dfrac{(-1)^a(2 a-1)+1}{4} =& \dfrac{(-1)^b(2 b-1)+1}{4}&&O(a)\land O(b)\\
    \dfrac{2 a}{4} =& \dfrac{2b}{4}&&O(a)\land O(b)\\
    a =& b&&O(a)\land O(b)\\
    \dfrac{(-1)^a(2 a-1)+1}{4} =& \dfrac{(-1)^b(2 b-1)+1}{4}&&E(a)\land E(b)\\
    \dfrac{1 - 2a + 1}{4} =& \dfrac{1-2b+1}{4}&&E(a)\land E(b)\\
    \dfrac{2 - 2a}{4} =& \dfrac{2-2b}{4}&&E(a)\land E(b)\\
    \dots &&&E(a)\land E(b)\\
    a =& b&&E(a)\land E(b)
\end{align*}
\surj{}
\item Suppose some $b \in \mathbb{Z}$. Now we attempt to prove $\exists
    a\in\mathbb{N}\mid f(a)=b$.
\begin{align*}
    f(a)=b&=\dfrac{(-1)^a(2 a-1)+1}{4} &&E(a)\\
    b&=\dfrac{1-2a+1}{4} &&E(a)\\
    b&=\dfrac{1-a}{2} &&E(a)\\
    2b&=1-a &&E(a)\\
    a&=1-2b &&E(a)\\
    f(a)=b&=\dfrac{(-1)^a(2 a-1)+1}{4} &&O(a)\\
    b&=\dfrac{2a}{4} &&O(a)\\
    b&=\dfrac{a}{2} &&O(a)\\
    a&=2b &&O(a)
\end{align*}
\item Therefore we have show that $a=2b-1$ when a is odd. Every odd natural number can
be constructted by $2b-1$ for some $b\in\mathbb{Z}$.
\item We have also shown that $a=2b$ when a is even. Every even natural number can
be constructted by $2b$ for some $b\in\mathbb{Z}$.
\item Therefore we have shown that $f$ is both injective and surjective, and
    therefore is also bijective.
\end{prooflist}

\end{document}
