\documentclass{hippoidC}

\memoto{Idris}
\memosubject{Book of Proof}
\memodate{2024.03.24}
\status{\S 12.4 Composition}

\begin{document}
\toc
\thispagestyle{styleTOC}
\pagebreak
\pagestyle{styleE}

\begin{prooflist}{1. Suppose $A=\{5,6,8\}, B=\{0,1\}, C=\{1,2,3\}$. Let $f: A
			\rightarrow B$ be the function $f=$ $\{(5,1),(6,0),(8,1)\}$, and $g: B
			\rightarrow C$ be $g=\{(0,1),(1,1)\}$. Find $g \circ f$.}
	\item $g \circ f = g(f(x)), \forall x\in A$
	\begin{align*}
		g \circ f(5) & = g(f(5)) = g(1) = 1 \\
		g \circ f(6) & = g(f(6)) =g(0) = 1  \\
		g \circ f(8) & = g(f(8)) =g(1) = 1  \\
	\end{align*}
	\item Therefore $g\circ f$ is the set $\set{(5, 1), (6, 1), (8, 1)}$.
\end{prooflist}

\begin{prooflist}{2. $g \circ f$.}
	\item $A=\{1,2,3,4\}$
	\item $B=\{0,1,2\}$
	\item $C=\{1,2,3\}$
	\item $f: A \rightarrow B$, $f=\{(1,0),(2,1)$, $(3,2),(4,0)\}$
	\item $g: B \rightarrow C$, $g=\{(0,1),(1,1),(2,3)\}$
	\item $g \circ f = g(f(x)), \forall x\in A$
	\begin{align*}
		g \circ f(1) & = g(f(1)) = g(0) = 1 \\
		g \circ f(2) & = g(f(2)) = g(1) = 1 \\
		g \circ f(3) & = g(f(3)) = g(2) = 3 \\
		g \circ f(4) & = g(f(4)) = g(0) = 1 \\
	\end{align*}
	\item Therefore $g\circ f$ is the set $\set{(1, 1), (2, 1), (3, 3), (4, 1)}$.
\end{prooflist}

\begin{prooflist}{3. Find $g \circ f$ and $f \circ g$.}
	\item
	$A=\{1,2,3\}$
	\item
	$f: A \rightarrow A$, $f=\{(1,2),(2,2),(3,1)\}$
	\item
	$g: A \rightarrow A$, $g=\{(1,3),(2,1),(3,2)\}$.
	\item $g \circ f = g(f(x)), \forall x\in A$
	\begin{align*}
		g \circ f(1) & = g(f(1)) = g(2) = 1 \\
		g \circ f(2) & = g(f(2)) = g(2) = 1 \\
		g \circ f(3) & = g(f(3)) = g(1) = 3
	\end{align*}
	\item Therefore $g\circ f$ is the set $\set{(1, 1), (2, 1), (1, 3)}$.
	\item $f \circ g = f(g(x)), \forall x\in A$
	\begin{align*}
		f \circ g(1) & = f(g(1)) = f(3) = 1 \\
		f \circ g(2) & = f(g(2)) = f(1) = 2 \\
		f \circ g(3) & = f(g(3)) = f(2) = 2
	\end{align*}
	\item Therefore $f\circ g$ is the set $\set{(1, 1), (2, 2), (3, 2)}$.
\end{prooflist}

\begin{prooflist}{4. Find $g \circ f$ and $f \circ g$.}
	\item $A=\set{a, b, c}$.
	\item $f: A \rightarrow A$ be the function $f=\{(a, c),(b, c),(c, c)\}$
	\item $g: A \rightarrow A$ be the function $g=\{(a, a),(b, b),(c, a)\}$.
	\item Find $g \circ f$ and $f \circ g$.
	\item $g \circ f = g(f(x)), \forall x\in A$
	\begin{align*}
		g \circ f(a) & = g(f(a)) = g(c) = a \\
		g \circ f(b) & = g(f(b)) = g(c) = a \\
		g \circ f(c) & = g(f(c)) = g(c) = a
	\end{align*}
	\item Therefore $g\circ f$ is the set $\set{(a, a), (b, a), (c, a)}$.
	\item $f \circ g = f(g(x)), \forall x\in A$
	\begin{align*}
		f \circ g(a) & = f(g(a)) = f(a) = c \\
		f \circ g(b) & = f(g(b)) = f(b) = c \\
		f \circ g(c) & = f(g(c)) = f(a) = c
	\end{align*}
	\item Therefore $f\circ g$ is the set $\set{(a, c), (b, c), (c, c)}$.
\end{prooflist}

\begin{prooflist}{5. Consider the functions $f, g: \mathbb{R} \rightarrow
			\mathbb{R}$ defined as $f(x)=\sqrt[3]{x+1}$ and $g(x)=x^3$. Find the
		formulas for $g \circ f$ and $f \circ g$.}
	\item $g \circ f = g(f(x)), \forall x\in \mathbb{R}$
	\begin{align*}
		g \circ f(x) = & g(f(x))                      \\
		=              & g\left(\sqrt[3]{x+1}\right)  \\
		=              & \left(\sqrt[3]{x+1}\right)^3 \\
		=              & x+1
	\end{align*}
	\item $f \circ g = f(g(x)), \forall x\in \mathbb{R}$
	\begin{align*}
		f \circ g(x) = & f(g(x))         \\
		=              & f(x^3)          \\
		=              & \sqrt[3]{x^3+1}
	\end{align*}
\end{prooflist}

\begin{prooflist}{6. Consider the functions $f, g: \mathbb{R} \rightarrow
			\mathbb{R}$ defined as $f(x)=\frac{1}{x^2+1}$ and $g(x)=3 x+2$. Find the
		formulas for $g \circ f$ and $f \circ g$.}
	\item $g \circ f = g(f(x)), \forall x\in \mathbb{R}$
	\begin{align*}
		g \circ f = & g(f(x))                           \\
		=           & g\left(\frac{1}{x^2+1}\right)     \\
		=           & 3\left(\frac{1}{x^2+1}\right) + 2 \\
		=           & \frac{3}{x^2+1} + 2               \\
	\end{align*}
	\item $f \circ g = f(g(x)), \forall x\in \mathbb{R}$
	\begin{align*}
		f \circ g = & f(g(x))                \\
		=           & f(3x^2+1)              \\
		=           & \frac{1}{(3x^2+1)^2+1} \\
	\end{align*}
\end{prooflist}

\begin{prooflist}{7. Consider the functions $f, g: \mathbb{Z} \times \mathbb{Z}
			\rightarrow \mathbb{Z} \times \mathbb{Z}$ defined as $f(m, n)=(m n,
			m^2)$ and $g(m, n)=(m+1, m+n)$. Find the formulas for $g \circ f$
		and $f
			\circ g$.}
	\item $g \circ f = g(f((m, n))), \forall (m, n)\in \mathbb{Z}\times\mathbb{Z}$
	\begin{align*}
		g \circ f = & g(f((m, n)))               \\
		=           & g\left( (m n, m^2) \right) \\
		=           & (mn + 1, mn + m^2)
	\end{align*}
	\item $f \circ g = f(g((m, n))), \forall (m, n)\in \mathbb{Z}\times\mathbb{Z}$
	\begin{align*}
		f \circ g = & f(g((m, n)))         \\
		=           & f((m+1, m + n))      \\
		=           & ((m+1)(m+n),(m+1)^2)
	\end{align*}
\end{prooflist}

\begin{prooflist}{8. Consider the functions $f, g: \mathbb{Z} \times \mathbb{Z}
			\rightarrow \mathbb{Z} \times \mathbb{Z}$ defined as $f(m, n)=(3 m-4 n, 2
			m+n)$ and $g(m, n)=(5 m+n, m)$. Find the formulas for $g \circ f$ and $f \circ
			g$.}
	\item $g \circ f = g(f((m, n))), \forall (m, n)\in \mathbb{Z}\times\mathbb{Z}$
	\begin{align*}
		g \circ f = & g(f((m, n)))                  \\
		=           & g\left( (3m-4n, 2m+n) \right) \\
		=           & (5(3m-4n) + 2m+n, 3m-4n)      \\
		=           & (15m-20n+ 2m+n, 3m-4n)        \\
		=           & (17m-19n, 3m-4n)
	\end{align*}
	\item $f \circ g = f(g((m, n))), \forall (m, n)\in \mathbb{Z}\times\mathbb{Z}$
	\begin{align*}
		f \circ g = & f(g((m, n)))                \\
		=           & f((5m+n, m))                \\
		=           & (3(5m+n) - 4m, 2(5m+n) + m) \\
		=           & (11m+3n, 11m+2n)
	\end{align*}
\end{prooflist}

\begin{prooflist}{9. Consider the functions $f: \mathbb{Z} \times \mathbb{Z}
			\rightarrow \mathbb{Z}$ defined as $f(m, n)=m+n$ and $g: \mathbb{Z}
			\rightarrow \mathbb{Z} \times \mathbb{Z}$ defined as $g(m)=(m, m)$. Find the
		formulas for $g \circ f$ and $f \circ g$.}
	\item $g \circ f = g(f((m, n))), \forall (m, n)\in \mathbb{Z}\times\mathbb{Z}$
	\begin{align*}
		g \circ f = & g(f((m, n))) \\
		=           & g(m+n )      \\
		=           & (m+n, m+n)
	\end{align*}
	\item $f \circ g = f(g(m)), \forall m\in \mathbb{Z}$
	\begin{align*}
		f \circ g = & f(g(m))   \\
		=           & f((m, m)) \\
		=           & 2m
	\end{align*}

\end{prooflist}

\begin{prooflist}{10. Consider the function $f: \mathbb{R}^2 \rightarrow
			\mathbb{R}^2$ defined by the formula $f(x, y)=\left(x y, x^3\right)$. Find a
		formula for $f \circ f$.}
	\item $f \circ f = f(f((x, y))), \forall , (x, y)\in \mathbb{R}^2$.
	\begin{align*}
		f \circ f = & f(f((x, y)))  \\
		=           & f((xy, x^3) ) \\
		=           & (x^4y, x^9)
	\end{align*}
\end{prooflist}

\end{document}
