\documentclass{idrisMemo}

\usepackage{amsthm}
\usepackage{amsfonts}
\usepackage{hyperref}
\usepackage{enumitem}
\usepackage{amssymb}
\usepackage{graphicx}
\usepackage{tocloft}
\usepackage{bookOfProof}

\memoto{Idris}
\memosubject{Book of Proof}
\memodate{2024.03.12}
\status{\S 3.8 Counting Multisets}

\begin{document}
\tableofcontents
\pagebreak

\begin{prooflist}{1. How many 10-element multisets can be made from the symbols $\{ 1,2,3,4\}$?}
    \item Let $X= \{ 1,2,3,4\}$, and $|X|=4$.
    \item Let $Y=$ a multiset of cardinality 10.
    \item The total permutations of a k-multiset constructed from a set with
        n-cardinality is
        $$ \dbinom{k+n-1}{k} = \dbinom{k+n-1}{n-1} $$
    \item For this problem $k=10, n=4$.
    \item Therefore there are $\binom{13}{4}$ possible 10-element multisets made
        from the symbols $1\dots4$.
\end{prooflist}

\begin{prooflist}{2. How many 2-element multisets can be made from the 26 letters of the alphabet?}
    \item Let $A=$ 26 letter alphabet.
    \item Let $Y=$ a multiset of cardinality 2.
    \item For this problem, $k=2, n=26$
        $$ \dbinom{k+n-1}{k} = \dbinom{k+n-1}{n-1} $$
        $$ \dbinom{27}{2}$$
        $$ \dfrac{27!}{25!\cdot2!}$$
        $$ \dfrac{26\cdot27}{2}$$
\end{prooflist}

\begin{prooflist}{3. You have a dollar in pennies, a dollar in nickels, a dollar in dimes, and a dollar in quarters. You give a friend four coins. How many ways can this be done?}
    \item Let the set $A= \{a_1, a_2, a_3, a_4\}$, where $a_i$ is a coin type.
    \item The question is equivalent to: given a set of cardinality 4, how many
        different multisets of cardinality 4 can be created?
    \item The information about the different number of starting coins is
        irrelevant other than assuring there are at least $k=4$ of each coin.
    \item Therefore the answer is the following
        $$ \dbinom{k+n-1}{k} $$
        $$ \dbinom{7}{4}$$
\end{prooflist}

\begin{prooflist}{4. A bag contains 20 identical red balls, 20 identical blue
    balls, 20 identical green balls, and 20 identical white balls. You reach in
    and grab 15 balls. How many different outcomes are possible?}
    \item This question is asking about how many multisets can be created from
        an initial set. Let $a_i$ be the number of starting balls for color $i$.
        Let $k$ be the cardinality of the desired multiset. As long as $a_i\geq k$,
        then all possible multisets can be created.
    \item Therefore the answer is the following
        $$ \dbinom{k+n-1}{k} $$
        $$ \dbinom{15+4-1}{15}$$
        $$ \dbinom{18}{15}$$
\item
\end{prooflist}

\begin{prooflist}{5. A bag contains 20 identical red balls, 20 identical blue balls, 20 identical green balls, and one white ball. You reach in and grab 15 balls. How many different outcomes are possible?}
\item Let $A$=multiset of 15 of colored balls, from 3 colors
\item Let $B$=multiset of 14 of colored balls, from 3 colors
\item Let $C$=multisets of 15 of colored balls starting from 20
    identical red balls, 20 identical blue balls, 20 identical green balls, and
    one white ball.
\item We can partition $C$ into two sets: those with a white ball and those
    without a white ball. For the first partition, it is the same as asking how
    many multisets of cardinality 14 can be created from a set of 3 balls, which
    is represented by $B$. The second partition is the same as asking how many
    multisets of size 15 can be created from a set of 3 balls, which is
    represented by $A$.
\item Therefore $C=A \cup B$, and $|C| = |A| + |B|$.
\item
    $$ |A| = \dbinom{k+n-1}{k} = \dbinom{15+3-1}{15}=\dbinom{17}{15}$$
    $$ |B| = \dbinom{k+n-1}{k} = \dbinom{14+3-1}{14}=\dbinom{16}{14}$$
    $$ |C| = 256 $$
\item
\end{prooflist}

\begin{prooflist}{6. A bag contains 20 identical red balls, 20 identical blue
    balls, 20 identical green balls, one white ball, and one black ball. You
reach in and grab 20 balls. How many different outcomes are possible?}
\item Case 1, 0 white balls, 0 black balls, $k=20, n=3$
    $$ \dbinom{k+n-1}{k} = \dbinom{20+3-1}{20}=\dbinom{22}{20}$$
\item Case 2, 0 white balls, 1 black ball , $k=19, n=3$
    $$ \dbinom{k+n-1}{k} = \dbinom{19+3-1}{19}=\dbinom{21}{19}$$
\item Case 3, 1 white ball, 0 black balls, $k=19, n=3$
    $$ \dbinom{k+n-1}{k} = \dbinom{19+3-1}{19}=\dbinom{21}{19}$$
\item Case 4, 1 white ball, 1 black ball, $k=18, n=3$
    $$ \dbinom{k+n-1}{k} = \dbinom{18+3-1}{18}=\dbinom{20}{18}$$
\item Therefore the answer is
    $$ \dbinom{22}{20} +\dbinom{21}{19} +\dbinom{21}{19} +\dbinom{20}{18}$$
\end{prooflist}

\begin{prooflist}{7. In how many ways can you place 20 identical balls into five
    different boxes?}
\item This is a stars-and-bars type problem, which is the equivalent of asking,
    given a finite set X with cardinality $n$, how many multisets of cardinality
    $k$ can be created? Here the cardinality of X is 5, and k=20.
\item Therefore the answer is
    $$ \dbinom{k+n-1}{k} = \dbinom{20+5-1}{20}=\dbinom{24}{20}$$
\end{prooflist}

\begin{prooflist}{8. How many lists (x, y, z) of three integers are there with
    $0 \leq x \leq y \leq z \leq 100$?}
\item  Imagine 100 stars, and then placing 2 bars somewhere within them. This
    transforms the problems into asking how many multisets of cardinality 100
    can be created from a set of cardinality 3, which is solved by the
    following.
    $$ \dbinom{100+3-1}{100} = \dbinom{102}{100}$$
\item
\end{prooflist}

\begin{prooflist}{9. A bag contains 50 pennies, 50 nickels, 50 dimes and 50 quarters. You reach in and grab 30 coins. How many different outcomes are possible?}
\item Let $p_i$ be the count of count $i$. There are sufficient coins such that
    $\forall i, p_i\geq k$, where $k$ is the cardinality of the desired
    multiset.
\item There there are this many possible outcomes, where $n$ is the cardinality
    of the distinct coin set.
    $$ \dbinom{k+n-1}{k} = \dbinom{4+30-1}{30}=\dbinom{33}{30}$$
\end{prooflist}

\begin{prooflist}{10. How many non-negative integer solutions does $$u + v+ w+ x+
    y+ z = 90$$ have?}
\item Suppose a set of 90 stars.  We can partition those stars 6 times with 5
    bars, and then we will have transformed the original problem into an
    equivalent formulation stated as: how many multisets of cardinality 90 can
    be created from an initial set of cardinality 6, which is calculated as
    follows.
    $$ \dbinom{90+6-1}{90} = \dbinom{95}{90}$$
\end{prooflist}

\begin{prooflist}{11. How many integer solutions does the equation
        $w + x + y+ z = 100$ have if $w \geq 4, x \geq 2, y \geq 0, z \geq 0$?}
\item Let A = solutions to the problem
    $$w + x + y+ z = 100, w \geq 0, x \geq 0, y \geq 0, z \geq 0$$.
\item Let $B=w<4$ and $C=x<2$. From the set $|A|$, we must delete all solutions
    where $B\lor C$. So we are left with determining the cardinality of the
    solution set $B \lor C$, which is equal to
    $$
        |B\lor C| = |B| + |C| - |B\land C|
    $$
\item To determine $|B|$ solve this problem
$$w + x + y+ z = 100, \neg(w \geq 4), x \geq 0, y \geq 0, z \geq 0$$
$$\implies x + y+ z = 97, x \geq 0, y \geq 0, z \geq 0$$

\item To determine $|C|$ solve this problem
$$w + x + y+ z = 100, w \geq 0, \neg(x \geq 2), y \geq 0, z \geq 0$$
$$\implies w + y+ z = 99, w \geq 0, y \geq 0, z \geq 0$$

\item To determine $|B\land C|$ solve this problem
    (w, x)
    (0, 0)
    (0, 1)
    (1, 0)
    (1, 1)
    (2, 0)
    (2, 1)
    (3, 0)
    (3, 1)
\item There's a much simpler approach.
\end{prooflist}

\begin{prooflist}{12. How many integer solutions does the equation
        $$w + x + y+ z = 100, w \geq 7, x \geq 0, y \geq 5, z \geq 4$$
    }
\item This problem is the same as
        $$w + x + y+ z = 84, w \geq 0, x \geq 0, y \geq 0, z \geq 0$$
    \item Therefore the solution is calculated with $k=84, n=4$ as
    $$ \dbinom{k+n-1}{k} = \dbinom{84+4-1}{84}=\dbinom{87}{84}$$
\end{prooflist}

\begin{prooflist}{13. How many length-6 lists can be made from the symbols $\{a,
    b, c, d, e, f, g\}$, if repetition is allowed and the list is in
    alphabetical order? (Examples: bbcegg, but not bbbagg.)}
\item Creating lists is equivalent to counting permutations. Only allowing lists
    in some order is equivalent to counting multisets. Therefore this problem
    can be restated in an equivalent form as---how many cardinality 6 multisets
    can be created from an original set of cardinality 7.
\item This can be calculated as the following, with $k=6, n=7$.
    $$ \dbinom{k+n-1}{k} = \dbinom{6+7-1}{6}=\dbinom{12}{6}$$
\end{prooflist}

\begin{prooflist}{14. How many permutations are there of the letters in the word “PEPPERMINT”?}
\item This problem is the equivalent to asking---given a multiset, how many
    permutations can be made.  We must know the cardinality of the multiset, and each
    the multiplicity of each element from the underlying set.
\item Then the permutation can be calculated as follows
    $$ \dfrac{n!}{\Pi{p_i}!} = \dfrac{10!}{3!2!}$$
\end{prooflist}

\begin{prooflist}{15. How many permutations are there of the letters in the word “TENNESSEE”?}
\item The permutation can be calculated as follows
    $$ \dfrac{n!}{\Pi{p_i}!} = \dfrac{9!}{4!2!2!}$$
\end{prooflist}

\begin{prooflist}{16. A community in Canada’s Northwest Territories is known in the local language as “TUKTUYAAQTUUQ.” How many permutations does this name have?}
\item
      4 U
      3 T
      2 Q
      2 A
      1 Y
      1 K
\item The permutations can be calculated as follows
    $$ \dfrac{n!}{\Pi{p_i}!} = \dfrac{13!}{4!3!2!2!}$$
\end{prooflist}

\begin{prooflist}{17. You roll a dice six times in a row. How many possible outcomes are there that have two 1’s three 5’s and one 6?}
\item We begin with the multiset $\{1, 1, 5, 5, 5, 6\}$. We must determine how
    many permutations there are of this multiset.
\item The permutation can be calculated as follows
    $$ \dfrac{n!}{\Pi{p_i}!} = \dfrac{6!}{2!3!}$$
\end{prooflist}

\begin{prooflist}{18. Flip a coin ten times in a row. How many outcomes have 3 heads and 7 tails?}
\item We begin with the multiset of 3 heads and 7 tails. We must determine how
    many permutations there are of this multiset.
\item The permutation can be calculated as follows
    $$ \dfrac{n!}{\Pi{p_i}!} = \dfrac{10!}{7!3!}$$
\end{prooflist}

\begin{prooflist}{19. In how many ways can you place 15 identical balls into 20
    different boxes if each box can hold at most one ball?}
\item This can be encoded as a selection problem. We are choose 15 of the boxes
    to hold a single ball.
\item Therefore this can be calculated with
    $$ \dbinom{20}{15}$$
\end{prooflist}

\begin{prooflist}{20. You distribute 25 identical pieces of candy among five children. In how many ways can this be done?}
\item This can be encoded as stars and bars problem. Given a finite set of 5
    children, how many multisets of cardinality 25 can be made?
\item This can calculated with $n=5, k=25$
    $$ \dbinom{k+n-1}{k} = \dbinom{25+5-1}{29}=\dbinom{29}{25}$$
\end{prooflist}

\begin{prooflist}{21. How many numbers between 10,000 and 99,999 contain one or more of the digits 3, 4 and 8, but no others?}
\item Starting with the finite set $\{3, 4, 8\}$, how many multisets of
    cardinality 5 can be constructed? This is not a multiset problem, as that
    would imply an ordering on the digits, whereas for example 34444 and 43444
    should be counted separately, which would not happen with a multiset.
\item This can calculated with $n=3, k=5$
    $$ \dbinom{k+n-1}{k} = \dbinom{3+5-1}{5}=\dbinom{7}{5}$$
\end{prooflist}

\end{document}
