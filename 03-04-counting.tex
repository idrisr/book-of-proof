\documentclass{idrisMemo}

\usepackage{amsthm}
\usepackage{amsfonts}
\usepackage{hyperref}
\usepackage{enumitem}
\usepackage{amssymb}
\usepackage{tocloft}
\usepackage{bookOfProof}

\memoto{Idris}
\memosubject{Book of Proof}
\memodate{2024.03.10}
\status{Section 3.3}

\begin{document}

\tableofcontents
\pagebreak

\begin{prooflist}{1. What is the smallest n for which n! has more than 10 digits?}
    \item If a number is 10 digits, then it must be more than $10^{9}$.
    \item 13
\end{prooflist}

\begin{prooflist}{2. for which values of $n$ does $n!$ have n or fewer digits?}
\item Let $f$ be a function return the number of digits of a number
\item From the above question we know that $10 < f(13!) < 11$.
\item Now $14!$, will add least one more digit, and at some point around 15 or
    16 I think.
\item Just use a calculator, or stirling's approximation.
\end{prooflist}

\begin{prooflist}{3. How many 5-digit positive integers are there in which there are no repeated digits and all digits are odd? }
\item There are 5 digits to work with, namely the odd digits.
\item There are $5!$ such numbers, which are positive integers, in which no
    digit is repeated, the number is odd, and each digit is odd.
\end{prooflist}

\begin{prooflist}{4. Using only pencil and paper, find the value of
    $\dfrac{100!}{95!}$.}
\item Yeah, no.
\end{prooflist}

\begin{prooflist}{5. Using only pencil and paper, find the value of
    $\dfrac{120!}{118}$.}
\item Yeah, no.
\end{prooflist}

\begin{prooflist}{6. There are two 0’s at the end of $10! = 3,628,800$. Using
    only pencil and paper, determine how many 0’s are at the end of the number
100!.}
\item Each time the quantity $10!$ is multiplied by $10$, another $0$ will be
    appended to the end of the number.
\item We can factor out primes from the elements of the set 11-100, with the goal
    of factoring out 2s and 5s, so they can multiply together to get
    $2\cdot5=10$.
\item Between 11 and 100, 17 numbers are divisible by 5 (15, 20, ...100), and 3
    of them have 5 as a factor twice (25, 75, 100). Therefore we have 20 5's.
\item It's obvious there are at least 20 2's available in the range, so in total
    we can pair 5 and 2 together 20 times.
\item Each time we multiply by 2 and 5, we add another zero.
\item Therefore there will 20 be additional zeroes, for a total of 22 zeroes at the end
    of 100!.
\end{prooflist}

\begin{prooflist}
{Find how many 9-digit numbers can be made from the digits 1, 2, 3, 4, 5, 6, 7,
8, 9 if repetition is not allowed and all the odd digits occur first (on the left)
followed by all the even digits (i.e., as in 137598264, but not 123456789)}
\item $5! \cdot 4!$
\end{prooflist}

\begin{prooflist}{8. Compute how many 7-digit numbers can be made from the digits 1, 2, 3, 4, 5, 6, 7
if there is no repetition and the odd digits must appear in an unbroken sequence.
(Examples: 3571264 or 2413576 or 2467531, etc., but not 7234615.) }
\item First let's calculate the number of ways to create the odd sequence, which
    is $4!$.
\item There are 3 evens, therefore there are 4 spots the odd sequence can be
    interspersed within the evens.
\item Therefore there are $4\cdot 4!$ to make such a number.
\end{prooflist}

\begin{prooflist}{9. How many permutations of the letters A, B, C, D, E, F, G
    are there in which the three letters ABC appear consecutively, in
alphabetical order?}
\item There are 4 elements not part of the ABC sequence, therefore there are 5
    spots to insert the ABC sequence.
\item There the non ABC elements can be arranged in $4!$ permutations.
\item Therefore there are $4\cdot4!$ ways to make such a sequence.
\end{prooflist}

\begin{prooflist}{10. How many permutations of the digits 0, 1, 2, 3, 4, 5, 6, 7, 8, 9 are there in which the digits alternate even and odd? (For example, 2183470965.) }
\item The permutation can start 2 ways, either with an odd or an even.
\item There 5 possibilties for the 1st spot and 2nd spot, 4 for the 3rd and 4th
    spot, and so on.
\item Therefore there are $2\cdot 5!\cdot5!$. possibilities.
\end{prooflist}

\begin{prooflist}{11. You deal 7 cards off of a 52-card deck and line them up in
    a row. How many possible lineups are there in which not all cards are red?}
    \item Let $A=$ set of all 7 card lineups
    \item Let $B=$ set of all red-only 7 card lineups
    \item Let $C=$ set of all lineups in which not all the cards are red
    \item $|A| - |B| = |C|$
    \item $|A|=\dfrac{52!}{45!}$
    \item $|B|=\dfrac{13!}{6!}$
    \item $|C|=\dfrac{52!}{45!} - \dfrac{13!}{6!}$
\end{prooflist}

\begin{prooflist}{12. You deal 7 cards off of a 52-card deck and line them up in
    a row. How many possible lineups are there in which no card is a club?}
    \item For there to be no clubs, we can imagine starting with a deck of 39
        cards without any clubs.
    \item Therefore the answer is $\dfrac{39!}{32!}$.
\item
\end{prooflist}

\begin{prooflist}{13. How many lists of length six (with no repetition) can be
    made from the 26 letters of the English alphabet? }
    \item $\dfrac{26!}{20!}$
\end{prooflist}

\begin{prooflist}{14. Five of ten books are arranged on a shelf. In how many ways can this be done?}
\item Assuming that different orderings of books are considered different
    arrangements, there are $\dfrac{10!}{5!}$ possible arrangements.
\end{prooflist}

\begin{prooflist}{15. In a club of 15 people, we need to choose a president,
    vice-president, secretary, and treasurer. In how many ways can this be done?
}
\item There are 4 different positions, and the order matters, as the positions
    are not equivalent.
\item Therefore there are $\dfrac{15!}{11!}$ distinct possible arrangements.
\item
\end{prooflist}

\begin{prooflist}{16. How many 4-permutations are there of the set A,B,C,D,E,F
    if whenever A appears in the permutation, it is followed by E?}
    \item Let $X=$ 4-permutations with A.
    \item Let $Y=$ 4-permutations without A.
    \item Let $Z=$ all 4-permutations matching the specification.
    \item For the set X, AE will always appear together. Because there are 2
        remaining elements, there are 3 spots where AE can in interspersed.
    \item Therefore $|X| = 2 \cdot 4\cdot3 = 4!.$
    \item For the set Y, A will not appear by definition, therefore $|Y| =
        \dfrac{5!}{1!} = 5!$.
    \item Therefore $|Z| = |X| + |Y| = 4! + 5!$.
\end{prooflist}

\begin{prooflist}{17. Three people in a group of ten line up at a ticket counter
    to buy tickets. How many lineups are possible? }
\item $\dfrac{10!}{7!}$
\end{prooflist}

\begin{prooflist}
    {18. The gamma function provides a way of extending factorials to numbers
other than integers. Extra credit: Compute $\pi$ !.}

\item
    $$\Gamma:[0, \infty) \rightarrow \mathbb{R}$$
    $$\Gamma(x)=\int_0^{\infty} t^{x-1} e^{-t} d t$$
    $$x \in \mathbb{N} \implies \Gamma(x)=(x-1)!$$
    $$\forall n \in \mathbb{N},\quad n !=\Gamma(n+1)$$

\item Check that this is true for $x=1,2,3,4$.
\end{prooflist}

\end{document}
