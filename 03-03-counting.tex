\documentclass{idrisMemo}

\memoto{Idris}
\memosubject{Book of Proof}
\memodate{2024.03.10}
\status{Section 3.3}

\newcounter{exercise} % This defines a new counter named 'exercise'

\begin{document}

\tableofcontents
\pagebreak

\begin{prooflist}{1. Five cards are dealt off of a standard 52-card deck and
    lined up in a row.}
    \item How many such lineups are there that have at least one red card?
        \begin{itemize}
    \item Let $A=$ list of all lineups
    \item Let $B=$ list of all lineups without a red.
    \item Let $C=$ list of all lineups with at least one red.
        \end{itemize}
    \item Therefore $|A| - |B| = |C| = \dfrac{52!}{47!} - \dfrac{26!}{21!}$
    \item How many such lineups are there in which the cards are either all black or all hearts?
        \begin{itemize}
    \item Let $A=$ list of all blacks.
    \item Let $B=$ list of all hearts.
    \item Let $A \cap B= \emptyset$, because hearts are red.
        \end{itemize}
    \item Therefore $|A|+ |B| =\dfrac{26!}{21!} + \dfrac{13!}{8!}$.
\end{prooflist}

\begin{prooflist}{2. Five cards are dealt off of a standard 52-card deck and lined up in a row. How many such lineups are there in which all 5 cards are of the same suit?}
\item $4 \cdot \dfrac{13!}{8!}$
\end{prooflist}

\begin{prooflist}{3. Five cards are dealt off of a standard 52-card deck and lined up in a row. How many such lineups are there in which all 5 cards are of the same color (i.e., all black or all red)?}
\item $2 \cdot \dfrac{26!}{21!}$
\end{prooflist}

\begin{prooflist}{4. Five cards are dealt off of a standard 52-card deck and lined up in a row. How many such lineups are there in which exactly one of the 5 cards is a queen?}
\item $4 \cdot \dfrac{48!}{44!}$
\end{prooflist}

\begin{prooflist}{5. How many integers between 1 and 9999 have no repeated
digits?}
\item Let $A=1\ldots 9$
\item Let $B=0\ldots 9$
\item For the integers $1\ldots 9$, we can pick any element from $A$, therefore
    there are $9$ with non-consecutive digits.
\item For the integers $10\ldots 99$, we can pick any element from $A$, and then
    and non-matching element from $B$, therefore are $9\cdot9=81$ integers in
    this range with non-consecutive digits.
\item For the integers $C=100\ldots 999$, we can pick any element from $A$, then
    any non-matching element from $B$, and then lastly another non-matching
    element from $B$, therefore there are $9\cdot9\cdot9$ integers in this range
    with non-consecutive digits.
\item For the integers $C=1000\ldots 9999$, follows the same pattern, therefore
    there are $9\cdot9\cdot9\cdot9$ integers in this range
    with non-consecutive digits.
\item The answer is $9^1 + 9^2 + 9^3 + 9^4$.

\item How many have at least one repeated digit?
\item The set of numbers with at least one repeated digit is the inverse of the
    above calculated set, namely the set of numbers with no repeated digits.
\item Therefore the answer is $9999 - 9^1 - 9^2 - 9^3 - 9^4$.

\end{prooflist}

\begin{prooflist}{6. Consider lists made from the symbols A, B, C, D, E, with repetition allowed.}
\item (a) How many such length-5 lists have at least one letter repeated?
\item Let $A=\{\text{length-5 lists}\},\quad |A| = 5^5$
\item Let $B=\{\text{length-5 lists}\mid\text{no repeats}\}, \quad |B| = 5!$
\item Let $C=\{\text{length-5 lists}\mid\text{at least one repeat}\}$
\item $C=A - B, \quad |C| = 5^5 - 5! $
\item Herein it's easiest to use the subtraction princple, and calculate $|A|$
and $|B|$ to get $|C|$.
\item (b) How many such length-6 lists have at least one letter repeated?
\item The same principle applies, therefore the answer is $6^6 - 6!$.
\end{prooflist}

\begin{prooflist}{7. A password on a certain site must be five characters long,
    made from letters of the alphabet, and have at least one upper case letter.}
\item Let $A=\{\text{length-5 string}\},\quad |A| = 52^5$
\item Let $B=\{\text{length-5 lists}\mid\text{no upper-case}\}, \quad |B| = 26^5$
\item Let $C=\{\text{length-5 lists}\mid\text{at least one upper-case}\}$

\item How many different passwords are there?
\item $|C| = |A| - |B| = 52^5 - 26^5.$
\item What if there must be a mix of upper and lower case?
\item Let $D=\{\text{length-5 lists}\mid\text{no lower-case}\}, \quad |D| = 26^5$
\item Let $E=\{\text{length-5 lists}\mid\text{at least one lower-case, at least one
    lower-case}\}$
\item $|E| = |A| - |B| - |D| = 52^5 - 2\cdot26^5.$
\item
\end{prooflist}

\begin{prooflist}{8. This problem concerns lists made from the letters A, B, C, D, E, F, G, H, I, J.}
\item (a) How many length-5 lists can be made from these letters if repetition
    is not allowed and the list must begin with a vowel?
\item $3 \cdot 9 \cdot 8 \cdot 7 \cdot 6$.
\item (b) How many length-5 lists can be made from these letters if repetition
    is not allowed and the list must begin and end with a vowel?
\item $3 \cdot 8 \cdot 7 \cdot 6 \cdot 2$.
\item (c) How many length-5 lists can be made from these letters if repetition
    is not allowed and the list must contain exactly one A?
\item $5 \cdot 8 \cdot 7 \cdot 6 \cdot 5$.
\end{prooflist}

\begin{prooflist}{9. Consider lists of length 6 made from the letters A, B, C,
    D, E, F, G, H. How many such lists are possible if repetition is not allowed
    and the list contains two consecutive vowels?}
\item Let $X=\{\text{A, B, C, D, E, F, G, H}\}$
\item Let $Y=\{\text{B, C, D, F, G, H}\}$
\item All lists that fit the description will contain 2 vowels. First we can
    calculate the number of ways to arrange the non-vowels from $Y$, which is
    $6\cdot5\cdot4\cdot3$.
\item There are then $5$ locations to insert the consecutive vowels, and 2 ways
    to arrange the consecutive vowels, either AE or EA.
\item Therefore there are $6\cdot5\cdot4\cdot3 \cdot 5 \cdot 2$ ways to make the
    specified list.
\end{prooflist}

\begin{prooflist}{10. Consider the lists of length six made with the symbols P, R, O, F, S, where repetition is allowed. (For example, the following is such a list: (P,R,O,O,F,S).) How many such lists can be made if the list must end in an S and the symbol O is used more than once?}
\item Since the last letter must be an S, there are only 5 spots where there are
    different letters possible.
\item Therefore the question is transformed into the following: How many
    \mbox{length-five} lists can be made, with at least two Os.
\item Let $A=\{\text{length-5 lists}\}$
\item Let $B=\{\text{length-5 lists}\mid\text{0 O}\}$
\item Let $C=\{\text{length-5 lists}\mid\text{1 O}\}$
\item Let $D=\{\text{length-5 lists}\mid\text{at least 2 Os}\}$
\item $|D| = |A| - |B| - |C|$
\item $|A| = 5^5$, choose any of the 5 letters, 5 times
\item $|B| = 4^5$, choose any of the 5 letters other than O, 5 times
\item $|C| = 4^4 \cdot 5$, choose any length-4 list, and then insert an O in any
    of the 5 spots
\item Therefore, $|D| = |A| - |B| - |C| = 5^5 - 4^5 - 4^4 \cdot 5$
\end{prooflist}

\begin{prooflist}{11. How many integers between 1 and 1000 are divisible by 5? How many are not divisible by 5?}
\item There are 1000 numbers in that set, and that set is equally partitioned
    into 10 subsets by looking at the last digit.
\item A number is divisble by 5 if it ends in 0 or 5, which is $\dfrac{1}{5}$ of
    the set.
\item Therefore 200 numbers in the set of integers from 1 to 1000 are divisible
    \mbox{by 5}.
\end{prooflist}

\begin{prooflist}{12. Six math books, four physics books and three chemistry books are arranged on a shelf. How many arrangements are possible if all books of the same subject are grouped together?}
\item Within this problem there exists the subproblem of how many ways books of
    the same subject can be arranged, which is dependent of the size of the set
    $n$.
\item Therefore there are $n!$ ways to arrange books of the same category.
\item Next, we must determine how many ways there are to arrange sets of similar
    books, and again it is $m!$, where $m$ is the number of subjects.
\item Therefore there are $m! \cdot n_1! \cdot n_2! \cdot n_3!$ ways to arrange the
    books, with the constraint that books are grouped together by subject.
\item Therefore there are $3! \cdot 6! \cdot 4! \cdot 3!$ ways to arrange the
    books.
\end{prooflist}

\end{document}
