\documentclass{idrisMemo}
\usepackage{bookOfProof}

\memoto{Idris}
\memosubject{Book of Proof}
\memodate{2024.03.21}
\status{\S 11.5 Integer Modulo n}

\begin{document}
\toc
\thispagestyle{styleTOC}
\pagebreak
\pagestyle{styleE}

\begin{prooflist}{1. Write the addition and multiplication tables for $\mathbb{Z}_2$.}
\item
    \[
\begin{array}{c|cc}
\times & 0 & 1 \\
\hline
0 & 0 & 0 \\
1 & 0 & 1 \\
\end{array}
\]
\item
    \[
\begin{array}{c|cc}
+ & 0 & 1 \\
\hline
0 & 0 & 1 \\
1 & 1 & 0 \\
\end{array}
\]
\end{prooflist}

\begin{prooflist}{2. Write the addition and multiplication tables for $\mathbb{Z}_3$.}
\item \[
\begin{array}{c|ccc}
\times & 0 & 1 & 2 \\
\hline
0 & 0 & 0 & 0 \\
1 & 0 & 1 & 2 \\
2 & 0 & 2 & 1 \\
\end{array}
\]
\item \[
\begin{array}{c|ccc}
+ & 0 & 1 & 2 \\
\hline
0 & 0 & 1 & 2 \\
1 & 1 & 2 & 0 \\
2 & 2 & 0 & 1 \\
\end{array}
\]
\end{prooflist}

\begin{prooflist}{3. Write the addition and multiplication tables for $\mathbb{Z}_4$.}
\item \[
\begin{array}{c|cccc}
\times & 0 & 1 & 2 & 3 \\
\hline
0 & 0 & 0 & 0 & 0 \\
1 & 0 & 1 & 2 & 3 \\
2 & 0 & 2 & 0 & 2 \\
3 & 0 & 3 & 2 & 1 \\
\end{array} \]
\item \[
\begin{array}{c|cccc}
+ & 0 & 1 & 2 & 3 \\
\hline
0 & 0 & 1 & 2 & 3 \\
1 & 1 & 2 & 3 & 0 \\
2 & 2 & 3 & 0 & 1 \\
3 & 3 & 0 & 1 & 2 \\
\end{array} \]

\end{prooflist}

\begin{prooflist}{4. Write the addition and multiplication tables for $\mathbb{Z}_6$.}
 \item \[
\begin{array}{c|cccccc}
\times & 0 & 1 & 2 & 3 & 4 & 5 \\
\hline
0 & 0 & 0 & 0 & 0 & 0 & 0 \\
1 & 0 & 1 & 2 & 3 & 4 & 5 \\
2 & 0 & 2 & 4 & 0 & 2 & 4 \\
3 & 0 & 3 & 0 & 3 & 0 & 3 \\
4 & 0 & 4 & 2 & 0 & 4 & 2 \\
5 & 0 & 5 & 4 & 3 & 2 & 1 \\
\end{array}
\]
\item \[
\begin{array}{c|cccccc}
+ & 0 & 1 & 2 & 3 & 4 & 5 \\
\hline
0 & 0 & 1 & 2 & 3 & 4 & 5 \\
1 & 1 & 2 & 3 & 4 & 5 & 0 \\
2 & 2 & 3 & 4 & 5 & 0 & 1 \\
3 & 3 & 4 & 5 & 0 & 1 & 2 \\
4 & 4 & 5 & 0 & 1 & 2 & 3 \\
5 & 5 & 0 & 1 & 2 & 3 & 4 \\
\end{array}
\]
\end{prooflist}

\begin{prooflist}{5. Suppose $[a],[b] \in \mathbb{Z}_5$ and $[a] \cdot[b]=[0]$.
    Is it necessarily true that either $[a]=[0]$ or $[b]=[0]$ ?}
\item Assume: $[a]\cdot[b]=[0]$ for some $[a],[b]\in\mathbb{Z}_5$.
\item Translate: This means $[a\cdot b]\equiv[0]$, so $a\cdot b\equiv 0\bmod 5$
\item Property of Modulo 5: This implies that 5 divides the product $a\cdot b$.
\item Prime Divisibility: Since 5 is a prime number, it must divide either a or b.
\item Back to Equivalence Classes: Therefore, $a \equiv 0 \pmod 5$ (meaning
$[a]=[0]$) or $b \equiv 0 \pmod 5$ (meaning $[b]=[0]$).
\item \[
\begin{array}{c|ccccc}
\times & 0 & 1 & 2 & 3 & 4 \\
\hline
0 & 0 & 0 & 0 & 0 & 0 \\
1 & 0 & 1 & 2 & 3 & 4 \\
2 & 0 & 2 & 4 & 1 & 3 \\
3 & 0 & 3 & 1 & 4 & 2 \\
4 & 0 & 4 & 3 & 2 & 1 \\
\end{array}
\]
\end{prooflist}

\begin{prooflist}{6. Suppose $[a],[b] \in \mathbb{Z}_6$ and $[a] \cdot[b]=[0]$. Is it necessarily true that either $[a]=[0]$ or $[b]=[0]$ ? What if $[a],[b] \in \mathbb{Z}_7$ ?}
\item Assume: $[a]\cdot[b]=[0]$ for some $[a],[b]\in\mathbb{Z}_6$.
\item Translate: This means $[a\cdot b]\equiv[0]$, so $a\cdot b\equiv 0\bmod 6$
\item Property of Modulo 6: This implies that 6 divides the product $a\cdot b$.
\item Prime Divisibility: Since the prime factors of 6 are 2 and 3, 2 must
    factor a or b and 3 must factor a or b. Assume $3\mid a \land 2\mid b$. Let
    $a=3$ and $b=2$, and then $a\cdot b \equiv 0 \mod 6$.
\item Back to Equivalence Classes: Therefore, it's not necessarily true that
    either $[a] = [0] \lor [b] = [0]$ when $[a]\cdot[b]=[0]$ for $\mathbb{Z}_6$.
\item Assume: $[a]\cdot[b]=[0]$ for some $[a],[b]\in\mathbb{Z}_7$.
\item Translate: This means $[a\cdot b]\equiv[0]$, so $a\cdot b\equiv 0\bmod 6$
\item Property of Modulo 7: This implies that 7 divides the product $a\cdot b$.
\item Prime Divisibility: Since 7 is a prime number, it must divide either a or b.
\item Back to Equivalence Classes: Therefore, $a \equiv 0 \pmod 7$ (meaning
$[a]=[0]$) or $b \equiv 0 \pmod 7$ (meaning $[b]=[0]$).
\end{prooflist}

\begin{prooflist}{7. Do the following calculations in $\mathbb{Z}_9$, in each case expressing your answer as $[a]$ with $0 \leq a \leq 8$.}
\item
    (a) $[8]+[8] = [7]$
\item
    (b) $[24]+[11] = [8]$
\item
    (c) $[21] \cdot[15] = [0]$
\item
    (d) $[8] \cdot [8] = [1]$
\end{prooflist}

\begin{prooflist}{8. Suppose $[a],[b] \in \mathbb{Z}_n$, and $[a]=\left[a^{\prime}\right]$ and $[b]=\left[b^{\prime}\right]$. Alice adds $[a]$ and $[b]$ as $[a]+[b]=$ $[a+b]$. Bob adds them as $\left[a^{\prime}\right]+\left[b^{\prime}\right]=\left[a^{\prime}+b^{\prime}\right]$. Show that their answers $[a+b]$ and $\left[a^{\prime}+b^{\prime}\right]$ are the same.}
\item
\end{prooflist}

\end{document}
