\documentclass{hippoidC}

\memoto{Idris}
\memosubject{Book of Proof}
\memodate{2024.03.17}
\status{\S 8 Proofs about Sets}

\begin{document}
\toc{}

\begin{prooflist}{1. Prove that $\{12 n: n \in \mathbb{Z}\} \subseteq\{2 n: n \in \mathbb{Z}\} \cap\{3 n: n \in \mathbb{Z}\}$.}
	\item

	\[
		x \in \{12 n: n \in \mathbb{Z}\}
		\implies
		x \in \{2 n: n \in \mathbb{Z}\} \cap\{3 n: n \in \mathbb{Z}\}
	\]

	Suppose $x \in \{12 n: n \in \mathbb{Z}\}$.

	Then $12 \mid x$ and $12c=x$ for some integer $c$.

	Consequently, $2\cdot2\cdot3\cdot c =x$.

	Therefore $2\mid x$ and $3\mid x$, so $x =2d$ and $x=3e$ for some integers
	$d$ and $e$.
	This means that $x \in \{2 n: n \in \mathbb{Z}\}$ and $x \in \{3 n: n \in
		\mathbb{Z}\}$.

	We have shown that
	$x \in \{12 n: n \in \mathbb{Z}\}$
	implies
	$x \in\{2n: n \in \mathbb{Z}\}$
	and
	$x \in\{3n: n \in \mathbb{Z}\}$,
	so it follows that
	$\set{12 n: n \in \mathbb{Z}} \subseteq\set{2 n: n \in \mathbb{Z}} \cap\{3 n: n \in
		\mathbb{Z}\}$.
\end{prooflist}

\begin{prooflist}{2. Prove that $\set{6 n: n \in \mathbb{Z}}=\set{2 n: n \in
				\mathbb{Z}} \cap\set{3 n: n \in \mathbb{Z}}$.}
	\item To show that two sets are equal, we can equivalently prove that the following two subset
	relationships hold.
	\begin{align}
		\{6n: n \in \mathbb{Z}\}
		\subseteq &
		\{2n: n \in \mathbb{Z}\} \cap \{3n: n \in \mathbb{Z}\} \label{eq:line1} \\
		\{2n: n \in \mathbb{Z}\} \cap \{3n: n \in \mathbb{Z}\}
		\subseteq &
		\{6n: n \in \mathbb{Z}\} \label{eq:line2}
	\end{align}
	\item Those two subset relationships are equivalent to the following two
	implications.
	\begin{align}
		x \in \{6n: n \in \mathbb{Z}\}
		\implies &
		x \in \{2n: n \in \mathbb{Z}\} \cap \{3n: n \in \mathbb{Z}\} \label{eq:imp1} \\
		x \in \{2n: n \in \mathbb{Z}\} \cap \{3n: n \in \mathbb{Z}\}
		\implies &
		x \in \{6n: n \in \mathbb{Z}\} \label{eq:imp2}
	\end{align}

	We start with Equation \ref{eq:imp1}.
	\item Supppose $x \in \{6n: n \in \mathbb{Z}\}$. Therefore $6c = x$, for some
	$c\in\mathbb{N}$, which means $6\mid x.$ and also $2\cdot3\mid x$, so
	therefore $2\mid x$ and $3\mid x$.
	\item Because $2d=x$ and $3e=x$, for some $d, e \in \mathbb{N}$,
	it is also true that $\{2n: n \in \mathbb{Z}\} \cap \{3n: n \in \mathbb{Z}\}$.

	Now we will show Equation \ref{eq:imp2}.
	\item Suppose $x \in \set{2n: n \in \mathbb{Z}} \cap \set{3n: n \in
			\mathbb{Z}}$. Therefore $2f=x$ and $3g=x$, for some $f, g \in \mathbb{N}$.
	Because $x$ has as prime factors $2$ and $3$, it also has as a factor any
	product combination of those prime factors, therefore $6\mid x$. Because
	$6\mid x$, we can conclude $x \in \{6n: n \in \mathbb{Z}\}$.
	\item Because we have shown \ref{eq:imp1} and \ref{eq:imp2}, it follows that
	$\set{6 n: n \in \mathbb{Z}}=\set{2 n: n \in \mathbb{Z}} \cap\set{3 n: n \in
			\mathbb{Z}}$..
\end{prooflist}

\begin{prooflist}{3. If $k \in \mathbb{Z}$,}
	\item

	\begin{align*}
		\set{n \in \mathbb{Z}: n \mid k}       & \subseteq \set{n \in \mathbb{Z}: n \mid k^2}      \\
		x \in \set{n \in \mathbb{Z}: n \mid k} & \implies x \in \set{n \in \mathbb{Z}: n \mid k^2}
	\end{align*}
	\item Suppose $x \in \set{n \in \mathbb{Z}: n \mid k}$. Then it follows $\exists
		c\in\mathbb{N}: xc=k$.
	\item Now if we square both sides of
	$xc=k$ we get $x(xc^2) = k^2$.  This shows that $x\mid k^2$ and we have proven
	the implication.
\end{prooflist}

\begin{prooflist}{4. If $m, n \in \mathbb{Z}$, then}
	\item
	\begin{align*}
		\set{x \in \mathbb{Z}: mn \mid x}       & \subseteq \set{x \in \mathbb{Z}: m \mid x} \cap \set{x \in \mathbb{Z}: n \mid x}      \\
		a \in \set{x \in \mathbb{Z}: mn \mid x} & \implies a \in \set{x \in \mathbb{Z}: m \mid x} \cap \set{x \in \mathbb{Z}: n \mid x}
	\end{align*}
	\item Suppose $a \in \set{x \in \mathbb{Z}: mn \mid x}$. Then, $mn \mid a$, implying the existence of $c \in \mathbb{N}$ such that $cmn = a$.
	\item Therefore, we have $m(cn) = a$ and $n(cm) = a$, demonstrating that $a \in \left(\set{x \in \mathbb{Z}: m \mid x} \cap \set{x \in \mathbb{Z}: n \mid x}\right)$.
	\item Thus, $\set{x \in \mathbb{Z}: mn \mid x} \subseteq \left(\set{x \in \mathbb{Z}: m \mid x} \cap \set{x \in \mathbb{Z}: n \mid x}\right)$.
\end{prooflist}

\begin{prooflist}{5. If $p$ and $q$ are positive integers, then $\{p n: n \in
			\mathbb{N}\} \cap\{q n: n \in \mathbb{N}\} \neq \varnothing$.}
	\item Suppose $p, q \in \mathbb{N}$. Therefore $pn \in \mathbb{N}$ and
	$qn \in \mathbb{N}$. Therefore $pq \in \mathbb{N}$ and $pq \in \set{pn} \cap
		\set{qn}$.
	\item Therefore we have shown that
	$\{p n: n \in \mathbb{N}\} \cap\{q n: n \in \mathbb{N}\} \neq \varnothing$.
\end{prooflist}

\begin{prooflist}{5. If $p$ and $q$ are positive integers, then ${p n: n \in \mathbb{N}} \cap{q n: n \in \mathbb{N}} \neq \varnothing$.}
	\item Let $p$ and $q$ be positive integers.
	\item Let $A$ be the set such that $A=\set{pn:n\in\mathbb{N}}$.
	\item Let $B$ be the set such that $B=\set{qn:n\in\mathbb{N}}$.
	\item Let $k$ be an integer such that $k=\text{lcm}(p, q)$. Because $k$ is
	divisible both by $p$ and $q$ by definition, we have shown that the set $A
		\cap B$ is not empty.
	\item Therefore the union of the sets
	${p n: n \in \mathbb{N}} \cap{q n: n \in \mathbb{N}}$ is non-empty.
\end{prooflist}

\begin{prooflist}{6. Suppose $A, B$ and $C$ are sets. Prove that if $A \subseteq B$, then $A-C \subseteq B-C$.}
	\item
	Suppose that $x\in A\subseteq B$ and $x\in A-C$. Therefore $x\in A\land x\in B
		\land x \notin C$. Because $x\in B$ and $x\notin C$, it follows that $x\in B-C$
	which means that $A-C\subseteq B-C$.
\end{prooflist}

\begin{prooflist}{7. Suppose $A, B$ and $C$ are sets. If $B \subseteq C$, then $A \times B \subseteq A \times C$.}
	\item Let $(x, y)$ be an arbitrary element in $A\times B$. Therefore $x\in A$
	and $y\in B$.
	\item Because of $B\subseteq C$, all elemenets in $B$ are also in $C$, so $y\in C$.
	\item Because $x\in A$ and $y\in C$, then $(x, y) \in A \times C$.
	\item Consequently every element $(x, y) \in A\times B$ is also in $A\times C$,
	hence $A \times B \subseteq A \times C$ whenever $B\subseteq C$.
\end{prooflist}

\begin{prooflist}{8. If $A, B$ and $C$ are sets, then $A \cup(B \cap C)=(A \cup B) \cap(A \cup C)$.}
	\item To show an equality of sets, we must prove an implication and its
	converse.
	\begin{align*}
		A \cup(B \cap C)          & \implies(A \cup B) \cap(A \cup C) \\
		(A \cup B) \cap(A \cup C) & \implies A \cup(B \cap C)
	\end{align*}
	\item Suppose $x \in B\cap C$. Therefore $x$ is in $B$ and $C$. Because $x$ is
	in $B\cap C$, it is also in $A \cup (B \cap C)$.
	\item Because $x\in B$, it is also in $A\cup B$.
	\item Because $x\in C$, it is also in $A\cup C$.
	\item Because $x\in (A \cup B) \land x\in(A \cup C)$, it is also in $(A\cup B) \cap
		(A\cup C)$. Therefore we have shown the first implication
	$A \cup(B \cap C)\implies(A \cup B) \cap(A \cup C)$.
	\item
	Now to prove the converse, suppose that $y \in A$. Therefore $y\in A \cup B$
	and $y\in A\cup C$, which means also that $y$ is in $(A\cup B) \cap (A
		\cup C)$. For the union of any set and $A$, $y$ will be an element of that
	union, therefore $y \in A \cup (B\cap C)$.
	\item Thus we have shown the original implication and the converse, thus proving
	the set equality.
\end{prooflist}

\begin{prooflist}{9. If $A, B$ and $C$ are sets, then $A \cap(B \cup C)=(A \cap B) \cup(A \cap C)$.}
	\item
	\begin{align*}
		A \cap(B \cup C)          & \implies(A \cap B) \cup(A \cap C) \\
		(A \cap B) \cup(A \cap C) & \implies A \cap(B \cup C)
	\end{align*}
	Suppose there is some arbitrary $x\in A\cap(B\cup C)$. Then $x$ must be in $A$
	and $B\cup C$. Because $x\in A$ and $B\cup C$, then at least one of $A\cap B$ and $A\cap C$ will be true.
	Therefore $x$ will be in the union $(A\cap B) \cup (A\cap C)$ and we have
	proven the first implication.
	\item For the second implication, suppose $x\in A \land x \in B$. Then $x\in A
		\cap B$ and also $x$ will also be in in $(A\cap B) \cup (A\cap C)$. Because
	$x\in B$, then it will also be in $B\cup C$. And as $x\in A$, it will also
	be in $A \cap (B\cup C)$. Therefore we have proven the first and second
	implications, and also the set equivlance $A \cap(B \cup C)=(A \cap B)
		\cup(A \cap C)$.
\end{prooflist}

\begin{prooflist}{10. If $A$ and $B$ are sets in a universal set $U$, then $\overline{A \cap B}=\bar{A} \cup \bar{B}$.}
	\item
	\begin{align*}
		\overline{A \cap B}  & \implies \bar{A} \cup \bar{B} \\
		\bar{A} \cup \bar{B} & \implies \overline{A \cap B}
	\end{align*}
	\item Suppose $x\in \overline{A\cap B}$. Therefore $\neg(x\in{A\cap B})$. We can
	distribute the not using De Morgan's law to get $x\notin A \lor \notin B$.
	\item Because A and B are sets in a universal set, then $x\in\bar{A} \lor
		x\in\bar{B}$ implies $x\in(\bar{A} \cup \bar{B})$ and we have shown the first implication.
	\item Suppose $x\in \bar{A} \cup \bar{B}$. Therefore $x\in U -(A \cap B)$. The
	universe minus $(A\cap B)$ can be written as $\overline{A\cap B}$,  which
	proves the second implication.
	\item We have proven the implication and its converse, therefore we have proven
	set equality.
\end{prooflist}

\begin{prooflist}{11. If $A$ and $B$ are sets in a universal set $U$, then
		$\overline{A \cup B}=\bar{A} \cap \bar{B}$.}
	\item
	\begin{align*}
		\overline{A \cup B}
		 & \subseteq
		\bar{A} \cap \bar{B} \\
		x\in \overline{A \cup B}
		 & \implies
		x\in \bar{A} \cap \bar{B}
	\end{align*}

	\item For the first implication suppose $x\in\overline{A\cup B}$. Therefore $x \notin A\cup B$. Therefore,
	from DeMorgan's law, we know $x\in\bar{A} \cap \bar{B}$ and we have shown
	the subset relationship of the first implication.

	\begin{align*}
		\bar{A} \cap \bar{B}
		 & \subseteq
		\overline{A \cup B} \\
		x\in\bar{A} \cap \bar{B}
		 & \implies
		x\in\overline{A \cup B}
	\end{align*}
	\item For the second implication suppose $x\in\bar{A}\cap\bar{B}$. Therefore
	\begin{align*}
		x \in U - \neg\left(\bar{A}\cap\bar{B}\right) \\
		x \in U - \left(A\cup B\right)
	\end{align*}
	\item Because $U-C$ for some set C is equal to $\bar{C}$, $U - \left(A\cup
		B\right)$ is equivalent to $\overline{A\cup B}$. Therefore we have shown the
	second implication and the second subset relationship and thus proven the
	set equality of $x \notin A\cup B$.
\end{prooflist}

\begin{prooflist}{12. If $A, B$ and $C$ are sets, then $A-(B \cap C)=(A-B) \cup(A-C)$.}
	\item
	\begin{align*}
		A-(B \cap C)= & \set{x: x\in A \land x\notin(B \cap C)}                              & \text{def. of }-                \\
		=             & \set{x: x\in A \land x\notin B \land x\notin C}                      & \text{def. of $\cap$}           \\
		=             & \set{x: x\in A \land \neg{(x\in B \cap C})}                          &                                 \\
		=             & \set{x: x\in A \land (x\in \bar{B} \cup \bar{C}})                    & \text{Demorgan}                 \\
		=             & \set{x: x\in A \land x\in \bar{B} \lor x\in\bar{C}}                  & \text{def. of }\cup             \\
		=             & \set{x: x\in A \land x\in A \land x\in \bar{B} \lor x\in\bar{C}}     & (x\in A = x\in A \land x \in A) \\
		=             & \set{x: (x\in A \land x\in \bar{B}) \lor (x\in A \land x\in\bar{C})} & \text{rearrange}                \\
		=             & \set{x: (x\in A  - B) \lor (x\in A - C)}                             & \text{def. of }-                \\
		=             & (A - B) \cup (A-C)                                                   & \text{def. of }\cup             \\
	\end{align*}
\end{prooflist}


\begin{prooflist}{13. If $A, B$ and $C$ are sets, then $A-(B \cup C)=(A-B) \cap(A-C)$.}
	\item
	\begin{align*}
		A-(B \cup C)= & \set{x: x\in A \land x\notin(B \cup C)}                               & \text{def. of -}                \\
		=             & \set{x: x\in A \land (x\notin B \lor x\notin C}                       & \text{def. of $\cup$}           \\
		=             & \set{x: x\in A \land \neg{(x\in B \cup C})}                           &                                 \\
		=             & \set{x: x\in A \land (x\in \bar{B} \cap \bar{C}})                     & \text{Demorgan}                 \\
		=             & \set{x: x\in A \land x\in \bar{B} \land x\in\bar{C}}                  & \text{def. of }\cap             \\
		=             & \set{x: x\in A \land x\in A \land x\in \bar{B} \land x\in\bar{C}}     & (x\in A = x\in A \land x \in A) \\
		=             & \set{x: (x\in A \land x\in \bar{B}) \land (x\in A \land x\in\bar{C})} & \text{rearrange}                \\
		=             & \set{x: (x\in A  - B) \land (x\in A - C)}                             & \text{def. of }-                \\
		=             & (A - B) \cap (A-C)                                                    & \text{def. of }\cap
	\end{align*}
\end{prooflist}

\begin{prooflist}{14. If $A, B$ and $C$ are sets, then $(A \cup B)-C=(A-C) \cup(B-C)$.}
	\item
	\begin{align*}
		(A\cup B) - C = & \set{x: x\in (A\cup B) \land x\notin C}                         & \text{def. of -}                         \\
		=               & \set{x: x\in A \lor x \in B \lor x\notin C}                     & \text{def. of $\cup$}                    \\
		=               & \set{x: x\in A \land x\notin C \lor x\in B \land x\notin C}     & (x\notin C = x\notin C \land x \notin C) \\
		=               & \set{x: (x\in A \land x\notin C) \lor (x\in B \land x\notin C)} & \text{rearrange}                         \\
		=               & \set{x: (x\in A  - C) \lor (x\in B - C)}                        & \text{def. of -}                         \\
		=               & (A - B) \cup (B-C)                                              & \text{def. of }\cup
	\end{align*}
\end{prooflist}

\begin{prooflist}{15. If $A, B$ and $C$ are sets, then $(A \cap B)-C=(A-C) \cap(B-C)$.}
	\item
	\begin{align*}
		(A\cap B) - C = & \set{x: x\in (A\cap B) \land x\notin C}                          & \text{def. of -}                         \\
		=               & \set{x: x\in A \land x \in B \land x\notin C}                    & \text{def. of $\cup$}                    \\
		=               & \set{x: x\in A \land x\notin C \land x\in B \land x\notin C}     & (x\notin C = x\notin C \land x \notin C) \\
		=               & \set{x: (x\in A \land x\notin C) \land (x\in B \land x\notin C)} & \text{rearrange}                         \\
		=               & \set{x: (x\in A  - C) \land (x\in B - C)}                        & \text{def. of -}                         \\
		=               & (A - B) \cap (B-C)                                               & \text{def. of }\cap
	\end{align*}
\end{prooflist}

\begin{prooflist}{16. If $A, B$ and $C$ are sets, then $A \times(B \cup C)=(A \times B) \cup(A \times C)$.}
	\item
	\begin{align*}
		A\times (B\cup C) = & \set{(x, y): x\in A \land y\in B\cup C}                        & \text{def. of }\times \\
		=                   & \set{(x,y): x\in A \land (y \in B \lor y\in C)}                & \text{def. of $\cup$} \\
		=                   & \set{(x,y): (x\in A \land y \in B) \lor (x\in A \land y\in C)} &
		\text{distributive law}                                                                                      \\
		=                   & (A\times B) \cup (A\times C)
	\end{align*}
\end{prooflist}

\begin{prooflist}{17. If $A, B$ and $C$ are sets, then $A \times(B \cap C)=(A \times B) \cap(A \times C)$.}
	\item
	\begin{align*}
		A\times (B\cap C) = & \set{(x, y): x\in A \land y\in B\cap C}                         & \text{def. of }\times \\
		=                   & \set{(x,y): x\in A \land (y \in B \land y\in C)}                & \text{def. of $\cap$} \\
		=                   & \set{(x,y): (x\in A \land y \in B) \land (x\in A \land y\in C)} &
		\text{distributive law}                                                                                       \\
		=                   & (A\times B) \cap (A\times C)
	\end{align*}
\end{prooflist}

\begin{prooflist}{18. If $A, B$ and $C$ are sets, then $A \times(B-C)=(A \times B)-(A \times C)$.}
	\item
	\begin{align*}
		A\times (B-C) = & \set{(x, y): x\in A \land y\in (B-C)}                             & \text{def. of }\times \\
		=               & \set{(x,y): x\in A \land y \in B \land y\notin C)}                & \text{def. of }-      \\
		=               & \set{(x,y): x\in A \land y \in B \land x\in A \land y\notin C)}   & \text{identity law}   \\
		=               & \set{(x,y): (x, y) \in A\times B \land (x, y) \notin A \times C)} &
		\text{def. of }\times                                                                                       \\
		=               & (A\times B) - (A\times C)
	\end{align*}
\end{prooflist}

\begin{prooflist}{19. Prove that}
	\item
	\begin{align*}
		\set{9^n: n \in \mathbb{Z}} \subseteq\set{3^n: n \in \mathbb{Z}}          \\
		x\in \set{9^n: n \in \mathbb{Z}} \implies x\in\set{3^n: n \in \mathbb{Z}} \\
		\set{9^n: n \in \mathbb{Z}} \neq\set{3^n: n \in \mathbb{Z}}               \\
		x\in\set{3^n: n \in \mathbb{Z}}
		\not\implies
		x\in \set{9^n: n \in \mathbb{Z}}
	\end{align*}
	\item Suppose some $x\in \set{9^n: n\in\mathbb{Z}}$. Therefore $x\in\set{3^{2n}:
			n\in\mathbb{Z}}$, and therefore $x\in\set{3^n: n\in\mathbb{Z}}$. Therefore
	we have proven that $\set{9^n: n \in \mathbb{Z}} \subseteq\set{3^n: n \in \mathbb{Z}}$.
	\item To show that $\set{9^n: n \in \mathbb{Z}} \neq\set{3^n: n \in
			\mathbb{Z}}$, we must show that the converse implication does not hold. We
	can prove this with one counterexample. $3^1$ is in $\set{3^n}$ and not in
	$\set{9^n:n\in\mathbb{Z}}$. Therefore we have shown
	$\set{9^n: n \in \mathbb{Z}} \neq\set{3^n: n \in \mathbb{Z}}$.
\end{prooflist}

\begin{prooflist}{20. Prove that $\set{9^n: n \in \mathbb{Q}}=\set{3^n: n \in
				\mathbb{Q}}$.}
	\item
	\begin{align*}
		\set{9^n: n \in \mathbb{Q}}      & \subseteq\set{3^n: n \in \mathbb{Q}}     \\
		x\in \set{9^n: n \in \mathbb{Q}} & \implies x\in\set{3^n: n \in \mathbb{Q}} \\
		\set{3^n: n \in \mathbb{Q}}
		                                 & \subseteq
		\set{9^n: n \in \mathbb{Q}}                                                 \\
		x\in\set{3^n: n \in \mathbb{Q}}
		                                 & \implies
		x\in \set{9^n: n \in \mathbb{Q}}
	\end{align*}
	\item Let $x$ be some arbitrary element in $\set{9^n: n \in \mathbb{Q}}$.
	Because $n\in\mathbb{Q}$, $n=\frac{p}{q}: p,q\in\mathbb{Z}$,
	we can rewrite $9^n$ as $3^{2n}$, or $3^{\frac{2p}{q}}$. Because
	$\frac{2p}{q} \in \mathbb{Q}$, we have proven the first subset
	relationship $\set{9^n: n \in \mathbb{Q}} \subseteq\set{3^n: n \in
			\mathbb{Q}}$.
	\item Let $x$ be some arbitrary element in $\set{3^n: n \in \mathbb{Q}}$.
	Because $n\in\mathbb{Q}$, $n=\frac{p}{q}: p,q\in\mathbb{Z}$,
	we can rewrite $3^n$ as $9^{\frac{n}{2}}$, or $9^{\frac{p}{2q}}$. Because
	$\frac{p}{2q} \in \mathbb{Q}$, we have proven the second subset
	relationship $\set{9^n: n \in \mathbb{Q}} \subseteq\set{3^n: n \in
			\mathbb{Q}}$, and therefore we have proven the set equality.
\end{prooflist}.

\begin{prooflist}{21. Suppose $A$ and $B$ are sets. Prove $A \subseteq B$ if and only if $A-B=\varnothing$.}
	\item
	\begin{align*}
		P & = A \subseteq B       \\
		Q & = A - B = \varnothing
	\end{align*}
	To prove the if-and-only-if statement, we must prove the implication $P\implies
		Q$ and its converse $Q\implies P$.
	\item Suppose $x\in A\subseteq B$. Therefore $x\in A \land x \in B$. If $x\in
		A\land x\in B$, then by definition $A-B=\varnothing$, and we have shown the
	first implication.
	\item Suppose $x\in A-B\neq\varnothing$. Therefore there is some $x\in A\land
		x\notin B$, therefore there is some element in A that is not in B, showing
	$A\not\subseteq B$. Thus we have proven the second implication by the
	contrapositive and have proven $A \subseteq B$ if and only if
	$A-B=\varnothing$.
\end{prooflist}

\begin{prooflist}{22. Let $A$ and $B$ be sets. Prove that $A \subseteq B$ if and only if $A \cap B=A$.}
	\item
	\begin{align*}
		P & = A \subseteq B \\
		Q & = A \cap B = A
	\end{align*}
	To prove the if-and-only-if statement, we must prove the implication $P\implies
		Q$ and its converse $Q\implies P$.
	\item Suppose $x\in A\subseteq B$. Therefore $x\in A \land x\in B$. Because all
	elements of $A$ are in $B$, doing $A\cap B$ will result in only the
	elements that are in $A$, therefore $A\cap B=A$.
	\item Suppose $x\in A\cap B$.  This means $x\in A \land x\in B$, and by
	definition, this means $A\subseteq B$. Therefore we have proven the second
	implication and the statement---$A \subseteq B$ if and only if $A \cap B=A$.
\end{prooflist}

\begin{prooflist}{23. For each $a \in \mathbb{R}$, let $A_a=\set{(x,
				a(x^2-1)) \in \mathbb{R}^2: x \in \mathbb{R}}$.
		Prove that $\bigcap_{a \in \mathbb{R}} A_a=\set{(-1,0),(1,0)}$.}
	\item
	Let $A_a=\set{(x, a(x^2-1)) \in \mathbb{R}^2: x \in \mathbb{R}}$.
	\item Prove that $\bigcap_{a \in \mathbb{R}} A_a=\set{(-1,0),(1,0)}$.
\end{prooflist}

\begin{prooflist}{24. Prove that $\bigcap_{x \in \mathbb{R}}\left[3-x^2, 5+x^2\right]=[3,5]$.}
	\item
	\begin{align*}
		\bigcap_{x\in\mathbb{R}}\left[3-x^2, 5+x^2\right] & \subseteq [3,5]                                               \\
		[3,5]                                             & \subseteq \bigcap_{x \in \mathbb{R}}\left[3-x^2, 5+x^2\right] \\
	\end{align*}
	\item
	Suppose $a\in\mathbb{R}$. Because $a^2$ is a always a positive number $3-a^2\leq3$
	and $5+a^2\geq 5$, therefore $[3-a^2, 5+a^2]$ will always contain the range $[3,
				5]$. Thus we have shown
	$\bigcap_{x\in\mathbb{R}}\left[3-x^2, 5+x^2\right]\subseteq [3,5]$.
	\item Suppose $a \notin [3-x^2, 5+x^2]$ and $x\in \mathbb{R}$. Because $x^2$ is
	always positive, then $3-x^2\leq3$ and $5+x^2\geq 5$, which means the range
	is larger than $[3, 5]$. Therefore $a \notin [3, 5]$, and we have proven the
	second implication via the contrapositive.
\end{prooflist}

\begin{prooflist}{24a. Prove that $\bigcap_{x \in \mathbb{R}}\left[3-x^2, 5+x^2\right]=[3,5]$.}
	\item
	To prove that $\bigcap_{x \in \mathbb{R}}\left[3-x^2, 5+x^2\right]=[3,5]$, we need to show that $[3,5]$ is a subset of $\bigcap_{x \in \mathbb{R}}\left[3-x^2, 5+x^2\right]$.

	\item Suppose $y \in [3,5]$. This means $3 \leq y \leq 5$.

	\item For any $x \in \mathbb{R}$, we have $3 - x^2 \leq 3$ and $5 + x^2 \geq 5$.

	\item Therefore, $3 \leq 3 - x^2 \leq y \leq 5 \leq 5 + x^2$, which implies $y \in \left[3-x^2, 5+x^2\right]$ for all $x \in \mathbb{R}$.

	\item Hence, $y$ belongs to the intersection of all such intervals, $\bigcap_{x \in \mathbb{R}}\left[3-x^2, 5+x^2\right]$.

	\item Therefore, $[3,5] \subseteq \bigcap_{x \in \mathbb{R}}\left[3-x^2, 5+x^2\right]$.

\end{prooflist}

This proof focuses on demonstrating that $[3,5]$ is a subset of $\bigcap_{x \in \mathbb{R}}\left[3-x^2, 5+x^2\right]$, which is sufficient to establish the desired set equality.

\begin{prooflist}{25. Suppose $A, B, C$ and $D$ are sets. Prove that $(A \times
			B) \cup(C \times D) \subseteq(A \cup C) \times(B \cup D)$.} \item
	\begin{align*}
		(A\times B)\cup (C\times D)= & \set{x, y: (x\in A \land y\in B) \lor (x\in C
		\land y\in D) }              & \text{def. of }\times,\cup                                                \\
		=                            & \set{x,y: (x\in A \lor x\in C) \land (y\in B \lor y\in D) } & \text{dist.
		property}                                                                                                \\
		=                            & (A\cup C) \cap (B\cup D)
	\end{align*}
\end{prooflist}

\begin{prooflist}{26. Prove that $\{4 k+5: k \in \mathbb{Z}\}=\{4 k+1: k \in \mathbb{Z}\}$.}
	\item
	Let's move on...
\end{prooflist}

% \begin{prooflist}{27. Prove that $\{12 a+4 b: a, b \in \mathbb{Z}\}=\{4 c: c \in \mathbb{Z}\}$.}
% \item
% \end{prooflist}

% \begin{prooflist}{28. Prove that $\{12 a+25 b: a, b \in \mathbb{Z}\}=\mathbb{Z}$.}
% \item
% \end{prooflist}

% \begin{prooflist}{29. Suppose $A \neq \varnothing$. Prove that $A \times B \subseteq A \times C$ if and only if $B \subseteq C$.}
% \item
% \end{prooflist}

% \begin{prooflist}{30. Prove that $(\mathbb{Z} \times \mathbb{N}) \cap(\mathbb{N} \times \mathbb{Z})=\mathbb{N} \times \mathbb{N}$.}
% \item
% \end{prooflist}

% \begin{prooflist}{31. Suppose $B \neq \varnothing$ and $A \times B \subseteq B \times C$. Prove that $A \subseteq C$.}
% \item
% \end{prooflist}

\end{document}
