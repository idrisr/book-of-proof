\documentclass{hippoidC}

\memoto{Idris}
\memosubject{Book of Proof}
\memodate{2024.03.12}
\status{\S 3.7 Inclusion Exclusion}

\begin{document}

\toc
\pagebreak

\begin{prooflist}{1. At a certain university 523 of the seniors are history
    majors or math majors (or both). There are 100 senior math majors, and 33
seniors are majoring in both history and math. How many seniors are majoring in
history?}
    \item Let $A=$ math majors, $|A|=100$
    \item Let $B=$ history majors
    \item $A\cap B=$ both history and math major, $|A\cap B|=33$
    \item $A \cup B$ =math or history majors or both, $|A\cup B|=|A| +
        |B| - |A\cap B|=523$
    \item
    $$|A\cup B|=|A| + |B| - |A\cap B|=523$$
    $$|B| = 523 + |A\cap B| - |A|$$
    $$|B| = 523 + 33 - 100$$
    $$|B| = 456$$
\end{prooflist}

\begin{prooflist}{2. How many 4-digit positive integers are there for which
    there are no repeated digits, or for which there may be repeated digits, but
all digits are odd?}
    \item Let $A=$ 4-digit positive integers
    \item Let $B=$ 4-digit positive integers with no repeated digits, $B \subset
        A$
    \item Let $C=$ 4-digit positive integers with repeated digits and all digits
        are odd, $C \subset A$
    \item $C\cap B$  = 4-digit positive integers for which there are
        no repeated digits and all digits are odd.
    \item $C\cup B$  = 4-digit positive integers for which there are
        no repeated digits, or for which there may be repeated digits, but all
        digits are odd.
    \item
        $$|C\cup B| = |C| + |B| - |C\cap B|$$
    \item For B, we can pick any number 1..9 to start, and then for each
        subsequent digit we can pick between 0..9, excluding the one that came
        before, therefore there are also 9 choices for each non-starting digit.
        $|B| = 9^4$.
    \item For C, we can pick any number 1,3,5,7,9 for all 4 positions, therefore
        $|C|=5^4$.
    \item For $|C\cap B|$, we can pick any number 1,3,5,7,9 for the first
        position, and thereafter we can pick from any of the odd digits that did
        not directly precede the current one, therefore there are $|C \cap
        B|=5\cdot4^3$.
    \item $$|C\cup B| = |C| + |B| - |C\cap B|$$
    \item $$|C\cup B| = 5^4+9^4-5\cdot 4^3$$
\end{prooflist}

\begin{prooflist}{3. How many 4-digit positive integers are there that are even or contain no 0’s?}
    \item Let $A=$ 4-digit positive integers that are even
    \item Let $B=$ 4-digit positive integers that contain no zeros
    \item $A \cap B=$ 4-digit positive integers that are even and contain no zeros
    \item $A \cup B=$ 4-digit positive integers that are even or contain no zeros
    \item To create elements in A, we can pick 1..9 to start, then 0..9 for the
        second and third digits, and any of the 5 even digits for the last
        digit, therefore $|A|=9\cdot 10^2\cdot5$.
    \item To create elements in B, we can pick from 1..9 for all digits,
        therefore $|B|=9^4$.
    \item To create elements in $A\cap B$, we can pick from 1..9 for the first
        3 digits and 2,4,6,8 for the last digit, therefore $|A\cap B|=9^3\cdot4$
    \item $$|A\cup B| = |A| + |B| - |A\cap B|$$
    \item $$|A\cup B| = 9\cdot10^2\cdot5 + 9^4 - 9^3\cdot4$$
\end{prooflist}

\begin{prooflist}{4. This problem involves lists made from the letters T, H, E, O, R, Y, with repetition allowed.}
\item (a) How many 4-letter lists are there that don’t begin with T, or don’t
    end in Y?
        \item Let $A=$ 4-letter lists that dont begin with T
        \item Let $B=$ 4-letter lists that dont end with Y
        \item Let $A\cap B=$ 4-letter lists that dont begin with T and dont end with Y
        \item Let $A\cup B=$ 4-letter lists that dont begin with T or dont end with Y
        \item $|A|= 5\cdot6\cdot6\cdot6$
        \item $|B|= 6\cdot6\cdot6\cdot5$
        \item $|A\cap B|= 5\cdot6\cdot6\cdot5$
        \item $|A\cup B|= |A| + |B| - |A\cap B| = 2\cdot 6^3\cdot5 - 5^2\cdot6^2$
\item (b) How many 4-letter lists are there in which the sequence of letters T,
    H, E appears consecutively (in that order)?
\item Other than THE, there is one letter-spot left free, and there are 6 letters
    available for that spot. THE can be placed starting at either position 1 or
    position 2 for a total of two choices. Therefore by the multiplication
    principle there are $2\cdot6$ 4-letter lists matching the problem criteria.
\item (c) How many 6-letter lists are there in which the sequence of letters T,
    H, E appears consecutively (in that order)?
\item Other than THE, there are three letter-spots left free, and there are 6 letters
    available for those 3 spots. THE can be placed starting at either position
    1, 2, 3, or 4 for a total of four choices. Therefore by the multiplication
    principle there are $4\cdot6^3$ 6-letter lists matching the problem criteria.
\end{prooflist}

\begin{prooflist}{5. How many 7-digit binary strings begin in 1 or end in 1 or have exactly four 1’s?}
    \item Let $A=$ 7-digit binary strings that begin in 1
    \item Let $B=$ 7-digit binary strings that end in 1
    \item Let $C=$ 7-digit binary strings that have exactly four 1s
    \item Let $D=$ 7-digit binary strings
    \item $A\cup B \cup C=$ 7-digit binary strings begin in 1 or end in 1 or have exactly four 1’s
    \item $\neg(A\cup B \cup C)=\neg A \cap \neg B \cap \neg C=$
        7-digit binary strings dont begin in 1 and dont end in 1 and dont have
        exactly four 1’s.
    \item To create elements of $\neg A \cap \neg B \cap \neg C$,
        the first and last digit must be 0. For the middle 5 spots, there must
        be either 0, 1, 2, 3, or 5 1s. This means there are
        $\binom{5}{0}+\binom{5}{1}+\binom{5}{2}+\binom{5}{3}+\binom{5}{5}$ ways
        to arrange the middle 5 spots.
    \item We can use the fact that $A\cup B \cup C$ is the complement of
        $\neg(A\cup B \cup C)$ to determine that
        $$|D| - |\neg(A\cup B \cup C)|=| A\cup B \cup C |$$
        $$ | A\cup B \cup C | = 2^7 - \binom{5}{0}-\binom{5}{1}-\binom{5}{2}-\binom{5}{3}-\binom{5}{5}$$
\item
\end{prooflist}

\begin{prooflist}{6. Is the following statement true or false? Explain. If
        $A_1
    \cap A_2 \cap A_3 =\emptyset$ , then $|A_1 \cup A_2 \cup A_3| =
|A_1|+|A_2|+|A_3|$.}
\item
$$
A_1 \cap A_2 \cap A_3 =\emptyset
\implies
|A_1 \cup A_2 \cup A_3| = |A_1|+|A_2|+|A_3|
$$
\item Suppose $A_1 \cap A_2 \cap A_3 =\emptyset$.
\item Let
    $A_1=\{1, 2\},\quad
    A_2=\{1, 3\},\quad
    A_3=\{4, 5\} $
\item This is a counter-example as $|A_1 \cup A_2 \cup A_3|=5$ while
    $|A_1|+|A_2|+|A_3| = 2 +2 +2=6$.
\item For the implication to hold, we must also know that intersection between
    any two of the sets is also the empty-set.
\end{prooflist}

\begin{prooflist}{7. Consider 4-card hands dealt off of a standard 52-card deck.
    How many hands are there for which all 4 cards are of the same suit or all 4
    cards are red?}
\item Let $A=$ all 4 cards are the same suit
\item Let $B=$ all 4 cards are red
\item $A\cap B=$ all 4 cards are same suit AND all 4 cards are red
\item $A\cup B=$ all 4 cards are same suit OR all 4 cards are red
\item
    $$
        |A \cup B | = |A| + |B| - |A \cap B|
    $$
\item $|A| = 4\cdot\dbinom{13}{4}$
\item $|B| = \dbinom{26}{4}$
\item $|A\cap B| = 2\cdot\dbinom{13}{4}$
\item
    $$
    |A \cup B | = 4\cdot\dbinom{13}{4} + \dbinom{26}{4} - 2\cdot\dbinom{13}{4}
    $$
\end{prooflist}

\begin{prooflist}{8. Consider 4-card hands dealt off of a standard 52-card deck.
    How many hands are there for which all 4 cards are of different suits or all
    4 cards are red?}
\item Let $A=$ all 4 cards are of different suit
\item Let $B=$ all 4 cards are red
\item $A\cap B=\emptyset$ all 4 cards are different suit and all 4 cards are red
\item $|A| = \dfrac{13^4}{4!}$
\item $|B| = \dbinom{26}{4}$
\item
    $$
        |A \cup B | = |A| + |B| - |A \cap B|
        |A \cup B | = \dfrac{13^4}{4!} + \dbinom{26}{4} - 0
    $$
\end{prooflist}

\begin{prooflist}{9. A 4-letter list is made from the letters L, I, S, T, E, D
    according to the following rule: Repetition is allowed, and the first two
letters on the list are vowels or the list ends in D. How many such lists are
possible?}
\item Let $A=$ first two letters on the list are vowels
\item Let $B=$ the list ends in D
\item $A\cap B=$ first two letters on the list are vowels and the list ends in D
\item $A\cup B=$ first two letters on the list are vowels or the list ends in D
\item To determine $|A|$, we note that there are $2^2$ to pick the first two
    letters as vowels, and then $6^4$ possibilities for the next four letters.
\item To determine $|B|$, the last letter is fixed, and there $6^5$
    possibilities for the first 5 letters.
\item To determine $|A\cap B|$ there are $2^2$ possibilities for the first two
    letters as vowel, the last letter is fixed, and the middle 3 letters have
    $6^3$ possible arrangements.
\item Therefore
        $$|A \cup B | = |A| + |B| - |A \cap B|$$
        $$|A \cup B | = 2^2\cdot6^4 + 6^5 - 2^2\cdot6^3$$
\end{prooflist}

\begin{prooflist}{10. How many 6-digit numbers are even or are divisible by 5?}
\item Let $A=$ 6-digit numbers that are even
\item Let $B=$ 6-digit numbers that are divisible by 5
\item $A\cap B=$ 6-digit numbers that are even AND divisible by 5
\item $A\cup B=$ 6-digit numbers that are even OR divisible by 5
\item $|A|=9\cdot10^4\cdot5$
\item $|B|=9\cdot10^4\cdot2$
\item $|A\cap B|= 9\cdot10^4\cdot1$
\item $$|A\cup B|= 9\cdot10^4\cdot5 + 9\cdot10^4\cdot2 - 9\cdot10^4\cdot1$$
\item $$= (9\cdot10^4)(5 + 2 -1)$$
\item $$= 54\cdot10^4$$
\end{prooflist}

\begin{prooflist}{11. How many 7-digit numbers are even or have exactly three digits equal to 0?}
\item Let $A=$ 7-digit numbers that are even
\item Let $B=$ 7-digit numbers that have exactly three digits equal to 0
\item $A\cap B=$ 7-digit numbers that are even AND have exactly three digits equal to 0
\item $A\cup B=$ 7-digit numbers that are even OR have exactly three digits equal to 0
\item $|A|=9\cdot10^6\cdot5$
\item To calculate $|B|$, we note that there must be three 0s, and because we
    can't start the number with 0, there are $\dbinom{6}{3}$ possible choices of
    where to place them. For the remaining 4 digits, there are
    9 choices for the first digit and $10^3$ for the remaining choices. Then by the multiplication principle, there are
    $|B|=9\cdot10^3\cdot \dbinom{6}{3}$.
\item To calculate $|A\cap B|$, we consider two different cases. In the first
    case, the number ends with 0, therefore there are two zeros to place in 5
    spots. The first digit has 9 choices, the non-zeroes have $10^3$ choices,
    and so $9\cdot\dbinom{5}{2}\cdot10^3$. possibilties for this case. For the
    second case that does not end with a zero, there are 9 choices for the
    first digit, $\dbinom{5}{3}$ for the zeros, and 4 choices for the last
    digit, and $10^2$ for the remaining digits. Therefore there are
    $9\cdot\dbinom{5}{3}\cdot10^2\cdot4$ for the second case. The cases sum together for
    $9\cdot\dbinom{5}{3}\cdot10^2\cdot4 +
    9\cdot\dbinom{5}{2}\cdot10^3$ total choices from $|A\cap B|$.
\item Therefore
        $$|A \cup B | = |A| + |B| - |A \cap B|$$
        $$|A \cup B | =
9\cdot10^6\cdot5 + 9\cdot10^3\cdot \dbinom{6}{3} - 9\cdot\dbinom{5}{3}\cdot10^2\cdot4 - 9\cdot\dbinom{5}{2}\cdot10^3$$
\end{prooflist}

\begin{prooflist}{12. How many 5-digit numbers are there in which three of the
    digits are 7, or two of the digits are 2?}
\item Let $A=$ 5-digit numbers in which at least three of the digits are 7
\item Let $B=$ 5-digit numbers in which at least two of the digits are 2
\item $A\cap B= $ 5-digit numbers in which two of the digits are 2 AND three are 7
\item $A\cup B= $ 5-digit numbers in which two of the digits are 2 OR three are 7

\begin{itemize}
    \item 5 7s = If there are 5 7's, there is only way to create that digit. $1$
    \item 4 7s = If there are 4 7's, there are 8 ways to create that digit if
        the number does not start with 7, and $\dbinom{4}{3}\cdot9$ ways if it
        does start with 7.
    \item 3 7s = If there are 4 7's, there are 8 ways to create that digit if
        the number does not start with 7, and $\dbinom{4}{3}\cdot9$ ways if it
        does start with 7.
\end{itemize}

\item $|A|$, case 1, where it begins with 7. Then the remaining 7s can go in
    $\dbinom{4}{2}$ spots, and the remaining two digits have $9^2$ choices.
\item $|A|$, case 1, where it does not begin with 7. There are 8 choices for the
    first digit (exclude 0 and 7), $\dbinom{4}{3}$ spots for the 7s, and 9 spots
    for the remaining digit.

\item didnt finish. These problems are becoming a grind ...
\end{prooflist}
% 13. How many 8-digit binary strings end in 1 or have exactly four 1’s?
% 14. How many 3-card hands (from a standard 52-card deck) have the property that
% it is not the case that all cards are black or all cards are of the same suit?
% 15. How many 10-digit binary strings begin in 1 or end in 1?

\end{document}
