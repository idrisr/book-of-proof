\documentclass[openany, 11pt]{book}
\makeindex

\usepackage{amsmath}
\usepackage{amssymb}
\usepackage{booktabs}
\usepackage{csvsimple-l3}
\usepackage{bussproofs}
\usepackage{dirtytalk}
\usepackage[dvipsnames]{xcolor}
\usepackage{enumitem}
\usepackage{epigraph}
\usepackage{forest}
\usepackage{formal-grammar}
\usepackage{graphicx}
\usepackage[citecolor=blue,colorlinks=true, linkcolor=blue, urlcolor=blue]{hyperref}
\usepackage{kantlipsum}
\usepackage{makeidx}
\usepackage[margin=0.8in]{geometry}
\usepackage{mathrsfs}
\usepackage[outputdir=../build]{minted}
\usepackage{multicol}
\usepackage[mode=tex]{standalone}
\usepackage[style=authortitle]{biblatex}
\usepackage[T1]{fontenc}
\usepackage[tableaux]{prooftrees}
\usepackage{tcolorbox}
\usepackage{tikz}
\usepackage{titlesec}
\usepackage{xcolor}

\usetikzlibrary{arrows}
\usetikzlibrary{arrows.meta}
\usetikzlibrary{automata}
\usetikzlibrary{calc}
\usetikzlibrary{fit}
\usetikzlibrary{petri}
\usetikzlibrary{positioning}

\tcbuselibrary{breakable}
\tcbuselibrary{listings}
\tcbuselibrary{minted}
\tcbuselibrary{skins}
\tcbuselibrary{theorems}

\newcounter{filePrg}

\addbibresource{biblio.bib}
\setlength{\parindent}{0pt}

\renewcommand{\emph}[1]{\textit{#1}}
\setlength{\parindent}{0pt}

\newcommand\setboxcounter[2]{\setcounter{tcb@cnt@#1}{#2}}
\newcommand\qed[0]{\blacksquare}
\setlength{\parindent}{10pt}
\newcommand{\set}[1]{\{#1\}}
\newcommand{\up}[1]{ \left\lceil#1\right\rceil }
\newcommand{\down}[1]{ \left\lfloor#1\right\rfloor}

\definecolor{CaribbeanBlue}{RGB}{0, 206, 209} % Define Caribbean Blue
\NewTcbTheorem[list inside=definition]{definition}
{Definition}{
	breakable,
	colback=CaribbeanBlue!05,
	colframe=CaribbeanBlue!35!black,
	fonttitle=\bfseries}{th}

\NewTcbTheorem[list inside=intuition]{intuition}{Intuition}{
	breakable,
	colback=blue!5,
	colframe=blue!35!black,
	fonttitle=\bfseries}{th}

\NewTcbTheorem{example}{Example}{
	breakable,
	colback=white,
	colframe=green!35!black,
	fonttitle=\bfseries}{th}

\NewTcbTheorem{verify}{Verify}{
	breakable,
	float,
	colback=red!5,
	colframe=red!35!black,
	fonttitle=\bfseries}{th}

\NewTcbTheorem[list inside=theorem]{theorem}{Theorem}{
	breakable,
	colback=gray!10,
	colframe=gray!35!black,
	fonttitle=\bfseries}{th}

% \NewTcbTheorem[
% list inside=exercise,
% number within=chapter,
% number within=section,
% ]

\NewTcbTheorem[
	list inside=exercise,
	number within=section,
]{exercise}{Exercise}{
	breakable,
	colback=white,
	colframe=black,
	fonttitle=\bfseries}{th}

\newcommand{\hask}[1]{\mintinline{haskell}{#1}}

\newenvironment{alist}
{\begin{enumerate}[label={*}, leftmargin=*, itemsep=0pt, parsep=0pt]}
		{\end{enumerate}}

\newenvironment{blist}
{\begin{enumerate}[label={}, leftmargin=*, itemsep=0pt, parsep=0pt]}
		{\end{enumerate}}

\renewcommand{\thesection}{\arabic{section}}
\tcbset{enhanced jigsaw}

\newtcbinputlisting{\codeFromFile}[2]{
	listing file={#1},
	listing engine=minted,
	minted style=colorful,
	minted language=haskell,
	minted options={breaklines,linenos,numbersep=3mm},
	colback=blue!5!white,colframe=blue!75!black,listing only,
	left=5mm,enhanced,
	title={#2},
	overlay={\begin{tcbclipinterior}\fill[red!20!blue!20!white] (frame.south west)
				rectangle ([xshift=5mm]frame.north west);\end{tcbclipinterior}}
}

\newtcblisting{haskell}[1]
{
	listing engine=minted,
	minted style=colorful,
	minted language=haskell,
	minted options={breaklines,linenos,numbersep=3mm},
	colback=blue!5!white,colframe=blue!75!black,listing only,
	left=5mm,enhanced,
	title={#1},
	overlay={\begin{tcbclipinterior}\fill[red!20!blue!20!white] (frame.south west)
				rectangle ([xshift=5mm]frame.north west);\end{tcbclipinterior}}
}

\title{Boof Proof}
\author{Idris}
\date{March 2024}


\begin{document}
\maketitle{}
\tableofcontents
% \tcblistof[\section]{definition}{List of Definitions}
% \listoffigures
% \listoftables

\part{Fundamentals}
% \chapter{Sets}
% \chapter{Logic}
\setcounter{chapter}{2}
\chapter{Counting}
\setcounter{section}{1}
\section{Multiplication Principle}
\begin{exercise}{}{}
	Consider lists made from the letters T, H, E, O, R, Y,
	with repetition allowed.
	\begin{enumerate}[label = {(\arabic*)}]
		\item How many length-4 lists are there? \\
		      $6^4$
		\item How many length-4 lists are there that begin with T?\\
		      $1 \cdot 6^3$
		\item How many length-4 lists are there that do not begin with T?\\
		      $5 \cdot 6^3$
	\end{enumerate}
\end{exercise}

\begin{exercise}{}{}
	Airports are identified with 3-letter codes. For example,
	Richmond, Virginia has the code RIC, and Memphis, Tennessee has MEM. How
	many different 3-letter codes are possible?
	\begin{enumerate}[label = {(\arabic*)}]
		\item $26^3$
	\end{enumerate}
\end{exercise}

\begin{exercise}{}{}
	How many lists of length 3 can be made from the symbols A,
	B, C, D, E, F if\ldots
	\begin{enumerate}[label = {(\arabic*)}]
		\item repetition is allowed.
		      $ 6^3$
		\item repetition is not allowed.
		      $ 6\cdot 5 \cdot 4$
		\item repetition is not allowed and the list must contain the letter A.
		      $ 6\cdot 5 \cdot 3$
		\item repetition is allowed and the list must contain the letter A.
		\item Let $A=$ list of all length-three strings.
		      Let $B=$ list of all length-three strings without A.
		      Then $|A| - |B| = 6^3 - 5^3$
	\end{enumerate}
\end{exercise}

\begin{exercise}{}{}
	In ordering coffee you have a choice of regular or decaf; small, medium or
	large; here or to go. How many different ways are there to order a coffee?
	\begin{enumerate}[label = {(\arabic*)}]
		\item Let $A = \{\text{regular}, \text{decaf}\}$.
		      Let $B = \{\text{small}, \text{medium}, \text{large}\}$.
		      Let $C = \{\text{here}, \text{to-go}\}$.
		      Different ways to order is the product $|A|\cdot|B|\cdot|C| = 2 \cdot 3 \cdot 2$.
	\end{enumerate}
\end{exercise}

\begin{exercise}{}{}
	This problem involves 8-digit binary strings such as 10011011 or 00001010
	(i.e., 8-digit numbers composed of 0's and 1's).
	\begin{enumerate}[label = {(\alph*)}]
		\item How many such strings are there?
		      There are $2^8$ different 8-digit binary strings.
		\item How many such strings end in 0?
		      There are $2^7$ different 8-digit binary strings that end in $0$.
		\item How many such strings have 1's for their second and fourth digits?
		      There are $2^6$ different 8-digit binary strings that have two of
		      their bits constant.
		\item How many such strings have 1's for their second or fourth digits?
		      Here we must be careful not to double count.
		      Let $A =$ 8-digit strings with 1 as the fourth digit.
		      Let $B =$ 8-digit strings with 1 as the second digit.
		      Let $C =$ 8-digit strings with 1 as the second digit and 1 as the
		      fourth digit.
		      The total number of 8-digit strings with 1 as the second or fourth
		      digit is $|A| + |B| - |C| = 2^7+2^7-2^6$.
	\end{enumerate}
\end{exercise}

\begin{exercise}{}{}
	You toss a coin, then roll a dice, and then draw a card
	from a 52-card deck.
	\begin{enumerate}[label = {(\arabic*)}]
		\item Let $A = \{\text{H}, \text{T}\}$
		      Let $B = \{1, 2, 3, 4, 5, 6\}$
		      Let $C = \{1\ldots 52 \}$
		      How many different outcomes are there?
		      $|A| \cdot |B| \cdot |C| = 2 \cdot 6 \cdot 52$.
		\item How many outcomes are there in which the dice lands on $3$?
		      $|A| \cdot |C| = 2 \cdot 52$.
		      How many outcomes are there in which the dice lands on an odd number?
		      $|A| \cdot |\text{odd}| \cdot |C| = 2 \cdot 3 \cdot 52$.
		\item How many outcomes are there in which the dice lands on an odd number
		      and the card is a King?
		      $|A| \cdot |\text{odd}| \cdot |\text{King}| = 2 \cdot 3 \cdot 4$.
	\end{enumerate}
\end{exercise}

\begin{exercise}{}{}
	This problem concerns 4-letter codes made from the letters A, B, C, D,
	$\ldots$ Z.
	\begin{enumerate}[label = {(\arabic*)}]
		\item How many such codes can be made?
		      $26^4$, assuming repetitions.
		\item How many such codes have no two consecutive letters the same?
		      We can pick any letter to be the first one. To assure a non-consecutive
		      2nd letter, we can pick from 25 letters.  To pick a 3rd letter which is
		      non-consecutive with the 2nd letter, there are 25 choices, as it is now ok
		      to re-use the first letter. And so it goes, so that for every non-first
		      letter, there are $n-1$ choices.
		      Therefore the answer is $26\cdot 25^3$.
	\end{enumerate}
\end{exercise}

\begin{exercise}{}{}
	A coin is tossed 10 times in a row. How many possible
	sequences of heads and tails are there?
	$2^{10}$
\end{exercise}

\begin{exercise}{}{} A new car comes in a choice of five colors, three engine
	sizes and two transmissions. How many different combinations are there?
	$5 \cdot 3 \cdot 2$.
\end{exercise}

\begin{exercise}{}{}
	A dice is tossed four times in a row. There are many
	possible outcomes. How many different outcomes are possible?
	$6^4$
\end{exercise}

\section{Addition and Subtraction Principles}
\begin{exercise}{}{}
	Five cards are dealt off of a standard 52-card deck and
	lined up in a row.
	\begin{itemize}
		\item How many such lineups are there that have at least one red card?
		      Let $A=$ list of all lineups
		      Let $B=$ list of all lineups without a red.
		      Let $C=$ list of all lineups with at least one red.
		      Therefore $|A| - |B| = |C| = \dfrac{52!}{47!} - \dfrac{26!}{21!}$
		\item How many such lineups are there in which the cards are either all
		      black or all hearts?
		      Let $A=$ list of all blacks.
		      Let $B=$ list of all hearts.
		      Let $A \cap B= \emptyset$, because hearts are red.
		      Therefore $|A|+ |B| =\dfrac{26!}{21!} + \dfrac{13!}{8!}$.
	\end{itemize}
\end{exercise}

\begin{exercise}{}{}
	Five cards are dealt off of a standard 52-card deck and lined up in a row.
	How many such lineups are there in which all 5 cards are of the same suit?

	$4 \cdot \dfrac{13!}{8!}$
\end{exercise}

\begin{exercise}{}{}
	Five cards are dealt off of a standard 52-card deck and lined up in a row. How many such lineups are there in which all 5 cards are of the same color (i.e., all black or all red)?
	$2 \cdot \dfrac{26!}{21!}$
\end{exercise}

\begin{exercise}{}{}
	Five cards are dealt off of a standard 52-card deck and lined up in a row.
	How many such lineups are there in which exactly one of the 5 cards is a
	queen?

	$4 \cdot \dfrac{48!}{44!}$
\end{exercise}

\begin{exercise}{}{}
	Consider the integers between 1 and 9999.
	\begin{enumerate}[label = {(\arabic*)}]
		\item How many have no repeated digits?
		      Let $A=1\ldots 9$
		      Let $B=0\ldots 9$
		      For the integers $1\ldots 9$, we can pick any element from $A$, therefore
		      there are $9$ with non-consecutive digits.
		      For the integers $10\ldots 99$, we can pick any element from $A$, and then
		      and non-matching element from $B$, therefore are $9\cdot9=81$ integers in
		      this range with non-consecutive digits.
		      For the integers $C=100\ldots 999$, we can pick any element from $A$, then
		      any non-matching element from $B$, and then lastly another non-matching
		      element from $B$, therefore there are $9\cdot9\cdot9$ integers in this range
		      with non-consecutive digits.
		      For the integers $C=1000\ldots 9999$, follows the same pattern, therefore
		      there are $9\cdot9\cdot9\cdot9$ integers in this range
		      with non-consecutive digits.
		      The answer is $9^1 + 9^2 + 9^3 + 9^4$.

		\item How many have at least one repeated digit?
		      The set of numbers with at least one repeated digit is the inverse of the
		      above calculated set, namely the set of numbers with no repeated digits.
		      Therefore the answer is $9999 - 9^1 - 9^2 - 9^3 - 9^4$.
	\end{enumerate}
\end{exercise}

\begin{exercise}{}{}
	Consider lists made from the symbols A, B, C, D, E, with repetition allowed.
	\begin{enumerate}[label={}, leftmargin=*, itemsep=0pt, parsep=0pt]
		\item Let $A=\{\text{length-5 lists}\},\quad |A| = 5^5$
		\item Let $B=\{\text{length-5 lists}\mid\text{no repeats}\}, \quad |B| = 5!$
		\item Let $C=\{\text{length-5 lists}\mid\text{at least one repeat}\}$
	\end{enumerate}
	\begin{enumerate}[label = {(\alph*)}]
		\item How many such length-5 lists have at least one letter repeated?
		      Herein it's easiest to use the subtraction princple, and calculate $|A|$
		      and $|B|$ to get $|C|$.
		      $C=A - B, \quad |C| = 5^5 - 5! $
		\item How many such length-6 lists have at least one letter repeated?
		      The same principle applies, therefore the answer is $6^6 - 6!$.
	\end{enumerate}
\end{exercise}

\begin{exercise}{}{}
	A password on a certain site must be five characters long,
	made from letters of the alphabet, and have at least one upper case letter.
	\begin{enumerate}[label={}, leftmargin=*, itemsep=0pt, parsep=0pt]
		\item Let $A=\{\text{length-5 string}\},\quad |A| = 52^5$
		\item Let $B=\{\text{length-5 lists}\mid\text{no upper-case}\}, \quad |B| = 26^5$
		\item Let $C=\{\text{length-5 lists}\mid\text{at least one upper-case}\}$
	\end{enumerate}
	\begin{enumerate}[label = {(\arabic*)}]
		\item How many different passwords are there?
		      $|C| = |A| - |B| = 52^5 - 26^5.$
		\item What if there must be a mix of upper and lower case?
		      \begin{enumerate}[label={}, leftmargin=*, itemsep=0pt, parsep=0pt]
			      \item Let $D=\{\text{length-5 lists}\mid\text{no lower-case}\}, \quad |D| = 26^5$
			      \item Let $E=\{\text{length-5 lists}\mid\text{at least one lower-case, at least one lower-case}\}$
			      \item $|E| = |A| - |B| - |D| = 52^5 - 2\cdot26^5.$
		      \end{enumerate}
	\end{enumerate}
\end{exercise}

\begin{exercise}{}{}
	This problem concerns lists made from the letters A, B, C, D, E, F, G, H, I, J.
	\begin{enumerate}[label = {(\arabic*)}]
		\item How many length-5 lists can be made from these letters if
		      repetition is not allowed and the list must begin with a vowel? $3
			      \cdot 9 \cdot 8 \cdot 7 \cdot 6$.
		\item How many length-5 lists can be made from these letters if
		      repetition is not allowed and the list must begin and end with a
		      vowel? $3 \cdot 8 \cdot 7 \cdot 6 \cdot 2$.
		\item How many length-5 lists can be made from these letters if
		      repetition is not allowed and the list must contain exactly one A?
		      $5 \cdot 8 \cdot 7 \cdot 6 \cdot 5$.
	\end{enumerate}
\end{exercise}

\begin{exercise}{}{}
	Consider lists of length 6 made from the letters A, B, C,
	D, E, F, G, H. How many such lists are possible if repetition is not allowed
	and the list contains two consecutive vowels?
	\begin{enumerate}[label={}, leftmargin=*, itemsep=0pt, parsep=0pt]
		\item Let $X=\{\text{A, B, C, D, E, F, G, H}\}$
		\item Let $Y=\{\text{B, C, D, F, G, H}\}$
		\item All lists that fit the description will contain 2 vowels. First we can
		      calculate the number of ways to arrange the non-vowels from $Y$, which is
		      $6\cdot5\cdot4\cdot3$.
		\item There are then $5$ locations to insert the consecutive vowels, and 2 ways
		      to arrange the consecutive vowels, either AE or EA.
		\item Therefore there are $6\cdot5\cdot4\cdot3 \cdot 5 \cdot 2$ ways to make the
		      specified list.
	\end{enumerate}
\end{exercise}

\begin{exercise}{}{}
	Consider the lists of length six made with the symbols P, R, O, F, S,
	where repetition is allowed. (For example, the following is such a list:
	(P,R,O,O,F,S).) How many such lists can be made if the list must end in an S
	and the symbol O is used more than once?
	\begin{enumerate}[label={\textbullet}, leftmargin=*, itemsep=0pt, parsep=0pt]
		\item Since the last letter must be an S, there are only 5 spots where there are
		      different letters possible.
		\item Therefore the question is transformed into the following: How many
		      \mbox{length-five} lists can be made, with at least two Os.
		\item Let $A=\{\text{length-5 lists}\}$
		\item Let $B=\{\text{length-5 lists}\mid\text{0 O}\}$
		\item Let $C=\{\text{length-5 lists}\mid\text{1 O}\}$
		\item Let $D=\{\text{length-5 lists}\mid\text{at least 2 Os}\}$
		\item $|D| = |A| - |B| - |C|$
		\item $|A| = 5^5$, choose any of the 5 letters, 5 times
		\item $|B| = 4^5$, choose any of the 5 letters other than O, 5 times
		\item $|C| = 4^4 \cdot 5$, choose any length-4 list, and then insert an O in any
		      of the 5 spots
		\item Therefore, $|D| = |A| - |B| - |C| = 5^5 - 4^5 - 4^4 \cdot 5$
	\end{enumerate}
\end{exercise}

\begin{exercise}{}{}
	\begin{enumerate}[label = {(\arabic*)}]
		\item How many integers between 1 and 1000 are divisible by 5?
		      \begin{enumerate}[label={}, leftmargin=*, itemsep=0pt, parsep=0pt]
			      \item There are 1000 numbers in that set, and that set is equally partitioned
			            into 10 subsets by looking at the last digit.
			      \item A number is divisble by 5 if it ends in 0 or 5, which is $\dfrac{1}{5}$ of
			            the set.
			      \item Therefore 200 numbers in the set of integers from 1 to 1000 are divisible
			            \mbox{by 5}.
		      \end{enumerate}
		\item How many are not divisible by 5?
		      The rest, 800.
	\end{enumerate}
\end{exercise}

\begin{exercise}{}{}
	Six math books, four physics books and three chemistry books are
	arranged on a shelf. How many arrangements are possible if all books of the
	same subject are grouped together?
	\begin{enumerate}[label={\textbullet}, leftmargin=*, itemsep=0pt, parsep=0pt]
		\item Within this problem there exists the subproblem of how many ways books of
		      the same subject can be arranged, which is dependent of the size of the set
		      $n$.
		\item Therefore there are $n!$ ways to arrange books of the same category.
		\item Next, we must determine how many ways there are to arrange sets of similar
		      books, and again it is $m!$, where $m$ is the number of subjects.
		\item Therefore there are $m! \cdot n_1! \cdot n_2! \cdot n_3!$ ways to arrange the
		      books, with the constraint that books are grouped together by subject.
		\item Therefore there are $3! \cdot 6! \cdot 4! \cdot 3!$ ways to arrange the
		      books.
	\end{enumerate}
\end{exercise}

\section{Factorials and Permutations}
\begin{exercise}{}{}
	What is the smallest n for which n! has more than 10 digits?
	If a number is 10 digits, then it must be more than $10^{9}$. 13
\end{exercise}

\begin{exercise}{}{}
	For which values of $n$ does $n!$ have n or fewer digits?
	\begin{enumerate}[label={\textbullet}, leftmargin=*, itemsep=0pt, parsep=0pt]
		\item Let $f$ be a function return the number of digits of a number
		\item From the above question we know that $10 < f(13!) < 11$.
		\item Now $14!$, will add least one more digit, and at some point around
		      15 or 16 I think.
		\item Just use a calculator, or stirling's approximation.
	\end{enumerate}
\end{exercise}

\begin{exercise}{}{}
	How many 5-digit positive integers are there in which there are no repeated
	digits and all digits are odd?
	\begin{enumerate}[label={\textbullet}, leftmargin=*, itemsep=0pt, parsep=0pt]
		\item There are 5 digits to work with, namely the odd digits.
		\item There are $5!$ such numbers, which are positive integers, in which
		      no digit is repeated, the number is odd, and each digit is odd.
	\end{enumerate}
\end{exercise}

\begin{exercise}{}{}
	Using only pencil and paper, find the value of $\dfrac{100!}{95!}$.

	Yeah, no.
\end{exercise}

\begin{exercise}{}{}
	Using only pencil and paper, find the value of $\dfrac{120!}{118}$.

	Yeah, no.
\end{exercise}

\begin{exercise}{}{}
	There are two 0’s at the end of $10! = 3,628,800$. Using
	only pencil and paper, determine how many 0’s are at the end of the number
	100!.
	\begin{enumerate}[label={\textbullet}, leftmargin=*, itemsep=0pt, parsep=0pt]
		\item Each time the quantity $10!$ is multiplied by $10$, another $0$ will be
		      appended to the end of the number.
		\item We can factor out primes from the elements of the set 11-100, with the goal
		      of factoring out 2s and 5s, so they can multiply together to get
		      $2\cdot5=10$.
		\item Between 11 and 100, 17 numbers are divisible by 5 (15, 20, ...100), and 3
		      of them have 5 as a factor twice (25, 75, 100). Therefore we have 20 5's.
		\item It's obvious there are at least 20 2's available in the range, so in total
		      we can pair 5 and 2 together 20 times.
		\item Each time we multiply by 2 and 5, we add another zero.
		\item Therefore there will 20 be additional zeroes, for a total of 22 zeroes at the end
		      of 100!.
	\end{enumerate}
\end{exercise}

\begin{exercise}{}{}
	Find how many 9-digit numbers can be made from the digits 1, 2, 3, 4, 5, 6, 7,
	8, 9 if repetition is not allowed and all the odd digits occur first (on the left)
	followed by all the even digits (i.e., as in 137598264, but not 123456789).

	$5! \cdot 4!$ \square
\end{exercise}

\begin{exercise}{}{}
	Compute how many 7-digit numbers can be made from the digits 1, 2, 3, 4, 5, 6, 7
	if there is no repetition and the odd digits must appear in an unbroken sequence.
	(Examples: 3571264 or 2413576 or 2467531, etc., but not 7234615.)
	\begin{enumerate}[label={\textbullet}, leftmargin=*, itemsep=0pt, parsep=0pt]
		\item First let's calculate the number of ways to create the odd sequence, which
		      is $4!$.
		\item There are 3 evens, therefore there are 4 spots the odd sequence can be
		      interspersed within the evens.
		\item Therefore there are $4\cdot 4!$ to make such a number.
		      \square
	\end{enumerate}
\end{exercise}

\begin{exercise}{}{}
	How many permutations of the letters A, B, C, D, E, F, G
	are there in which the three letters ABC appear consecutively, in
	alphabetical order?
	\begin{enumerate}[label={\textbullet}, leftmargin=*, itemsep=0pt, parsep=0pt]
		\item There are 4 elements not part of the ABC sequence, therefore there are 5
		      spots to insert the ABC sequence.
		\item There the non ABC elements can be arranged in $4!$ permutations.
		\item Therefore there are $4\cdot4!$ ways to make such a sequence.
	\end{enumerate}
\end{exercise}

\begin{exercise}{}{}
	How many permutations of the digits 0, 1, 2, 3, 4, 5, 6, 7, 8, 9 are there
	in which the digits alternate even and odd? (For example, 2183470965.)
	\begin{enumerate}[label={\textbullet}, leftmargin=*, itemsep=0pt, parsep=0pt]
		\item The permutation can start 2 ways, either with an odd or an even.
		\item There 5 possibilties for the 1st spot and 2nd spot, 4 for the 3rd and 4th
		      spot, and so on.
		\item Therefore there are $2\cdot 5!\cdot5!$. possibilities. \square
	\end{enumerate}
\end{exercise}

\begin{exercise}{}{}
	You deal 7 cards off of a 52-card deck and line them up in
	a row. How many possible lineups are there in which not all cards are red?
	\begin{enumerate}[label={\textbullet}, leftmargin=*, itemsep=0pt, parsep=0pt]
		\item Let $A=$ set of all 7 card lineups
		\item Let $B=$ set of all red-only 7 card lineups
		\item Let $C=$ set of all lineups in which not all the cards are red
		\item $|A| - |B| = |C|$
		\item $|A|=\dfrac{52!}{45!}$
		\item $|B|=\dfrac{13!}{6!}$
		\item $|C|=\dfrac{52!}{45!} - \dfrac{13!}{6!}$ \square
	\end{enumerate}
\end{exercise}

\begin{exercise}{}{}
	You deal 7 cards off of a 52-card deck and line them up in
	a row. How many possible lineups are there in which no card is a club?
	\begin{enumerate}[label={\textbullet}, leftmargin=*, itemsep=0pt, parsep=0pt]
		\item For there to be no clubs, we can imagine starting with a deck of 39
		      cards without any clubs.
		\item Therefore the answer is $\dfrac{39!}{32!}$. \square
	\end{enumerate}
\end{exercise}

\begin{exercise}{}{}
	How many lists of length six (with no repetition) can be
	made from the 26 letters of the English alphabet?

	$\dfrac{26!}{20!}$\square
\end{exercise}

\begin{exercise}{}{}
	Five of ten books are arranged on a shelf. In how many ways can this be done?

	Assuming that different orderings of books are considered different
	arrangements, there are $\dfrac{10!}{5!}$ possible arrangements. \square
\end{exercise}

\begin{exercise}{}{}
	In a club of 15 people, we need to choose a president,
	vice-president, secretary, and treasurer. In how many ways can this be done?
	\begin{enumerate}[label={\textbullet}, leftmargin=*, itemsep=0pt, parsep=0pt]
		\item There are 4 different positions, and the order matters, as the positions
		      are not equivalent.
		\item Therefore there are $\dfrac{15!}{11!}$ distinct possible
		      arrangements. \square
	\end{enumerate}
\end{exercise}

\begin{exercise}{}{}
	How many 4-permutations are there of the set A,B,C,D,E,F
	if whenever A appears in the permutation, it is followed by E?
	\begin{enumerate}[label={\textbullet}, leftmargin=*, itemsep=0pt, parsep=0pt]
		\item Let $X=$ 4-permutations with A.
		\item Let $Y=$ 4-permutations without A.
		\item Let $Z=$ all 4-permutations matching the specification.
		\item For the set X, AE will always appear together. Because there are 2
		      remaining elements, there are 3 spots where AE can in interspersed.
		\item Therefore $|X| = 2 \cdot 4\cdot3 = 4!.$
		\item For the set Y, A will not appear by exercise, therefore $|Y| =
			      \dfrac{5!}{1!} = 5!$.
		\item Therefore $|Z| = |X| + |Y| = 4! + 5!$. \square
	\end{enumerate}
\end{exercise}

\begin{exercise}{}{}
	Three people in a group of ten line up at a ticket counter to buy tickets.
	How many lineups are possible?

	$\dfrac{10!}{7!}$
\end{exercise}

\begin{exercise}{}{}
	The gamma function provides a way of extending factorials to numbers
	other than integers. Extra credit: Compute $\pi$ !.

	\begin{align*}
		\Gamma                    & :[0, \infty) \rightarrow \mathbb{R} \\
		\Gamma(x)                 & =\int_0^{\infty} t^{x-1} e^{-t} d t \\
		x \in \mathbb{N}          & \implies \Gamma(x)=(x-1)!           \\
		\forall n \in \mathbb{N}, & \quad n !=\Gamma(n+1)
	\end{align*}

	Check that this is true for $x=1,2,3,4$.
\end{exercise}

\part{How to Prove Conditional Statements}
\part{More on Proof}
\part{Relations, Functions, and Cardinality}
% \printbibliography{}
% \printindex{}
\end{document}
