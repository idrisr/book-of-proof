\makeindex

\usepackage{amsmath}
\usepackage{amssymb}
\usepackage{booktabs}
\usepackage{csvsimple-l3}
\usepackage{bussproofs}
\usepackage{dirtytalk}
\usepackage[dvipsnames]{xcolor}
\usepackage{enumitem}
\usepackage{epigraph}
\usepackage{forest}
\usepackage{formal-grammar}
\usepackage{graphicx}
\usepackage[citecolor=blue,colorlinks=true, linkcolor=blue, urlcolor=blue]{hyperref}
\usepackage{kantlipsum}
\usepackage{makeidx}
\usepackage[margin=0.8in]{geometry}
\usepackage{mathrsfs}
\usepackage{minted}
\usepackage{multicol}
\usepackage{standalone}
\usepackage[style=authortitle]{biblatex}
\usepackage[T1]{fontenc}
\usepackage[tableaux]{prooftrees}
\usepackage{tcolorbox}
\usepackage{tikz}
\usepackage{titlesec}
\usepackage{xcolor}

\usetikzlibrary{arrows}
\usetikzlibrary{arrows.meta}
\usetikzlibrary{automata}
\usetikzlibrary{calc}
\usetikzlibrary{fit}
\usetikzlibrary{petri}
\usetikzlibrary{positioning}

\tcbuselibrary{breakable}
\tcbuselibrary{listings}
\tcbuselibrary{minted}
\tcbuselibrary{skins}
\tcbuselibrary{theorems}

\newcounter{filePrg}

\addbibresource{biblio.bib}
\setlength{\parindent}{0pt}

\renewcommand{\emph}[1]{\textit{#1}}
\setlength{\parindent}{0pt}

\newcommand\setboxcounter[2]{\setcounter{tcb@cnt@#1}{#2}}
\setlength{\parindent}{10pt}
\newcommand{\set}[1]{\{#1\}}
\newcommand{\up}[1]{ \left\lceil#1\right\rceil }
\newcommand{\down}[1]{ \left\lfloor#1\right\rfloor}

\definecolor{CaribbeanBlue}{RGB}{0, 206, 209} % Define Caribbean Blue
\NewTcbTheorem[list inside=definition]{definition}
{Definition}{
	breakable,
	colback=CaribbeanBlue!05,
	colframe=CaribbeanBlue!35!black,
	fonttitle=\bfseries}{th}

\NewTcbTheorem[list inside=intuition]{intuition}{Intuition}{
	breakable,
	colback=blue!5,
	colframe=blue!35!black,
	fonttitle=\bfseries}{th}

\NewTcbTheorem{example}{Example}{
	breakable,
	colback=white,
	colframe=green!35!black,
	fonttitle=\bfseries}{th}

\NewTcbTheorem{verify}{Verify}{
	breakable,
	float,
	colback=red!5,
	colframe=red!35!black,
	fonttitle=\bfseries}{th}

\NewTcbTheorem[list inside=theorem]{theorem}{Theorem}{
	breakable,
	colback=gray!10,
	colframe=gray!35!black,
	fonttitle=\bfseries}{th}

% \NewTcbTheorem[
% list inside=exercise,
% number within=chapter,
% number within=section,
% ]

\NewTcbTheorem[
	list inside=exercise,
	number within=section,
]{exercise}{Exercise}{
	breakable,
	colback=white,
	colframe=black,
	fonttitle=\bfseries}{th}

\newcommand{\hask}[1]{\mintinline{haskell}{#1}}

\newenvironment{alist}
{\begin{enumerate}[label={*}, leftmargin=*, itemsep=0pt, parsep=0pt]}
		{\end{enumerate}}

\renewcommand{\thesection}{\arabic{section}}
\tcbset{enhanced jigsaw}

\newtcbinputlisting{\codeFromFile}[2]{
	listing file={#1},
	listing engine=minted,
	minted style=colorful,
	minted language=haskell,
	minted options={breaklines,linenos,numbersep=3mm},
	colback=blue!5!white,colframe=blue!75!black,listing only,
	left=5mm,enhanced,
	title={#2},
	overlay={\begin{tcbclipinterior}\fill[red!20!blue!20!white] (frame.south west)
				rectangle ([xshift=5mm]frame.north west);\end{tcbclipinterior}}
}

\newtcblisting{haskell}[1]
{
	listing engine=minted,
	minted style=colorful,
	minted language=haskell,
	minted options={breaklines,linenos,numbersep=3mm},
	colback=blue!5!white,colframe=blue!75!black,listing only,
	left=5mm,enhanced,
	title={#1},
	overlay={\begin{tcbclipinterior}\fill[red!20!blue!20!white] (frame.south west)
				rectangle ([xshift=5mm]frame.north west);\end{tcbclipinterior}}
}
